%\magnification=\magstep1
\documentstyle{amsppt}

\topmatter
\parskip=\baselineskip
\baselineskip=1.1\baselineskip  % was 1.1
\parindent=0pt
\loadmsbm
\UseAMSsymbols
\raggedbottom
\hoffset=0.75truein
\voffset=0.5truein


\def\reft{\relax}  % items
\def\refz{\relax}
\def\bul{\noindent$\bullet$\quad }
\def\lb#1{\line{$\bullet$ #1 \hfill}}
\def\cir{\noindent$\circ$\quad }
\def\heads#1{\rightheadtext{#1}}
\def\doct{\delta_{oct}}
\def\pt{\hbox{\it pt}}
\def\Vol{\hbox{vol}}
\def\and{\operatorname{and}}
\def\sol{\operatorname{sol}}
\def\quo{\operatorname{quo}}
\def\anc{\operatorname{anc}}
\def\cro{\operatorname{crown}}
\def\vor{\operatorname{vor}}
\def\octavor{\operatorname{octavor}}
\def\dih{\operatorname{dih}}
\def\arc{\operatorname{arc}}
\def\rad{\operatorname{rad}}
\def\if{\operatorname{if}}
\def\A{{\bold A}}
\def\squander{(4\pi\zeta-8)\,\pt}
\def\score{8\,\pt}
\def\Sfour{\Cal{\bold S}_4^+}
\def\Sminus{\Cal{\bold S}_3^-}
\def\Splus{\Cal{\bold S}_3^+}
\def\maxpi{\pi_{\max}}
\def\xiG{\xi_\Gamma}
\def\xiV{\xi_V}
\def\xik{\xi_\kappa}
\def\xikG{\xi_{\kappa,\Gamma}}

\font\twrm=cmr8
\def\DLP{\operatorname{D}_{\hbox{\twrm LP}}}
\def\ZLP{\operatorname{Z}_{\hbox{\twrm LP}}}
\def\tLP{\operatorname{\hbox{$\tau$}LP}}  % 3 args small LP.
\def\tlp{\tau_{\hbox{\twrm LP}}}  % 2 args (p,q) tri, quad
\def\slp{\sigma_{\hbox{\twrm LP}}}  % 2 args (p,q) tri, quad
\def\sLP{\operatorname{\hbox{$\sigma$}LP}}
\def\LPmin{\operatorname{LP-min}}
\def\LPmax{\operatorname{LP-max}}
\def\geom{{\operatorname{g}}}
\def\anal{{\operatorname{an}}}



\def\sol{\operatorname{sol}}
\def\dih{\operatorname{dih}}
\def\V{\operatorname{V}}
\def\vol{\operatorname{vol}}
\def\Area{\operatorname{Area}}
\def\Per{\operatorname{Per}}
\def\rad{\operatorname{rad}}
\def\quo{\operatorname{quo}}

\def\R{{\Bbb R}}
\def\ldot{\cdot}
\def\S{{\Cal S}}
\def\del{\partial}
\def\B#1{{\bold #1}}
\def\tri#1#2{\langle#1;#2\rangle}
\def\ha{ \hangindent=20pt \hangafter=1\relax }
\def\ho{ \hangindent=20pt \hangafter=0\relax }
\def\i{I}
\def\was#1{\relax}

\def\diag|#1|#2|{\vbox to #1in {\vskip.3in\centerline{\tt Diagram #2}\vss} }
\def\v{\hskip -3.5pt }
% to place a fig file  align the bottom left corner at (0,-11).
% save as left justified.
% resize so that the box is just under 5 in width
\def\gram|#1|#2|#3|{
        {
        \smallskip
        \hbox to \hsize
        {\hfill
        \vrule \vbox{ \hrule \vskip 6pt \centerline{\it Diagram #2}
         \vskip #1in %
             \special{psfile=#3 hoffset=5 voffset=5 }\hrule }
        \v\vrule\hfill
        }
\smallskip}}

\def\today{\ifcase\month\or
    January\or February\or March\or April\or May\or June\or
    July\or August\or September\or October\or November\or December\fi
    \space\number\day, \number\year}




\bigskip

\centerline{\bf Appendix 1. Inequalities}

\bigskip

{\bf 1. Tetrahedral Inequalities.}

\smallskip

The inequalities in Sections 1-3 apply only to quasi-regular tetrahedra.
The edges are numbered in the usual way.

\smallskip

{\bf 1.1 Function Bounds on Quasi-Regular Tetrahedra.}


$1:  \vol(S) > 0.202804$\newline
$2:  \sol(S) > 0.315696$\newline
$3:  \sol(S) < 1.051232$\newline
$4:  \dih(S) > 0.856147$\newline
$5:  \dih(S) < 1.886730$\newline
$6:  \mu(S) > 0$\newline


\smallskip

{\bf 1.2. Vol, Sol, Dih inequalities on Quasi-Regular Tetrahedra.}

\smallskip

$1:  \vol(S) - 0.68 \sol(S) + 1.88718 \dih(S) > 1.54551 $\newline
$2:  \vol(S) - 0.68 \sol(S) + 0.90746 \dih(S) > 0.706725$\newline
$3:  \vol(S) - 0.68 \sol(S) + 0.46654 \dih(S) > 0.329233$\newline
$4:  \vol(S) - 0.55889 \sol(S) - 0 \dih(S) > -0.0736486$\newline
$5:  \vol(S) - 0.63214 \sol(S) - 0 \dih(S) > -0.13034$\newline
$6:  \vol(S) - 0.73256 \sol(S) - 0 \dih(S) > -0.23591$\newline
$7:  \vol(S) - 0.89346 \sol(S) - 0 \dih(S) > -0.40505$\newline
$8:  \vol(S) - 0.3 \sol(S) - 0.5734 \dih(S) > -0.978221$\newline
$9:  \vol(S) - 0.3 \sol(S) - 0.03668 \dih(S) > 0.024767$\newline
$10:  \vol(S) - 0.3 \sol(S) + 0.04165 \dih(S) > 0.121199$\newline
$11:  \vol(S) - 0.3 \sol(S) + 0.1234 \dih(S) > 0.209279$\newline
$12:  \vol(S) - 0.42755 \sol(S) - 0.11509 \dih(S) > -0.171859$\newline
$13:  \vol(S) - 0.42755 \sol(S) - 0.04078 \dih(S) > -0.050713$\newline
$14:  \vol(S) - 0.42755 \sol(S) + 0.11031 \dih(S) > 0.135633$\newline
$15:  \vol(S) - 0.42755 \sol(S) + 0.13091 \dih(S) > 0.157363$\newline
$16:  \vol(S) - 0.55792 \sol(S) - 0.21394 \dih(S) > -0.417998$\newline
$17:  \vol(S) - 0.55792 \sol(S) - 0.0068 \dih(S) > -0.081902$\newline
$18:  \vol(S) - 0.55792 \sol(S) + 0.0184 \dih(S) > -0.051224$\newline
$19:  \vol(S) - 0.55792 \sol(S) + 0.24335 \dih(S) > 0.193993$\newline
$20:  \vol(S) - 0.68 \sol(S) - 0.30651 \dih(S) > -0.648496$\newline
$21:  \vol(S) - 0.68 \sol(S) - 0.06965 \dih(S) > -0.278$\newline
$22:  \vol(S) - 0.68 \sol(S) + 0.0172 \dih(S) > -0.15662$\newline
$23:  \vol(S) - 0.68 \sol(S) + 0.41812 \dih(S) > 0.287778$\newline
$24:  \vol(S) - 0.64934 \sol(S) - 0 \dih(S) > -0.14843$\newline
$25:  \vol(S) - 0.6196 \sol(S) - 0 \dih(S) > -0.118$\newline
$26:  \vol(S) - 0.58402 \sol(S) - 0 \dih(S) > -0.090290$\newline
$27:  \vol(S) - 0.25181 \sol(S) - 0 \dih(S) > 0.096509$\newline
$28:  \vol(S) - 0.00909 \sol(S) - 0 \dih(S) > 0.199559$\newline
$29:  \vol(S) + 0.93877 \sol(S) - 0 \dih(S) > 0.537892$\newline
$30:  \vol(S) + 0.93877 \sol(S) - 0.20211 \dih(S) > 0.27313$\newline
$31:  \vol(S) + 0.93877 \sol(S) + 0.63517 \dih(S) > 1.20578$\newline
$32:  \vol(S) + 1.93877 \sol(S) - 0 \dih(S) > 0.854804$\newline
$33:  \vol(S) + 1.93877 \sol(S) - 0.20211 \dih(S) > 0.621886$\newline
$34:  \vol(S) + 1.93877 \sol(S) + 0.63517 \dih(S) > 1.57648$\newline
$35:  \vol(S) - 0.42775 \sol(S) - 0 \dih(S) > -0.000111$\newline
$36:  \vol(S) - 0.55792 \sol(S) - 0 \dih(S) > -0.073037$\newline
$37:  \vol(S) - 0 \sol(S) - 0.07853 \dih(S) > 0.08865$\newline
$38:  \vol(S) - 0 \sol(S) - 0.00339 \dih(S) > 0.198693$\newline
$39:  \vol(S) - 0 \sol(S) + 0.18199 \dih(S) > 0.396670$\newline
$40:  \vol(S) - 0.42755 \sol(S) - 0.2 \dih(S) > -0.332061$\newline
$41:  \vol(S) - 0.3 \sol(S) - 0.36373 \dih(S) > -0.58263$\newline
$42:  \vol(S) - 0.3 \sol(S) + 0.20583 \dih(S) > 0.279851$\newline
$43:  \vol(S) - 0.3 \sol(S) + 0.40035 \dih(S) > 0.446389$\newline
$44:  \vol(S) - 0.3 \sol(S) + 0.83259 \dih(S) > 0.816450$\newline
$45:  \vol(S) - 0.42755 \sol(S) - 0.51838 \dih(S) > -0.932759$\newline
$46:  \vol(S) - 0.42755 \sol(S) + 0.29344 \dih(S) > 0.296513$\newline
$47:  \vol(S) - 0.42755 \sol(S) + 0.57056 \dih(S) > 0.533768$\newline
$48:  \vol(S) - 0.42755 \sol(S) + 1.18656 \dih(S) > 1.06115$\newline
$49:  \vol(S) - 0.55792 \sol(S) - 0.67644 \dih(S) > -1.29062$\newline
$50:  \vol(S) - 0.55792 \sol(S) + 0.38278 \dih(S) > 0.313365$\newline
$51:  \vol(S) - 0.55792 \sol(S) + 0.74454 \dih(S) > 0.623085$\newline
$52:  \vol(S) - 0.55792 \sol(S) + 1.54837 \dih(S) > 1.31128$\newline
$53:  \vol(S) - 0.68 \sol(S) - 0.82445 \dih(S) > -1.62571$\newline
\smallskip

{\bf 1.3. Edge length inequalities on Quasi-Regular Tetrahedra.}

\smallskip

$1: \sol(S) > 0.551285 - 0.245 (y_1+y_2+y_3-6) + 0.063 (y_4+y_5+y_6-6)$\newline
$2: \sol(S) > 0.551285 - 0.3798 (y_1+y_2+y_3-6) + 0.198 (y_4+y_5+y_6-6)$\newline
$3: \sol(S) < 0.551286 - 0.151 (y_1+y_2+y_3-6) + 0.323 (y_4+y_5+y_6-6)$\newline

$4: \mu(S) > 0.0392 (y_1+y_2+y_3-6) + 0.0101 (y_4+y_5+y_6-6) $\newline
$5: \vol > 0.235702 -0.107 (y_1+y_2+y_3-6) + 0.116 (y_4+y_5+y_6-6)$\newline
$6: \vol > 0.235702 -0.0623 (y_1+y_2+y_3-6) + 0.0722 (y_4+y_5+y_6-6)$\newline
$7: \dih(S) > 1.23095 + 0.237 (y_1-2) - 0.372 (y_2+y_3+y_5+y_6-8) + 0.708 (y_4-2) $\newline
$8: \dih(S) > 1.23095 + 0.237 (y_1-2) - 0.363 (y_2+y_3+y_5+y_6-8) + 0.688 (y_4-2)$\newline
$9: \dih(S) < 1.23096 + 0.505 (y_1-2) - 0.152(y_2+y_3+y_5+y_6-8) + 0.766 (y_4-2)$\newline



\smallskip

{\bf  2.Quadrilateral Inequalities}

\smallskip

We define $Q(x_1,x_2,x_3,x_4,x_5,x_6,x_7,x_8,x_9)$, abreviated as Q, to be a quad cluster with edge lengths $x_1,\dots, x_9$ with the following order on the edges.  The edges $x_1,\dots, x_6$ are ordered the same as a tetrahedron.  (Note: $x_4$ corresponds to a diagonal of the quadrilateral.  The two vertices which are connected by $x_4$ are {\sl NOT} close neighbors.) The edge $x_7$ refers to the vector connecting 0 and the 4th sphere.  Edge $x_8,x_9$ are opposite $x_2,x_3$ respectively.

What follow are the quad cluster equivalent of the tetrahedral inequalities.
The function $dih_1(Q)$ refers to the dihedral angle at edge 1 or 7.  Similarly, $dih_2(Q)$ refers to the dihedral angle at edge 2 or 3.  If simply $dih(Q)$ is used, the inequality holds for any choice of dihedral angle.  (Note that a quadrilateral is always truncated so we use $\omega$ consistently as the volume function.)

{\bf 2.1. Function Bounds on Quadrilateral Clusters.}

$1:  \omega(Q) > 0.455149$\newline
$2:  \sol(Q) > 0.731937$\newline
$3:  \sol(Q) < 2.85860$\newline
$4:  \dih(Q) > 1.15242$\newline
$5:  \dih(Q) < 3.25887$\newline
$6:  \mu(Q) > 0.031350$\newline

\smallskip

{\bf 2.2. Vol,Sol,Dih inequalities on Quad Clusters.}

\smallskip

$1:  \omega(Q) - 0.42775 \sol(Q) - 0.15098 \dih(S) > -0.3670$\newline
$2:  \omega(Q) - 0.42775 \sol(Q) - 0.09098 \dih(S) > -0.1737$\newline
$3:  \omega(Q) - 0.42775 \sol(Q) - 0.00000 \dih(S) > 0.0310$\newline
$4:  \omega(Q) - 0.42775 \sol(Q) + 0.18519 \dih(S) > 0.3183$\newline
$5:  \omega(Q) - 0.42775 \sol(Q) + 0.20622 \dih(S) > 0.3438$\newline
$6:  \omega(Q) - 0.55792 \sol(Q) - 0.30124 \dih(S) > -1.0173$\newline
$7:  \omega(Q) - 0.55792 \sol(Q) - 0.02921 \dih(S) > -0.2101$\newline
$8:  \omega(Q) - 0.55792 \sol(Q) - 0.00000 \dih(S) > -0.1393$\newline
$9:  \omega(Q) - 0.55792 \sol(Q) + 0.05947 \dih(S) > -0.0470$\newline
$10:  \omega(Q) - 0.55792 \sol(Q) + 0.39938 \dih(S) > 0.4305$\newline
$11:  \omega(Q) - 0.55792 \sol(Q) + 2.50210 \dih(S) > 2.8976$\newline
$12:  \omega(Q) - 0.68000 \sol(Q) - 0.44194 \dih(S) > -1.6264$\newline
$13:  \omega(Q) - 0.68000 \sol(Q) - 0.10957 \dih(S) > -0.6753$\newline
$14:  \omega(Q) - 0.68000 \sol(Q) - 0.00000 \dih(S) > -0.4029$\newline
$15:  \omega(Q) - 0.68000 \sol(Q) + 0.86096 \dih(S) > 0.8262$\newline
$16:  \omega(Q) - 0.68000 \sol(Q) + 2.44439 \dih(S) > 2.7002$\newline
$17:  \omega(Q) - 0.30000 \sol(Q) - 0.12596 \dih(S) > -0.1279$\newline
$18:  \omega(Q) - 0.30000 \sol(Q) - 0.02576 \dih(S) > 0.1320$\newline
$19:  \omega(Q) - 0.30000 \sol(Q) + 0.00000 \dih(S) > 0.1945$\newline
$20:  \omega(Q) - 0.30000 \sol(Q) + 0.03700 \dih(S) > 0.2480$\newline
$21:  \omega(Q) - 0.30000 \sol(Q) + 0.22476 \dih(S) > 0.5111$\newline
$22:  \omega(Q) - 0.30000 \sol(Q) + 2.31852 \dih(S) > 2.9625$\newline
$23:  \omega(Q) - 0.23227 \dih(S) > -0.1042$\newline
$24:  \omega(Q) + 0.07448 \dih(S) > 0.5591$\newline
$25:  \omega(Q) + 0.22019 \dih(S) > 0.7627$\newline
$26:  \omega(Q) + 0.80927 \dih(S) > 1.5048$\newline
$27:  \omega(Q) + 5.84380 \dih(S) > 7.3468$\newline



\smallskip

{\bf 2.3. Edge Length Inequalities on Quad Clusters.}

This inequality will be called (*) and is not used in the general linear programming problems.  It is applied as needed to disprove counterexamples.  It is 
difficult to prove and will be discussed whenever it is used.

\smallskip

$(*): \omega(Q) > 0.590491 -0.166(y_1+y_2+y_3+y_7-8) + 0.143 (y_5+y_6+y_8+y_9-8)$\newline

\smallskip


\bye
