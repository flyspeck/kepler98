%\magnification=\magstep1
\documentstyle{amsppt}
\topmatter

\parskip=0.5\baselineskip
\baselineskip=1.1\baselineskip
\parindent=0pt
\loadmsbm
\UseAMSsymbols
%\raggedbottom
\hoffset=0.75truein
\voffset=0.5truein

\font\twrm=cmr8
\def\doct{\delta_{oct}}
\def\pt{\hbox{\it pt}}
\def\Vol{\hbox{vol}}
\def\sol{\operatorname{sol}}
\def\dih{\operatorname{dih}}
\def\vor{\operatorname{vor}}
\def\quo{\operatorname{quo}}
\def\octavor{\operatorname{octavor}}
\def\R{{\Bbb R}}
\def\ldot{\cdot}
\def\S{{\Cal S}}
\def\rad{\operatorname{rad}}
%\def\qed{{\it q.e.d.\/}}
\def\ast{\hbox{\tt **}}
\def\sq{\sqrt{2}}
\def\refy{\relax}
\def\tlp{\tau_{\hbox{\twrm LP}}}  % 2 args (p,q) tri, quad
\def\slp{\sigma_{\hbox{\twrm LP}}}  % 2 args (p,q) tri, quad

%\def\qed{{\hbox{}\nobreak\hfill\vrule height8pt width6pt depth 0pt}\medskip}
\def\diag|#1|#2|{\vbox to #1in {\vskip.3in\centerline{\tt Diagram #2}\vss} }
\def\v{\hskip -3.5pt }
\def\gram|#1|#2|#3|{
        {
        \smallskip
        \hbox to \hsize
        {\hfill
        \vrule \vbox{ \hrule \vskip 6pt \centerline{\it Diagram #2}
         \vskip #1in %
             \special{psfile=#3 hoffset=5 voffset=5 }\hrule }
        \v\vrule\hfill
        }
\smallskip}}

\def\x#1{\hbox{\vbox{\hrule width #1 cm height 0.5pt}}}
\def\heads#1{\leftheadtext{#1}\rightheadtext{#1}}

\title Sphere Packings III\endtitle
\author Thomas C. Hales\endauthor
\endtopmatter

\abstract{Abstract}
An earlier paper describes a five-step program to prove the Kepler
Conjecture.  This paper carries out the third step of the program.
\endabstract

\document
\footnote""{\line{\hfill\it version - 3/14/98}}
\footnote""{The results of this paper depend on an inequality
	that has not yet been rigorously established by interval
	arithmetic, the octahedral case of Inequality 4.2.3.}
	

\head 1. Introduction and review\endhead
%\heads{1. Introduction}

This paper is a continuation of the first two parts of this
series \cite{I},\cite{II}. It relies on the formulation of the 
Kepler conjecture in \cite{F}. The terminology and notation of this
paper are consistent with these earlier papers, and we refer
to results from them by prefixing the relevant
section numbers with I, II, or F. Around each vertex are
is a modification of the Voronoi cell, called the $V$-cell and
a collection of quarters and quasi-regular tetrahedra.
These objects comprise the {\it decomposition star\/} at the vertex.
 A decomposition star may be decomposed into {\it
standard clusters}.
By definition, a
standard cluster is the collection
of standard simplices 
in a given decomposition star
that lie over a given {\it standard region\/}
on the unit sphere.  

A real number, called the {\it score},
 is attached to each cluster.  
Each star
receives a score by summing the scores $\sigma(R)$ for the clusters 
$R$ in the
star.   The scores are measured in multiples of a {\it point\/}
(\pt), where
$\pt \approx 0.055$.  If every star scores at most $8\,\pt$,
then the Kepler conjecture follows.

The steps of the Kepler conjecture, as outlined in Part I, are

{

\def\ha{ \hangindent=20pt \hangafter=1\relax }
1. A proof that if all standard
regions are triangular, the total score
is less than $8\,\pt$

\ha
2.  A proof that the standard regions
with more than
three sides
score at most $0\,\pt$

\ha
3. A proof that if all of the
standard regions are triangles or quadrilaterals,
then the total score is less than $8\,\pt$ (excluding the
case of pentagonal prisms)


\ha
4.  A proof that if some
standard region has
more than four sides, then the
star scores less than $8\,\pt$


\ha
5.  A proof that pentagonal prisms score less than $8\,\pt$

}

The proofs of steps I, II, and V are complete \cite{I}, \cite{II}, \cite{V}.
This paper completes III.

The standard regions of the decomposition
stars in the face-centered cubic and the hexagonal-close packings
are regular
triangles and quadrilaterals.  These stars score exactly
$8\,\pt$.  The local optimality results for hexagonal-close packings
and face-centered cubic packings have been established in \cite{II}.
If the planar map is that of a close-packing, there are eight quasi-regular
tetrahedra.  To score $8\,\pt$, the quasi-regular tetrahedra must
be regular of edge length 2 \cite{I.9.1}.  Such decomposition stars
are precisely those of the close packings.  Conjecturally, 
all the decomposition stars have scores strictly less than $8\,\pt$,
so that various approximations may be introduced to prove the desired
bounds.

The standard regions of pentagonal prisms are
triangles and quadrilaterals (10 triangles and 5
quadrilaterals).  The pentagonal prisms are the subject of
the fifth step of the program.  This paper classifies
the possible combinatorial types of
counterexamples to the third step of the Kepler
conjecture and lists them in Appendix I.

\proclaim{Theorem 1} 
 Let $D^*$ be a decomposition star whose combinatorial
structure is not a pentagonal prism.
Suppose that
each standard region of $D^*$ is a triangle or quadrilateral.
Then the score
of $D^*$ is at most $8\,\pt$.  Equality is attained exactly
when the decomposition of the unit sphere into standard regions
coincides with the decomposition determined by a decomposition star in
the face-centered cubic or hexagonal-close packing.
\endproclaim

The proof of Theorem 1 relies on many computer calculations.
We make a list of combinatorial properties that a decomposition
star must have for it to have a possibility of scoring more
than $8\,\pt$.  We then make a computer
search to find all decompositions of the unit sphere into
triangles and quadrilaterals that satisfy all the properties
on the list.  The computer search produces an explicit
list containing nearly two thousand 
combinatorial types. 

For each of these combinatorial arrangements 
of triangles and quadrilaterals,
we have a nonlinear optimization problem:  maximize the
score over the space of all decomposition stars $D^*$ with
the given combinatorial arrangement.
It is not necessary to solve this optimization problem. It is
sufficient to establish an upper bound of $8\,\pt$.  To do this,
we define a {\it linear relaxation\/} of the original problem,
that is, a linear programming 
problem whose solution strictly dominates the
global maximum of the original nonlinear problem. 
This gives an upper
bound on the linear problem,  which is usually less than 
$8\,\pt$.

In some cases, this procedure leads to a bound
greater than $8\,\pt$, and further analysis will be required.

One advantage of our method of linear relaxation
is that the verification of
the bounds is particularly simple.  If the linear relaxation
asks to maximize $c\cdot x$ subject to the system of linear inequalities
$Ax\le b$ and  $x\ge0$, then duality theory produces a vector $z$
with nonnegative entries such that $c\le zA$.  
To verify the bound of $8\,\pt$, it is enough to
check that  $c\le zA$, $z\ge0$, and  $z\cdot b<8\,\pt$ (because
then $c\cdot x \le z A x \le z\cdot b < 8\,\pt$).

\medskip
Most of the linear inequalities that were used in the
linearly relaxed optimization problems are obtained as
follows.  We use numerical methods to find a convex
polygon containing the set of ordered pairs
$$(\dih(S), \sigma(S))$$
as $S$ ranges over quasi-regular tetrahedra.  
  The edges of the polygon represent linear inequalities
relating the dihedral angles to the score $\sigma$.  
More generally, additional inequalities are obtained by considering
polygons that contain the ordered pairs
$$(\dih(S),\sigma(S)-\lambda\sol(S)),$$
for appropriate constants $\lambda$.   Similar inequalities
are obtained for quadrilateral regions.

We must remember that ultimately
we are dealing with a nonlinear optimization problem
that is larger, by a considerable order,
than what is conventionally thought to
be solvable by exact methods.  The domain of our
optimization problem has many components, 
and the dimension varies from component
to component. Even the magnitude of the problem is poorly understood.  The
best-known bound on the 
dimension of the components is about 155 dimensions.  Components
of interest
frequently have more than 35 dimensions.
The optimization problem does not
have any of the special properties such as linearity, convexity,
 or a quadratic
structure assumed by some of the more popular
techniques of global optimization.  The decomposition into standard
regions gives the problem an aspect of separability.
Nevertheless, certain complications will have
to be tolerated.

The biggest weakness of this method is that the output from
the computer search for the combinatorial arrangements of
triangles and quadrilaterals is not easily checked for errors.  
The algorithm is described in some detail in Section 9, but the
only assurance that no cases have been skipped comes through a
careful reading of the computer code.
It would be advantageous to have a more transparent proof of the
results of this section.

This paper is supplemented by an
appendix giving further details of the combinatorial arrangements.
Further
details about these calculations, including the full source
code for all of the computer verifications of this paper, can be
found in \cite{H2}.

The outline of this paper was developed at the University of Chicago
during the summer of 1994.  I would like to 
thank to P. Sally for making computer resources and
other facilities available to me during my stay in Chicago,
and for his encouragement with this work. 
I would also like to give special thanks S. Ferguson for many helpful
discussions concerning this topic.   His investigations have led
to a number of improvements in the results presented here.
Many of the inequalities were checked numerically using a nonlinear
optimization package {\tt cfsqp}
\footnote"*"{\tt \quad www.isr.umd.edu/Labs/CACSE/FSQP/fsqp.html}
 before they were rigorously
established by interval methods.  I thank the University of Maryland
for this software.


\head 2. Geometric considerations\endhead
%\heads{2. Geometric considerations}

\bigskip
\subhead 2.1\endsubhead
We will call a standard cluster over a quadrilateral
region a {\it quad cluster}.  
The four vertices of the quad cluster whose projections to the
unit sphere mark the
extreme points of the quadrilateral region will be called
the {\it corners\/} of the cluster.
We call the four angles of the
standard region associated with the quad cluster its {\it dihedral\/}
angles.  

The rules defining the score
have undergone a long series of revisions
over the last several years.  The formulation used in this
paper is described in \cite{F}.

\proclaim {Lemma 2.2}
A quadrilateral region does not enclose any
vertices of height at most $2.51$.
\endproclaim

\demo{Proof}
Let $v_1,\ldots,v_4$ be the corners of the quad cluster, and
let $v$ be an enclosed vertex of height at most $2.51$.
We cannot have $|v_i-v|\le2.51$ for two different vertices
$v_i$,  because two
such inequalities would partition the region into
 two separate
standard regions instead of a single quadrilateral region.
We apply I.4.3 to simplify the quad cluster.  (Lemma I.4.3 assumes
the existence of another enclosed vertex $v'$, but it can be
omitted from both the statement fo the Lemma and in the proofs without
affecting matters.)  Then I.4.3 allows us to assume
$$|v_i-v_{i+1}|=2.51,\quad |v_i|=2, \quad |v|=2.51,$$
for $i=1,\ldots,4$.
Reindexing and perturbing $v$ as necessary, 
we may assume that $2\le |v_1-v|\le2.51$
and $|v_i-v|\ge2.51$, for $i=2,3,4$. 
Moving $v$, we may assume it reaches the minimal distance to two adjacent
corners ($2$ for $v_1$ or $2.51$ for $v_i$, $i>1$).  Keeping $v$
fixed at this minimal distance, perturb the quad cluster along its
remaining degree of freedom until $v$ attains its minimal distance
to three of the corners.  This is a rigid figure.  There are four
possibilities depending on which corner is not included with the three.
Pick coordinates to show that the distance from $v$ to the remaining
vertex violates its inequality.\qed
\enddemo


\head 3. Functions related to the score\endhead
%\heads{3. Functions}

Set $\zeta^{-1}:=\sol(S(2,2,2,2,2,2))=2\arctan(\sqrt{2}/5)$.
The constant $\zeta$ is related to the other fundamental
constants by the relations $\pt= 2/\zeta-\pi/3$ and
$\doct=(\pi-2/\zeta)/\sqrt{8}$.  Rogers's bound
is $\sqrt{2}/\zeta\approx 0.7796$.

We consider the functions
$\sigma_\lambda(R):=\sigma(R)-\lambda \zeta\sol(R)\,\pt$, for
$\lambda=0$, $1$, or $3.2$, where
$R$ is a standard cluster.
The constant $3.2$ was determined experimentally.
We will see that $\sigma_1$ has a simple interpretation.  
 We write $\tau(R) = -\sigma_1(R)$.
If $D^*$ is a decomposition star with standard clusters $\{R\}$, set
$\tau(D^*) = \sum_{R}\tau(R)$.
\smallskip

\proclaim{Lemma 3.1} 
$\tau(R)\ge 0$, for all standard clusters $R$.
\endproclaim

\demo{Proof}  If $R$ is not a quasi-regular tetrahedron, or
if it is but $\rad(R)\ge1.41$, then
$\sigma(R)\le0$ and $\sol(R)> 0$, so that the result is immediate
(see I.9.17).
Assume that $R$ is a 
quasi-regular tetrahedron and $\rad(R)\le 1.41$.  
  The result follows from
Calculation 10.3.6, which asserts that $\Gamma(R)\le \sol(R)\zeta\pt$.
(Equality is attained only for the regular tetrahedron of edge $2$.)
  \qed
\enddemo

\proclaim{Lemma 3.2}  
$$\sigma(D^*) = {4\pi \zeta\,\pt} - \tau(D^*).$$
\endproclaim

\demo{Proof}  
Let $R_1,\ldots,R_k$ be the standard clusters in $D^*$. Then 
$$\sigma(D^*) = \sum\sigma(R_i) + (4\pi-\sum\sol(R_i))\zeta\,\pt = 4\pi \zeta\,\pt - \sum\tau(R_i).$$
\qed
\enddemo


Since $4\pi \zeta< 22.8$, we find as an immediate corollary that
if there are standard clusters satisfying $\tau(R_1)+\cdots+\tau(R_k)\ge14.8\,\pt$,
then the score of the star is less than $8\,\pt$.


The function $\tau(R)$ gives the amount {\it squandered\/} by 
a particular standard cluster $R$.  If nothing is squandered,
then $\tau(R_i)=0$ for every standard cluster, and the upper bound
is $4\pi \zeta\,\pt\approx 22.8\,\pt$.  
This is Rogers's bound
on density.  It is the unattainable bound that would
be obtained by 
packing regular tetrahedra around a common vertex with no distortion and
no gaps.
(More precisely, in the terminology of \cite{H1}, the score 
$s_0=4\pi \zeta\,\pt$
corresponds to the {\it effective density\/}
 $16\pi\doct/(16\pi- 3 s_0)  =\sqrt{2}/\zeta \approx 0.7796$, 
which is Rogers's bound.)  Every positive lower bound
on $\tau(R_i)$ translates into an improvement on Rogers's bound.
 To say that a decomposition star scores at most $8\,\pt$
is to say that at least
$(4\pi \zeta-8)\pt\approx 14.8\,\pt$ are squandered.


\head 4. Some linear constraints\endhead
%\heads{4. Linear constraints}

This section gives some linear inequalities between
$\sigma(R)-\lambda \sol(R)\zeta\pt$ and
$\dih(R)$.

\proclaim {Proposition 4.1}  Let $R$ be a quad cluster.
  Let $\sigma(R)$ denote its score,
$\dih(R)$ one of the four dihedral angles of $R$,
 and let $\sol(R)$ be the solid angle of the
standard region of $R$.  The following inequalities hold among
$\dih(R)$, $\sol(S)$,  and $\sigma(R)$:
\endproclaim

{\parskip=0pt
\hbox{}



$1$: \quad $\sigma(R)< -5.7906 + 4.56766 \dih(R)$,

$2$: \quad $\sigma(R)< -2.0749 + 1.5094 \dih(R)$,

$3$: \quad $\sigma(R)< -0.8341 + 0.5301 \dih(R)$,

$4$: \quad $\sigma(R) < -0.6284 + 0.3878\dih(R)$,


$5$: \quad $\sigma(R) < 0.4124 - 0.1897 \dih(R)$,

$6$: \quad $\sigma(R) < 1.5707- 0.5905\dih(R)$,


$7$: \quad $\sigma(R) < 0.41717 - 0.3\sol(R)$,

$8$: \quad $\sigma_1(R) < -5.81446 + 4.49461 \dih(R)$,

$9$: \quad $\sigma_1(R) < -2.955 + 2.1406 \dih(R)$,

$10$: \quad $\sigma_1(R) < -0.6438 + 0.316 \dih(R)$,

$11$: \quad $\sigma_1(R) < -0.1317$,

$12$: \quad $\sigma_1(R) < 0.3825 - 0.2365 \dih(R)$,

$13$: \quad $\sigma_1(R) < 1.071 - 0.4747 \dih(R)$.


$14$: \quad $\sigma_{3.2}(R) < -5.77942 + 4.25863\dih(R)$,

$15$: \quad $\sigma_{3.2}(R) < -4.893 + 3.5294 \dih(R)$,

$16$: \quad $\sigma_{3.2}(R) < -0.4126$,

$17$: \quad $\sigma_{3.2}(R) < 0.33 - 0.316 \dih(R)$.

$18$: \quad $\sigma(R) < -0.419351 \sol(R) -5.350181+ 4.611391\dih(R)$,


$19$: \quad 
	$\sigma(R) < -0.419351 \sol(R) -1.66174 + 1.582508\dih(R)$,

$20$: \quad 
	$\sigma(R) < -0.419351 \sol(R) +0.0895+ 0.342747\dih(R)$,

$21$: \quad 
	$\sigma(R) < -0.419351 \sol(R) +3.36909 - 0.974137\dih(R)$,

}

\proclaim{Proposition 4.2}
\footnote""{The octahedral case of Inequality 4.2.3 has not
	been rigorously established by interval arithmetic.}
 Let $R$ be a quad cluster.
  Let $\dih_1(R)$ and $\dih_2(R)$
be two adjacent dihedral angles of $R$.
  Set $d(R) = \dih_1(R)+\dih_2(R)$.  The following
inequalities hold between $d(R)$ and $\sigma(R)$.
\endproclaim

{\parskip=0pt

\hbox{}

$1$:  \quad $\sigma(R) < -9.494 + 3.0508\, d(R)$,

$2$:  \quad $\sigma(R) < -1.0472 + 0.27605\, d(R)$,

$3$:  \quad $\sigma(R) < 0.7624 -0.198867\, d(R)$,

$4$:  \quad $\sigma(R) < 3.5926 - 0.844 \, d(R)$,

}

\proclaim {Proposition \refy{4.3}}
\endproclaim

{\parskip=0pt
\hbox{}

$1$: \quad $1.153< \dih(R)$,

$2$: \quad $\dih(R)< 3.247$,

}

\demo{Proof}  Proposition 4.3 follows from 
interval arithmetic calculations
based on the methods of \cite{I}.\qed
\enddemo



Many of these inequalities have now been proved by 
interval arithmetic methods
with a computer.  The verifications of these inequalities are
extremely long and difficult. They strain the
limits of what a computer is able to prove by rigorous methods
with current technology.  (Of course, there is nothing inherently
difficult about these computations.  The difficulty is a result
of the current limitations of computer technology.)
The need to simplify the scoring system
sufficiently to make the inequalities within reach of computer
verifications is one of the primary motivations of the reformulation
of the scoring system proposed in \cite{F}.

An appendix gives
a general description of the cases involved in the verification.
  Further details are available at \cite{H2}.

\head 5. Types of vertices\endhead
%\heads{5. Types of vertices}

The combinatorial structure of a decomposition star is conveniently
described as  a {\it planar map}.
A planar graph is a graph that can be embedded into the 
plane or sphere.  A planar map is a planar graph with
additional combinatorial structure that encodes a particular embedding
of the graph \cite{T}.  All our planar maps will be unoriented:
we do not distinguish between a planar map and its reflection.
Associated with a planar map are faces, (combinatorial) angles
between adjacent edges, and so forth.  Associated with each planar
map $L$ is a planar graph $G(L)$, obtained by forgetting the additional
combinatorial structure.  Each planar map has a dual $L^*$, obtained by
interchanging faces and vertices.
The faces of a planar map are in natural
bijection with the vertices of $L^*$.  We say that a face
is an $n$-gon if the corresponding vertex in the dual $L^*$ 
has degree $n$.  The {\it boundary\/} of a face is
an $n$-circuit in $G(L)$. The edges of the boundary are in natural bijection
with the edges in $L^*$ that are joined to the face's
dual vertex in $L^*$.  

Associated with each decomposition star is a standard decomposition
of the unit sphere, as described in Part I.  We form
a planar map $L$ by associating with each standard region a face of $L$
and with each edge of a standard region an edge of $L$.
This paper is concerned with the special case of the Kepler
conjecture in which every face of $L$ is a triangle or quadrilateral.

We say that a vertex $v$ of $L$ has {\it type\/} $(p,q)$ if there
are exactly $p$ triangular faces and $q$ quadrilateral faces
that meet at $v$. We write $(p_v,q_v)$ for the type of $v$.

We use the following strategy in the proof of step 3 of the
Kepler conjecture.  The linear inequalities that were stated
in Section 4 will be combined to give a bound on the score of
the standard clusters around a given vertex of a given type.
This bound will depend only on the type of the vertex.
The bound comes as the solution to the linear programming
problem of optimizing
the sum of scores, subject to the linear constraints of Section 4
and to the constraint that the dihedral angles around the vertex
sum to $2\pi$.  Similarly, we obtain a lower bound on what is
squandered around each vertex.  

This gives certain obvious
constraints on decomposition stars.  For example, if more than
$14.8\,\pt$ are squandered at a vertex of a given type,
then that type of vertex cannot be part of a decomposition star
scoring more than $8\,\pt$.  These relations between
scores and vertex types 
will allow us to reduce the feasible planar maps to an
explicit finite list.
For each of the planar maps on this list, we calculate
a second, more refined linear programming bound on the
score.
Often, the refined linear programming bound is less than $8\,\pt$.

This section derives the bounds on the scores of the
clusters around a given vertex as a function of the
type of the vertex.  Define constants 
$\tlp(p,q)/\pt$ by Table 5.1.  The entries marked with an asterisk will
not be needed.

\bigskip
% Table 5.1 of constants.


% page 246 of TeXBook
\def\pt{\hbox{\it pt}}

$$
\vbox{\offinterlineskip
\hrule
\halign{&\vrule#&\strut\quad\hfil#\hfil\quad\cr
height 7pt&\omit&&\omit&&\omit&&\omit&&\omit&&\omit&&\omit&\cr
&\hfil $\tlp(p,q)/\pt$\hfil
        &&\hfil0\hfil
        &&\hfil1\hfil
        &&\hfil2\hfil
        &&\hfil3\hfil
        &&\hfil4\hfil
        &&\hfil5\hfil&
\cr
height 7pt&\omit&&\omit&&\omit&&\omit&&\omit&&\omit&&\omit&\cr
\noalign{\hrule}
height7pt&\omit&&\omit&&\omit&&\omit&&\omit&&\omit&&\omit&\cr
&0&&	*&&	*&&	15.18&& 7.135&& 10.6497&& 22.27&\cr
&1&&	*&& *&&  6.95&& 7.135&&17.62  && 32.3&\cr
&2&&	*&& 8.5&&4.756&&12.9814&&*&&*&\cr
&3&&	*&& 3.6426&&8.334&&20.9&&*&&*&\cr
&4&&4.1396&&3.7812&&16.11&&*&&*&&*&\cr
&5&&0.55&&11.22&&*&&*&&*&&*&\cr
&6&&0.6339&&*&&*&&*&&*&&*&\cr
&7&&14.76&&*&&*&&*&&*&&*&\cr
height7pt&\omit&&\omit&&\omit&&\omit&&\omit&&\omit&&\omit&\cr}
\hrule
}\tag 5.1
$$
% based on sp in more.m

\bigskip

\proclaim{Proposition 5.2}  
Let $S_1,\ldots,S_p$ and $R_1,\ldots,R_q$ be
the tetrahedra and quad clusters around a vertex of type $(p,q)$.
Consider the constants 
of Table 5.1.  We have
$$\align
&\sum^p\tau(S_i) + \sum^q\tau(R_i) \ge \tlp(p,q),\\
\endalign
$$
\endproclaim

\demo{Proof}
Set 
$$(d_i^0,t_i^0)=(\dih(S_i),\tau(S_i)),\qquad 
(d_i^1,t_i^1)=(\dih(R_i),\tau(R_i)).$$  Then
$\sum^p\tau(S_i)+\sum^q\tau(R_i)$ is at least the minimum
of $\sum^p t_i^0+\sum^q t_i^1$ subject to
$\sum^p d_i^0+\sum^q d_i^1 = 2\pi$ and to the system
of linear inequalities of Section 11 Group 3 and 
Proposition 4.1 (obtained
by replacing $-\sigma_1$ and dihedral angles by $t_i^j$ and $d_i^j$).
The constant $\tlp(p,q)$ was chosen to be slightly smaller
than the true minimum of this linear programming problem.  
By convexity, we may take the constants $d_i^0$ to be equal and
the constants $d_i^1$ to be equal (to $(2\pi-pd_1^0)/q$), so the
optimization reduces to a single variable $d_1^0$.

The entry $\tlp(5,0)$ is based on Lemma
5.3, $k=1$.  \qed
\enddemo


\proclaim{Lemma 5.3}
Let $v_1,\ldots, v_k$, for some
$k\le 4$ be distinct vertices of a decomposition
star of type $(5,0)$.  Let $S_1,\ldots, S_r$ be quasi-regular
tetrahedra around the edges $(0,v_i)$, for $i\le k$.
Then 
$$\sum_{i=1}^r \tau(S_i)> 0.55k\,\pt,$$
and
$$\sum_{i=1}^r \sigma(S_i) < r\,\pt - 0.48k\,\pt.$$
\endproclaim


\demo{Proof}
We have $\tau(S)\ge 0$, for any quasi-regular
tetrahedron $S$.  We refer to the edges $y_4,y_5,y_6$ of a simplex
$S(y_1,\ldots,y_6)$ as its top edges. Set $\xi=2.1773$.

We claim (Claim 1) that if $S_1,\ldots,S_5$ are quasi-regular tetrahedra around
an edge $(0,v)$ and if $S_1=S(y_1,\ldots,y_6)$, where $y_5\ge\xi$
is the length of a top edge $e$ on $S_1$ shared with $S_2$, then
$\sum_1^5\tau(S_i) > 3(0.55)\,\pt$.  This claim follows from Inequalities
10.5.1 and 10.5.2 if some other top edge in this group 
of quasi-regular tetrahedra has length greater than $\xi$.
Assuming all the top edges other than $e$ have length at most
$\xi$, the estimate follows from $\sum_1^5\dih(S_i)=2\pi$ and
Inequalities 10.5.3, 10.5.4.

Now let $S_1,\ldots,S_8$ be the eight quasi-regular tetrahedra
around two edges $(0,v_1)$, $(0,v_2)$ of type $(5,0)$.  
Let $S_1$ and $S_2$ be the simplices along the face $(0,v_1,v_2)$.
Suppose
that the top edge $(v_1,v_2)$ has length at least $\xi$.
We claim (Claim 2) that $\sum_1^8\tau(S_i)> 4(0.55)\,\pt$.  If there is a top
edge of length at least $\xi$ that does not lie on $S_1$ or
$S_2$,
then this claim reduces to 
Inequality 10.5.1 and Claim 1.
If any of the top edges of $S_1$ or $S_2$ other than $(v_1,v_2)$ has
length at least $\xi$, then the claim follows from
Inequality 10.5.1 and 10.5.2.  We assume all top edges other
than $(v_1,v_2)$ have length at most $\xi$.  The claim now
follows from Inequalities 10.5.3 and 10.5.5, since the dihedral
angles around each vertex sum to $2\pi$.

We prove the bounds for $\tau$.  The proof for $\sigma$ is entirely
similar, but uses the constant $\xi=2.177303$ and the
Inequalities 10.5.8--10.5.14 rather than 10.5.1--10.5.7.
Claims analogous to Claims 1 and 2 hold for the $\sigma$ bound
by Inequalities 10.5.8--10.5.12.

Consider $\tau$ for $k=1$.  If a top edge has length at least $\xi$, this
is Inequality 10.5.1.  If all top edges have length less than $\xi$,
this is Inequality 10.5.3, since dihedral angles sum to $2\pi$.

We say that a top edge lies around a vertex $v$, if it is an
edge of a quasi-regular tetrahedron with vertex $v$.
We do not require $v$ to be the endpoint of the edge.

Take $k=2$, vertices $v_1$ and $v_2$. 
If there is an edge of length at least $\xi$ that
lies around only one of $v_1$ and $v_2$, then Inequality 10.5.1
reduces us to the case $k=1$.  Any other edge of length at
least $\xi$ is covered by Claim 1.  So we may assume that all
top edges have length less than $\xi$.  And then the result
follows easily from Inequalities 10.5.3 and 10.5.6.

Take $k=3$, vertices $v_1,\ldots,v_3$.  If there is an
edge of length at least $\xi$ lying around only one of the $v_i$,
then Inequality 10.5.1 reduces us to the case $k=2$.
If an edge of length at least $\xi$
lies around exactly two of the $v_i$, then it
is an edge of two of the quasi-regular tetrahedra.  These
quasi-regular tetrahedra give $2(0.55)\,\pt$, and the quasi-regular
tetrahedra around the third vertex $v_i$ give $0.55\,\pt$ more.
If a top edge of length at least $\xi$ lies around all three
vertices, then one of the endpoints of the edge
lies in $\{v_1,v_2,v_3\}$, so the result follows from Claim 1.
Finally, if all top edges have length at most $\xi$, we use
Inequalities 10.5.3, 10.5.6, 10.5.7.

Take $k=4$, vertices $v_1,\ldots,v_4$.  Suppose there
is a top edge $e$ of length at least $\xi$.  If $e$ 
lies around only one of the $v_i$, we
reduce to the case $k=3$.  If it lies around two of them, then
the two quasi-regular tetrahedra along this edge give $2(0.55)\,\pt$
and the quasi-regular tetrahedra around the other two vertices
$v_i$ give another $2(0.55)\,\pt$.  If both endpoints of $e$ are
among the vertices $v_i$, the result follows from Claim 2.  This
happens in particular if $e$ lies around four vertices.  If $e$
only lies around three vertices, one of its endpoints is one of
the vertices $v_i$, say $v_1$.  Assume $e$ is not around $v_2$.
If $v_2$ is not adjacent to $v_1$, then Claim 1 gives the
result.  So taking $v_1$ adjacent to $v_2$, we adapt Claim 1,
by using Inequalities 10.5.1--10.5.7, to show that the eight quasi-regular
tetrahedra around $v_1$ and $v_2$ give $4(0.55)\,\pt$.
Finally, if all top edges have length at most $\xi$, we use
Inequalities 10.5.3, 10.5.6, 10.5.7.
\qed
\enddemo

\bigskip


\head 6. Limitations on types\endhead
%\heads{6. Limitations on types}

Recall that a vertex of a planar map has type $(p,q)$ if it
is the vertex of exactly $p$ triangles and $q$ quadrilaterals.
This section restricts the possible types that appear
in a decomposition star.

Let $\tau_4$ denote the constant $0.1317\approx 2.37838774\,\pt$.
Proposition 4.1.11 asserts that every
quad cluster $R$ satisfies $\tau(R)\ge\tau_4$.  

\proclaim{Lemma 6.1}  The following eight types $(p,q)$ are impossible:
(1)  $p\ge 8$,
(2) $p\ge 6$ and $q\ge 1$,
(3) $p \ge 5$ and $q\ge 2$,
(4) $p \ge 4$ and $q\ge 3$,
(5) $p \ge 2$ and $q\ge 4$,
(6) $p \ge 0$ and $q\ge 6$,
(7) $p \le 3$ and $q=0$,
(8) $p \le 1$ and $q=1$.
\endproclaim

\demo
{Proof}  By Proposition \refy{4.1.3} and Calculation \refy{10.1,3},
a lower bound on the dihedral
angle of $p$ simplices and $q$ quadrilaterals is
$0.8638p+1.153 q$.   If the type exists, this constant must
be at most $2\pi$.  One readily verifies in Cases 1--6 
that $0.8638p+1.153q >2\pi$.  By \refy{4.3} and \refy{10.1.2},
an upper bound on the dihedral angle of $p$ triangles and $q$
quadrilaterals is $1.874445 p + 3.247 q$.  In Cases 7 and 8 this
constant is less than $2\pi$.  \qed
\enddemo

\proclaim{Lemma 6.2}  If the type of any vertex of a decomposition star
is one of $(4,2)$, $(3,3)$, $(1,4)$, $(1,5)$, $(0,5)$, $(0,2)$,
$(7,0)$, then the decomposition star scores less than 8\,\pt.
\endproclaim

\demo{Proof}  According to Table 5.1, we have $\tlp(p,q)> 14.8\,\pt$,
for $(p,q) = (4,2)$, $(3,3)$, $(1,4)$, $(1,5)$, $(0,5)$, or $(0,2)$.
By Lemma 3.2, the result follows in these cases.
Now suppose that one of the vertices has type $(7,0)$. 
By the results of Part I, which treats the case in which all standard
regions are triangles, we may assume that the star
has at least one quadrilateral.  We then
have $\tau(D^*)\ge\tlp(7,0) +\tau_4 >14.8\,\pt$.  The result
follows.  \qed
\enddemo

In summary of the preceding two lemmas, we find that we may
restrict our attention to the following types of vertices.

$$\matrix
   (6,0)&      &       &       &       \\
   (5,0)&(5,1) &       &       &       \\
   (4,0)&(4,1) &       &       &       \\
        &(3,1) &(3,2)  &       &       \\
        &(2,1) &(2,2)  &(2,3)  &       \\
        &      &(1,2)  &(1,3)  &       \\
        &      &       &(0,3)  &(0,4)  \\
\endmatrix
$$

\head 7. Properties of planar maps\endhead
%\heads{7. Properties of planar maps}

\proclaim{Proposition 7.1}  
Suppose that 
$\sigma(D^*)\ge 8\,\pt$.
The planar map $L$ of $D^*$ has the following properties
	(without loss of generality):
\endproclaim

1.  The graph $G(L)$ has no loops or multiple joins.

2.  Each face of $L$ is a triangle or quadrilateral.

3.  $L$ has at least eight triangular faces.

4.  $L$ has at most six quadrilateral faces, and at least one.

5.  Each vertex has one of the following types: 
	$(6,0)$, $(5,0)$, $(4,0)$, $(5,1)$, $(4,1)$, $(3,1)$, $(2,1)$,
		$(3,2)$, $(2,2)$, $(1,2)$, $(2,3)$, $(1,3)$, $(0,3)$, and
		$(0,4)$.

6.  If $C$ is a $3$-circuit in $G(L)$, then it bounds a triangular
	face.

7.  If $C$ is a $4$-circuit in $G(L)$, then one of the following is
	true:

	\quad (a) $C$ bounds some quadrilateral region,

	\quad (b) $C$ bounds a pair of adjacent triangles,
	
	\quad (c) $C$ encloses one vertex, and it has type $(4,0)$ or $(2,1)$.

8.  $\tau_4 q + \sum_{v\in V} (\tlp(p_v,q_v)-\tau_4q_v) \le 14.8\,\pt$, 
	for any collection $V$ of vertices
	in $L$ such that no two vertices of $V$ lie in
	a common face.

\demo{Proof}  A loop would give a closed geodesic on the
unit sphere of length less than $\pi$.  A multiple join
would give nonantipodal conjugate points on the sphere.
Property 2 is
the restriction of step 3 of the Kepler conjecture.
Property 3 follows from I.9.1 and Part II.  Property 4 
follows from $7\tau_4>14.8\,\pt$.
If there are no quadrilaterals,
the problem has been solved in Part I.  The restrictions
on types were obtained in Section 6.  Property 6 is
established in Part I.  A 4-circuit encloses at most one
vertex by I.4.2.  If it encloses none, it gives  a quad
cluster or two tetrahedra.  Otherwise, it encloses a
vertex of type $(4,0)$ or type $(2,1)$.  Property
8 is found in Section 5.
\qed\enddemo






\head 8. Combinatorics\endhead
%\heads{8. Combinatorics}

In parts III and IV of the Kepler conjecture, we need to generate
all planar maps satisfying various lists of conditions.
Here we describe a computer algorithm, which has been implemented
in {\it Java}, to do this.  We describe the algorithm in a way
that it can be used for part IV as well.

We assume that the planar maps satisfy the following conditions.

1.  There are no loops or multiple joins.

2.  Each face is a polygon.

3.  The graph has between 3 and $N$ vertices, for some explicit $N$.

4.  The degree at each vertex is at most $7$.

5.  If $C$ is a $3$-circuit in $G(L)$, then $C$ bounds a triangular face.

6.  If $C$ is a $4$-circuit in $G(L)$, then one side of $C$ contains
	at most $2$ vertices.

7.  There is a constant $T\ge 0$ and constants $\tau_n\ge0$ and
$\tlp(p,q)\ge 0$, such that
	$$\sum_I\tau_n + \sum_V\tlp(p_v,q_v) < T$$
	where the first sum runs over {\it finished\/} faces, and
	the second sum runs over any {\it separated\/} set of vertices.
	(The term {\it finished\/}
	 will be described below.  For planar maps
	produced as output of the algorithm, every face will be finished.
	But at intermediate stages of the algorithm some will be unfinished.)
	Also, a separated set of vertices means a a collection a vertices
	with the property that no two lie on a common face, every face
	at each of the vertices is finished, and none of the faces at any
	of the vertices has more than four sides.

There are additional properties that it might be helpful to impose,
but the ones stated are sufficient for the description of the algorithm.
In our context, $T=14.8\,\pt$, $\tau_4$ and $\tlp(p,q)$
are the constants of Sections \refy{5} and \refy{6}.

We produce all graphs satisfying the conditions ($1\ldots 7$), 
by extending
maps already satisfying these properties.  We assume that we have
a stack of maps $\{ L_i : i\in J\}$, satisfying the conditions 
such 
that each face of each of these maps is labeled finished or unfinished.
Each planar map in the stack will have at least one unfinished face.
In one iteration of the algorithm, we pop one of the maps $L$ from the
stack, modify it in various ways to get new maps, output any of the
new maps that are finished (meaning all faces are finished), and push the
remaining ones back on the stack.  When all maps have been popped from
the stack, we are guaranteed to have produced all maps satisfying
properties ($1\ldots7$). Here are the details of the algorithm.

1.    Let $L$ be a planar
map that has been popped from the stack.  Fix any unfinished face $F$ of $L$
and any edge $e$ of $F$. 
Label the vertices of $F$ consecutively $1\ldots\ell$, with $1$ and
$\ell$ the endpoints of $e$.  For each $m=3,\ldots,N$,
let $A_m$ be the set of all $m$-tuples
$(a_i)\in {\Bbb Z}^m$, 
satisfying $(1=a_1\le a_2\le\cdots\le a_m=\ell)$,
with $a_{m-1}\ne a_m$.  

2.  For each $a\in A_m$, we draw a new $m$-gon $F'$
along the edge $e$ of  $F$ as follows.  
We construct the vertices $v(1),\ldots,v(m)$
inductively.  If $i=1$ or $a_i\ne a_{i-1}$, then set $v(i)=$ vertex $a_i$
of $F$.  But if $a_i=a_{i-1}$, we add a new vertex to the planar map,
and let $v(i)$ be it.  The new face is to be drawn along the edge
$e$ over the face $F$.
(For example, if $\ell = 5$, the faces $F'$ corresponding to
$a = (1,1,3,4,4,5)$ and $a=(1,1,1,5)$ are shown in Diagram \refy{8.1}.)
As we run over all $m$ and $a\in A_m$, we run over all possibilities
for the finished face along the edge $e$ inside $F$.

\smallskip
\gram|2.2|\refy{8.1}|diag91.ps|
\smallskip

3.  The face $F'$ is to be marked as finished.  By drawing $F'$, $F$
is broken into a number of smaller polygons.  (In Diagram \refy{8.1},
 $F$ is replaced respectively by three and two polygons in the two 
examples shown.)  Each of these smaller polygons other than
$F'$ is labeled unfinished, except triangles,
which are always labeled
finished.

4.  Various planar maps extending $L$ are obtained by this process.
Those that do not satisfy the conditions ($1\ldots7$) are discarded.
Those that have no unfinished faces are output.  The remaining ones
are pushed back onto the stack.  If the stack is empty, the
program terminates.  Otherwise, we pop a planar map from the
stack and return to step one.  


To begin the algorithm, we need an initial stack of planar maps.
The planar maps in the initial stack will be called {\it seeds}.
To produce a list of seeds, it is enough to give any list that is
guaranteed to produce all possibilities by the algorithm described
in $1\ldots4$.  For example, for part III we want all configurations
with at least one quadrilateral and nothing but triangles and quadrilaterals.
We could let our initial stack consist of a single planar map $L$, where
the graph $G(L)$ is a single $4$-cycle.  Also, $L$ is to have two
faces, a finished quadrilateral and an unfinished complement.  Then
by iterating through the steps $1\ldots4$, we generate all possible
extensions of a quadrilateral to a planar map satisfying $1\ldots7$.

Although this seed would work, 
in order to improve the performance of the algorithm,
in the implementation used for this paper,
we used a more detailed list of seeds, based on the classification
of types $(p,q)$ in Section 6.

This algorithm produced nearly two thousand cases, 
even when the additional
properties listed in Proposition \refy{7.1} were used.   
To be exact, 1727 cases were obtained, but a few of the graphs
may be superfluous, because there was no need to discard every
last graph that we were allowed to.
The important point 
is that an explicit finite list was obtained.  Because of the number
of possibilities involved we have not listed them here.  The
java source code and pictures of the graphs are available at \cite{H2}.  




\head \refy{9}. Linear programming bounds\endhead
%\heads{\refy{9}. Linear programming bounds}


For each of the planar maps produced in Section \refy{8},
we define a linear programming
problem whose solution dominates the score of the decomposition stars
associated with the planar map.  A description of the linear
programs is presented in this section.

\proclaim{Theorem 9.1}  Let $L$ be any planar map obtained in
Section 8.  One of the following holds.  (1)  $L$ is the planar map
of the pentagonal prism, hexagonal close packing, or face-centered
cubic.  (2) Every decomposition star with planar map $L$ scores
less than $8\,\pt$.  (3)  $L$ is one of the $18$ cases presented
in Appendix I.
\endproclaim

The $18$ cases that occur in Appendix I have linear programming
bounds less than $9.59\,\pt$.  This corresponds to a density of
at most $0.7445$.
\smallskip

The variables of the linear
programming problem are the dihedral angles and the
scores of each of the standard regions.  

We subject
these variables to a system of linear inequalities.
First of all, the dihedral angles around each vertex sum to $2\pi$.
The dihedral angles, solid angles, and score are related by
the linear inequalities of Groups 1, 2, 3, and 4 in Section \refy{10}.
These include Propositions 4.1 and 4.2.
In all of these inequalities the solid angle variables
 may be eliminated, since
they are linear functions of dihedral angles.
The score of a decomposition star is
$$\sigma(S_1)+\cdots+\sigma(S_p)+\sigma(R_1)+\cdots+\sigma(R_q).$$
Forgetting the origin of the scores, solid angles, and dihedral
angles as nonlinear functions of the standard clusters and treating
them as formal
variables subject only to the given linear inequalities,
 we obtain a linear programming bound on the score.

Floating-point arithmetic was used freely in obtaining these bounds.
The linear programming package {\it CPLEX\/} was used 
(see {\tt www.cplex.com}).
However, the results, once obtained, could be checked rigorously as follows.
(We did not actually do this because the precision never seemed to be
an issue, but this is how it can easily be done.)
 For each quasi-regular tetrahedron $S_i$
we have a nonnegative variable $x_i = \pt-\sigma(S_i)$.
For each quad cluster $R_k$, we have a nonnegative variable
$x_k = -\sigma(R_k)$.  A bound on the score is
$p\,\pt-\sum_{i\in I} x_i$, where $p$ is the number of triangular
standard regions, and $I$ indexes the faces of the planar map.
Let the dihedral angles be $x_j$, for $j$ in some
indexing set $J$.  Write the linear constraints
as $Ax\le b$.  We wish to maximize $c\cdot x$ subject
to these constraints, where $c_i=-1$, for $i\in I$ and
$c_j=0$ for $j\in J$.  Let $z$ be an approximate solution
to the inequalities $zA\ge c$ and  $z\ge 0$ obtained by numerical
methods.  Replacing the negative entries of $z$ by $0$
we may assume that $z\ge0$ and that $zA_i> c_i-\epsilon$,
for $i\in I\cup J$, and some small error $\epsilon$.
If we obtain the numerical bound $p\,\pt+z\cdot b< 7.9999\,\pt$,
and if $\epsilon<10^{-8}$, then the score is less than $8\,\pt$.
In fact, note that
$$\left({z\over 1+\epsilon}\right) A_i$$
is at least $c_i$
for $i\in I$ (since $c_i=-1$), 
and that it is greater than $c_i - \epsilon/(1+\epsilon)$,
for $i\in J$ (since $c_i=0$).  
Thus, if $N\le 60$ is the number of vertices, and $p\le 2(N-2)\le116$
is the number of triangular faces, 
$$
\align
\sigma(D^*) &\le p\,\pt + c\cdot x \le
	     p\,\pt + \left({z\over 1+\epsilon}\right) A x
		+ {\epsilon\over 1+\epsilon}\sum_{j\in J} x_j \\
	&\le p\,\pt + {z\cdot b\over 1+\epsilon} +
	{\epsilon\over 1+\epsilon} 2\pi N \\
	&\le \left[{p\,\pt+z\cdot b +
	  {\epsilon}(p\,\pt+2\pi N)}\right]/(1+\epsilon)\\
	&\le \left[7.9999\,\pt +
		10^{-8}(116\,\pt+500)\right]/(1+10^{-8}) <8\,\pt.
\endalign
$$
\bigskip


\head \refy{10.} Calculations\endhead
%\heads{\refy{10.} Calculations}

In each of these calculations, when the cluster is a
quasi-regular tetrahedron $S$, we set
$\sigma=\sigma(S)$, $\dih=\dih(S)$, and so forth,
Let $\sigma_\lambda = \sigma-\lambda\zeta\pt\sol$, 
for $\lambda = 1,3.2$.  We make similar abbreviations
for quad clusters.
The inequalities in group 1 follow from
results appearing elsewhere.  
 These inequalities have been 
verified by interval arithmetic in \cite{H2}. 

\define\n#1{\quad #1.\quad}

\subhead Group 1 \endsubhead
 Calculations that have been verified elsewhere.
{
\baselineskip = 0.66\baselineskip
\obeylines
\parskip=0pt
 
\hbox{}
{\it quasi-regular tetrahedra: }
\n1  $\sigma\le\,\pt$,   (I.12.1)
\n2  $\dih <  1.874445$, (I.8.3.2)
\n3  $\dih > 0.8638$,   (I.9.3)
\n4  $\sigma <  -0.37642101\sol+0.287389$, (I.9.8)
\n5  $\sigma <  0.446634\sol-0.190249$, (I.9.9)
\n6  $\sigma <  -0.419351\sol+0.2856354+0.001$, (I.9.10,I.9.11,I.9.12,I.9.18)
\smallskip
{\it quad clusters: }
\n7  $\sigma \le  0$,   (II)
}


\bigskip
\subhead Group 2\endsubhead   
Inequalities for quasi-regular tetrahedra depending
on edge lengths.
{
\baselineskip = 0.66\baselineskip
\obeylines
\parskip=0pt
 
\hbox{}
\n1  $\sol > 0.551285 + 0.199235(y_4+y_5+y_6-6)-0.377076(y_1+y_2+y_3-6)$,
\n2  $\sol < 0.551286 + 0.320937(y_4+y_5+y_6-6)-0.152679(y_1+y_2+y_3-6)$,
\n3  $\dih > 1.23095 -0.359894(y_2+y_3+y_5+y_6-8)+0.003(y_1-2)+0.685(y_4-2)$,
\n4  $\dih < 1.23096-0.153598(y_2+y_3+y_5+y_6-8)+0.498(y_1-2)+0.76446(y_4-2)$,
\n5  $\sigma <  0.0553737-0.10857(y_1+\cdots+y_6-12)$,
\n6  $\sigma+0.419351\sol <  0.28665-0.2(y_1+y_2+y_3-6)$,
\n7  $\sigma_1 <  10^{-6} -0.129119(y_4+y_5+y_6-6)-0.0845696(y_1+y_2+y_3-6)$.

}

\bigskip
\subhead Group 3\endsubhead
  General inequalities for quad clusters and quasi-regular
tetrahedra.  
{
\baselineskip = 0.66\baselineskip
\obeylines
\parskip=0pt
 
\hbox{}
{\it quasi-regular tetrahedra: }
\n1  $\sigma < 0.37898\dih -0.4111$,
\n2  $\sigma < -0.142\dih+ 0.23021$,
\n3  $\sigma < -0.3302\dih +0.5353$,
\n4  $\sigma_1 < 0.3897\dih -0.4666$,
\n5  $\sigma_1 < 0.2993\dih -0.3683$,
\n6  $\sigma_1 \le 0$,
\n7  $\sigma_1 < -0.1689\dih +0.208$,
\n8  $\sigma_1 < -0.2529\dih +0.3442$,
\n9  $\sigma_{3.2} < 0.4233\dih -0.5974$,
\n{10}  $\sigma_{3.2} < 0.1083\dih -0.255$,
\n{11}  $\sigma_{3.2} < -0.0953\dih -0.0045$,
\n{12}  $\sigma_{3.2} < -0.1966\dih +0.1369$,
\n{13}  $\sigma < -0.419351\sol + 0.796456\dih -0.5786316$,
\n{14}  $\sigma < -0.419351\sol + 0.0610397\dih +0.211419$,
\n{15}  $\sigma < -0.419351\sol  - 0.0162028\dih +0.3085626$,
\n{16}  $\sigma < -0.419351\sol  - 0.0499559\dih +0.35641$,
\n{17}  $\sigma < -0.419351\sol  - 0.64713719\dih+ 1.3225$.
\smallskip
{\it quad clusters: } (Propositions 4.1 and 4.2).
}


\bigskip
\subhead Group 4\endsubhead
  Miscellaneous inequalities.
{
\baselineskip = 0.66\baselineskip
\obeylines
\parskip=0pt
 
\hbox{}
{\it quasi-regular tetrahedra: }
\n1  The quasi-regular tetrahedra at a vertex of type $(4,0)$
	score at most $0.33\,\pt$ (I.5.2).
\n2  The quasi-regular tetrahedra at a vertex of type $(5,0)$
	score at most $4.52\,\pt$ (I.5.2).
\n3  The sum of the dihedral angles around a vertex is $2\pi$.
\n4  The amount squandered by the quasi-regular tetrahedra at a 
	vertex of type $(5,0)$ is at least $0.55\,\pt$. (III.5.3).
\n5  The five quasi-regular tetrahedra $S_i$ at a vertex of type $(5,0)$
	satisfy $$\sum\sigma(S_i)  <  \sum (-0.419351\sol(S_i) + 0.2856354).$$
	(I.5.1.1)

\smallskip
In (6), let $\dih_1$ denote the dihedral angle along the edge opposite the
longest diagonal.
{\it flat quarters: }
\n6 $-0.398(y_2+y_3+y_5+y_6) + 0.3257y_1 - \dih_1 \le -4.14938$,
\n7 $\sol < -4.398954 + 0.719788 p$, where $p$ is the sum of the
	four edge lengths between the corners of the quad cluster.
	(By \cite{H1,6.1}, this verification reduces to a one-dimensional
	calculation.)
\n8 Proposition \refy{4.3}, Lemma \refy{5.3}.
\n9 Inequalities of Appendix 1: A.2.1--11, A.3.1--11, A.4.1--2, A.6.1--7,
	A.6.1'--4'.

}

\bigskip
\subhead Group 5\endsubhead
  Inequalities used by Lemma 5.3.
{
\baselineskip = 0.66\baselineskip
\obeylines
\parskip=0pt
 
\hbox{}
{\it quasi-regular tetrahedra: } Let $\xi=2.1773$, $m=0.2384$.
\n1  If $y_4\ge \xi$, then $\tau > 0.55\,\pt$,
\n2  If $y_4,y_5\ge\xi$, then $\tau > 2(0.55)\,\pt$,
\n3  If $y_4\le \xi$, then $\tau > -0.29349 + m\dih$,
\n4  If $y_4,y_6\le\xi$, $y_5\ge\xi$, then $\tau>-0.26303+m\dih$,
\n5  If $y_6\ge\xi$, $y_4,y_5\le\xi$, then $\tau>-0.5565+m(\dih_1+\dih_2)$,
\n6  If $y_4,y_5,y_6\le\xi$, then $\tau>-2(0.29349)+m(\dih_1+\dih_2)$,
\n7  If $y_4,y_5,y_6\le\xi$, then $\tau>-3(0.29349)+m(\dih_1+\dih_2+\dih_3)$.

\smallskip
Now set $\xi=2.177303$, $m =0.207045$.
\n8  If $y_4\ge \xi$, then $\sigma < (1-0.48)\,\pt$,
\n9  If $y_4,y_5\ge\xi$, then $\sigma < (1-2(0.48))\,\pt$,
\n{10}  If $y_4\le \xi$, then $\sigma < 0.31023815 - m\dih$,
\n{11}  If $y_4,y_6\le\xi$, $y_5\ge\xi$, then $\sigma<0.28365-m\dih$,
\n{12}  If $y_6\ge\xi$, $y_4,y_5\le\xi$, then $\sigma<0.53852-m(\dih_1+\dih_2)$,
\n{13}  If $y_4,y_5,y_6\le\xi$, then $\sigma<-pt+2(0.31023815)-m(\dih_1+\dih_2)$,
\n{14}  If $y_4,y_5,y_6\le\xi$, then $\sigma<-2\,\pt+3(0.31023815)-m(\dih_1+\dih_2+\dih_3)$.
}




\bigskip
\vfill\eject
\Refs
%\heads{\hbox{}}
\bigskip

[F].  S.P. Ferguson, T.C. Hales, A Formulation of the Kepler
	Conjecture, preprint.

[I].  T.C. Hales, Sphere Packings I, Discrete and
	Computational Geometry, 17:1-51 (1997).

[II]. T.C. Hales, Sphere Packings II, Discrete and
	Computational Geometry, 18:135-149 (1997).

[V]. S.P. Ferguson, Sphere Packings V, thesis, 
	University of Michigan, 1997.

[H1]. T.C. Hales, the Sphere Packing Problem, J. of Comp. and App. Math. 44
	(1992) 41--76.

[H2]. T.C. Hales, Packings 
	{\tt http://www.math.lsa.umich.edu/\~\relax 
		hales/packings.html}

[T]. W.T. Tutte, Graph theory, Addison-Wesley, 1984.

\endRefs
\newpage



 
\head Appendix 1. Some final cases\endhead

The body of this paper eliminates all but 18 planar graphs. 
In the numbering established by the graph archive \cite{H2},
these graphs are 
$$16, 58, 61, 69, 70, 72, 119, 121, 124, 126, 127, 139, 143,
    147, 157, 179, 238, 351.$$
 This
appendix indicates how to eliminate the final 18 graphs.  These
18 cases have been divided into more than 15,000 subcases.  A linear
programming bound of $8\,\pt$ was obtained in each case.  This
appendix lists all of the inequalities that have been used, and
gives a description of the cases.  We refer the reader to \cite{X} for
details about computer implementation of the linear programs. 

\subhead A.1. Types\endsubhead
Each quad cluster with corners $(v_1,v_2,v_3,v_4)$ is one of four
types \cite{F}.

	1.  Two flat quarters with diagonal $(v_1,v_3)$.  The score of
	each quarter is compression or the analytic Voronoi function.

	2.  
	  Two flat quarters with diagonal $(v_2,v_4)$.  The score of
	each quarter is compression or the analytic Voronoi function.

	3.  Four upright quarters forming an octahedron.  The score of
	each upright quarter is compression or the averaged analytic
	Voronoi function $\octavor(Q)=(\vor(Q)+\vor(\hat Q))/2$,

	4.  One of various mixed quad clusters.  The score is at most
	$\vor_0$, the truncated Voronoi function at radius $t_0=1.255$.

\smallskip
\subhead A.2. Flat quarters\endsubhead
Although it was advantageous to group these cases together to simplify
the combinatorics, it is now better to separate these cases and to
develop linear inequalities for each case.  

For flat quarters, we have the following inequalities that were
established by interval arithmetic.   The edge $y_4$ is taken
to be the diagonal of the flat quarter.

{
\baselineskip = 0.66\baselineskip
\obeylines
\parskip=0pt
 
\hbox{}

\n1  $-\dih_2+0.35 y_2 - 0.15 y_1 - 0.15 y_3 +0.7022 y_5 - 0.17 y_4 > -0.0123$.
\n2  $-\dih_3+0.35 y_3 - 0.15 y_1 - 0.15 y_2 +0.7022 y_6 - 0.17 y_4 > -0.0123$.
\n3  $\dih_2-0.13 y_2 + 0.631 y_1 + 0.31 y_3 -0.58 y_5+0.413 y_4+0.025 y_6 %
	> 2.63363$.
\n4  $\dih_3-0.13 y_3 +0.531 y_1+0.31 y_2 -0.58y_6+0.413 y_4 +0.025 y_5 %
	> 2.63363$.
\n5 $-\dih_1 +0.714y_1-0.221 y_2-0.221 y_3+0.92 y_4-0.221y_5-0.221 y_6 %
	> 0.3482$.
\n6 $\dih_1-0.315 y_1 +0.3972 y_2 +0.3972 y_3 - 0.715 y_4 +0.3972 y_5 %
	+0.3972 y_6 > 2.37095$.
\n7 $-\sol-0.187 y_1 -0.187 y_2 -0.187 y_3 +0.1185 y_4 + 0.479 y_5 %
	+0.479 y_6 > 0.437235$,
\n8 $\sol+0.488 y_1 + 0.488 y_2 + 0.488 y_3  - 0.334 y_5 %
	-0.334 y_6 > 2.244$,
\n9 $-\sigma -0.159 y_1 - 0.081 y_2 - 0.081 y_3 - 0.133 y_5 - 0.133 y_6 %
	> -1.17401$,
\n{10} $\sigma < -0.419351\sol + 0.1448 + 0.0436(y_5+y_6-4) %
	+ 0.079431\dih$,
\n{11} $\sigma < 10^{-6} -0.197 (y_4+y_5+y_6-2\sqrt{2}-4)$, 


}

\subhead A.3. Upright quarters\endsubhead
The following inequalities for upright quarters
have been established by interval arithmetic.
The first edge is taken to be the upright diagonal.

\bigskip
{
\baselineskip = 0.66\baselineskip
\obeylines
\parskip=0pt
 
\hbox{}

\n1 $\dih_1 - 0.636 y_1 + 0.462 y_2 + 0.462 y_3 - 0.82 y_4 + 0.462 y_5 %
	+0.462 y_6 > 1.82419$,
\n2 $-\dih_1 + 0.55 y_1 - 0.214 y_2 - 0.214 y_3 + 1.24 y_4 - 0.214 y_5 %
	-0.214 y_6 > 0.75281$,
\n3 $\dih_2 +0.4 y_1 -0.15 y_2 + 0.09 y_3 +0.631 y_4 -0.57 y_5 +0.23 y_6 %
	>2.5481$,
\n4 $-\dih_2-0.454 y_1 + 0.34 y_2 +0.154 y_3 -0.346 y_4 +0.805 y_5 %
	> -0.3429$,
\n5 $\dih_3 +0.4 y_1 -0.15 y_3 + 0.09 y_2 +0.631 y_4 -0.57 y_6 +0.23 y_5 %
	> 2.5481$,
\n6 $-\dih_3 -0.454 y_1 +0.34 y_3 +0.154 y_2 -0.346 y_4 +0.805 y_6 %
	> -0.3429$,
\n7 $\sol +0.065 y_2 + 0.065 y_3 + 0.061 y_4 -0.115 y_5 -0.115 y_6 %
	> 0.2618$,
\n8 $-\sol-0.293 y_1 -0.03 y_2 -0.03 y_3 + 0.12 y_4 +0.325 y_5 +0.325 y_6 %
	> 0.2514$,
\n9 $-\sigma-0.054 y_2 -0.054 y_3 - 0.083 y_4 - 0.054 y_5 -0.054 y_6 %
	> -0.59834$,
\n{10} $\sigma < -0.419351\sol + 0.089431\dih +0.06904 -0.0846(y_1-2.8)$.
\n{11} If $y_2,y_3\le 2.13$, then $\sigma < 0.07(y_1-2.51) %
	-0.133(y_2+y_3+y_5+y_6-8) - 0.135 (y_4-2)$.

}

\bigskip
\subhead A.4. Truncated quad clusters\endsubhead
Let $\phi(h,t) = (4 - 2\doct h t (h + t))/3$.
Set $t_0=1.255$ and $\phi_0=\phi(t_0,t_0)$.
In the truncated case $\vor_0$, we have
$$\vor_0 = \phi_0\sol + \sum A(y_i/2)\dih_i - 4\doct\sum_R\quo(R),$$
with
	$\phi_0 = \phi(t_0,t_0)$, and
$$A(h) = (1-h/t_0)(\phi(h,t_0)-\phi(t_0,t_0)).$$
Let $R$ be the Rogers simplex $R(y_1/2,\eta(y_1,y_2,y_6),t_0)$.
The function $\quo(R)$ is defined in \cite{F.3.3}.
We have $\quo(R)\ge0$.   Let $\vor_0^A$ denote the truncated Voronoi
score of half the quad cluster, divided into two simplices along
a diagonal, obtained by applying the formula
for $\vor_0$ to the simplex.
The following inequalities hold by interval arithmetic:

{
\baselineskip = 0.66\baselineskip
\obeylines
\parskip=0pt
 
\hbox{}

\n1 $\dih-0.372 y_1 +0.465 y_2 +0.465 y_3 + 0.465 y_5 +0.465 y_6 %
	> 4.885$,
\n2 $-\vor^A_0 - 0.06 y_2 -0.06 y_3 -0.185 y_5 -0.185 y_6 > -0.9978$,
	provided $\dih<2.12$, and $y_1,y_2,y_3\le 2.26$.
\n3 $\quo + 0.00758 y_1 + 0.0115 y_2 + 0.0115 y_6 > 0.06333$.

}

Also, $A\ge0$, $A'\le0$, and $A''\ge0$, for $h\in[1,t_0]$. 
If $\dih\in[\dih_{\min},\dih_{\max}]$, and $h\in[1,h_{\max}]$,
for some constants $\dih_{\min}$, $\dih_{\max}$, and $1<h_{\max}\le t_0$,
then setting $\lambda = (A(h_{\max})-A(1))/(h_{\max}-1)$, we obtain
the additional elementary inequalities for $Ad := A(h)\dih$.


{
\baselineskip = 0.66\baselineskip
\obeylines
\parskip=0pt
 
\hbox{}
\n4 $Ad - A(1)\dih \le \lambda (h-1) \dih_{\min}$,
\n5 $Ad \le (A(1)+\lambda (h-1))\dih_{\max}$.


}

We use linear programming methods to determine bounds $h_{\max}$,
$\dih_{\min}$, $\dih_{\max}$.  If $A x\le b$ is the system of
inequalities used in the linear programs in the main body of the paper,
then we obtain an upper bound on a variable $y$ by solving the
linear program
$\max y$ subject to the constraints $A x\le b$, and the constraint
that the sum of the variables $\sigma$ (that is, the linear variables
corresponding to
the score) is at least $8\,\pt$.  We find in particular by this
approach that the hypothesis $y_1,y_2,y_3\le 2.26$ of Inequality
A.4.2 is always satisfied for the 18 cases.

\bigskip
\subhead A.5. Linear Programs\endsubhead

Number the $18$ planar maps $C_1,\ldots,C_{18}$.  
To be concrete, the planar graphs
are all numbered $P_1,\ldots,P_{1749}$ in \cite{H2}.  The 18 that
give exceptions are $C_i = P_{k_i}$, where
$$k=(16, 58, 61, 69, 70, 72, 119, 121, 124, 126, 127, 139, 143,
    147, 157, 179, 238, 351).$$
Consider $C=C_n$, $n\le18$.  Suppose $C$ has $r$ quadrilateral
faces.  We run $4^r$ linear programs, depending on which type
$1$--$4$ of quad cluster each quadrilateral face represents.
In each case, we add the additional linear inequalities \cite{H2}
as appropriate.  Note that a few of these inequalities are only
conditionally true, so that the inequality can be used only if
it is known that the condition holds.

Most of these linear programs give bounds under $8\,\pt$.  
There
are $15$ cases that give bounds greater than $8\,\pt$.  To describe
these $15$ cases, let $(n,a_1,\ldots,a_r)$ denote the case involving
planar map $C_n$.  Let $a_1,\ldots, a_r$ 
denote the type of quad clusters on quadrilateral faces
$1,\ldots,r$ of the planar map, also numbered as in \cite{H2}.  
The exceptions are
$$
\align
&(6,1,1) \quad (6,3,1) \quad (6,1,3) \\
&(7,1,1,2,1) \\
&(10,1,1,1)\\
&(11,1,1),\quad (11,3,1),\quad (11,4,1),\quad (11,1,3),\quad (11,3,3),
	\quad (11,4,3)\\
&(12,1,2),\\
&(13,1,1),\\
&(16,1),(16,3).
\endalign
$$
For example, $(6,3,1)$ indicates map $C_{6}=P_{72}$, with an upright
octahedron at one quad cluster and two flat quarters at the other,
ordered according to the conventions of \cite{H2}.

\bigskip
\subhead A.6. Quasi-regular tetrahedra\endsubhead

For each quasi-regular tetrahedron $S(y_1,\ldots,y_6)$ in each of these
cases, we run a linear program maximizing the sum of the heights
$y_1+y_2+y_3$, subject to all the linear inequalities produced
up to this point.
In each case, we find that $y_1+y_2+y_3\le 6.13$.  Thus, we may
impose this constraint without loss of generality.   The following
inequalities have been established by interval arithmetic assuming
that $y_1+y_2+y_3\le 6.13$.  The first $7$ inequalities assume
that $y_4+y_5+y_6\le 6.25$.  The next $4$ inequalities assume
that $y_4+y_5+y_6\ge 6.25$.


{
\baselineskip = 0.66\baselineskip
\obeylines
\parskip=0pt
 
\hbox{}
\n1 $\sol>0.551285 + 0.221(y_4+y_5+y_6-6)-0.377076(y_1+y_2+y_3-6)$,
\n2 $\sol<0.55778 + 0.221(y_4+y_5+y_6-6)$,
\n3 $\dih>1.23095 - 0.27(y_2+y_3+y_5+y_6-8)+0.689(y_4-2)$,
\n4 $\dih<1.23116 - 0.212(y_5+y_6-4)+0.498(y_1-2)+0.731(y_4-2)$,
\n5 $\sigma<0.0553737 - 0.14135 (y_1+y_2+y_3+y_4+y_5+y_6-12)$,
\n6 $\sigma+0.419351\sol<0.28665 -0.2(y_1+y_2+y_3-6)-0.048(y_4+y_5+y_6-6)$,
\n7 $-\tau < 10^{-6} - 0.163(y_4+y_5+y_6-6)-0.0845696(y_1+y_2+y_3-6)$,

}

{
\baselineskip = 0.66\baselineskip
\obeylines
\parskip=0pt
 
\hbox{}

\n{1'} $\sol >0.60657 + 0.1781 (y_4+y_5+y_6-6.25)-0.356(y_1+y_2+y_3-6)$,
\n{2'} $\sol< 0.61298 + 0.3405 (y_4+y_5+y_6-6.25)-0.254(y_1+y_2+y_3-6)$,
\n{3'} $\sigma< 0.02004 -0.0781 (y_4+y_5+y_6-6.25)-0.167(y_1+y_2+y_3-6)$,
\n{4'} $\sigma+0.419351\sol< 0.27441+0.0106(y_3+y_4+y_5-6.25)-0.2(y_1+y_2+y_3-6)$.

}

For each of the 15 cases, we use a branch and bound methods
as follows.  We pick 10 quasi-regular
tetrahedra in the configuration.  We divide the domain into $2^{10}$
cases by imposing the constraint $y_4+y_5+y_6\le6.25$ or $y_4+y_5+y_6\ge6.25$
at each quasi-regular tetrahedron.  Depending on which constraint
is picked, we add (to the inequalities already present)
the first group 1--7 or the second group 1'--4'
of inequalities. This gives $2^{10}$ linear programs.  Most of these
give bounds under $8\,\pt$.

The following cases fail to give bounds under $8\,\pt$.
$$
(6,1,1) [26/1024], (11,3,1) [5/1024], (11,4,3) [14/1024], (16,3)
$$
The numbers in square brackets indicate the number of cases out of
$1024$ that fail.

\bigskip
In the case $(6,1,1)$, four additional quasi-regular tetrahedra
were selected and we branched into an additional $26\cdot 2^4$
cases depending on $y_4+y+5+y_6$.  Six of these cases fail to
give bounds under $8\,\pt$.  Two more quasi-regular tetrahedra were
selected, and this time all $6\cdot 2^2$ cases gave bounds under
$8\,\pt$.

In the case $(11,3,1)$, 5 additional quasi-regular tetrahedra
were selected.  All $5\cdot 2^5$ cases give linear programming bounds
less than $8\,\pt$.

In the remaining two cases $(11,4,3)$ and $(6,3)$ there is an upright
octahedron.  A linear programming bound for each of the two
shorter edges $y_2$, $y_3$ of each upright quarter gives the bound
$y_2,y_3\le 2.13$.  Thus, we may impose the Inequality A.3.11.
With this additional inequality, we find that all $14$ cases
stemming from $(11,4,3)$ fall below $8\,\pt$.  With this additional
inequality, all $1024$ cases for $(16,3)$ also fall below $8\,\pt$.

We conclude that all $18$ planar maps that have scores below $8\,\pt$.


\newpage
\head Appendix 2. Interval Verifications\endhead

We make a few remarks in this appendix on the verification of
the inequalities of Proposition 4.1 and 4.2.
The basic method in proving an inequality $f(x)<0$ for $x\in C$,
is to use computer-based interval arithmetic to obtain rigorous
upper bounds on the second derivatives:
$|f_{ij}(x)|\le a_{ij}$, for $x\in C$.  These bounds lead immediately
to upper bounds on $f(x)$ through a Taylor approximation with
explicit bounds on the error.  We divide the domain $C$ as necessary
until the Taylor approximation
on each piece is less that the desired bound.

The computations are exponentially difficult in the number of
variables.  To give a rough idea of the
difficulty, a verification in 6 dimensions might take several
hours on an UltraSparc2/300.  Verifications in 7 dimensions that are
structured to reduce quickly to 6 dimensions might take a day or more.
Verifications in 7 dimensions or higher
are generally beyond my current resources.  The entire
approach to the Kepler conjecture has been structured to bring
it within the reach of a desktop computer.

Some of the inequalities involve as many as 12 variables, such
as the octahedral cases of Proposition 4.2.  These are not directly
accessible by computer.  We describe some reductions we have used.
We start by applying the dimension reduction techniques described
in I.8.7.  We have used these whenever possible.

We will describe Proposition 4.2 because in various respects these
inequalities have been the most difficult to prove, although the
verifications of Propositions 4.1 and 4.3 are quite similar.
If there is a diagonal of length $\le2\sqrt{2}$, we have two flat
quarters $S_1$ and $S_2$.  The score breaks up into
	$\sigma=\sigma(S_1)+\sigma(S_2)$.  The simplices $S_1$ and
$S_2$ share a three-dimensional face.   The inequality we wish
to prove has the form
	$$\sigma \le a(\dih(S_1)+\dih_2(S_1)+\dih_2(S_2))+b.$$
We break the shared face into smaller domains on which we have
$$
\align
\sigma(S_1)&\le a (\dih(S_1)+\dih_2(S_1)) + b_1,\\
\sigma(S_2)&\le a \dih_2(S_1) + b_2,\\
\endalign
$$
for some $b_1,b_2$ satisfying $b_1+b_2\le b$.  These inequalities
are 6-dimensional verifications.

If the quad cluster is an octahedron with upright diagonal, there
are 4 upright quarters $S_1,\ldots,S_4$.
We consider inequalities of the form
$$\sigma(S_i)\le \sum _{j\ne 4} a_j^i y_j(S_i)
	+ a_7 (\dih_1(S_i)-\pi/2) + \sum_{j=2}^3 a \epsilon^i_j \dih_j(S_i)
	+ b^i.$$
If $\sum_{i=1}^4 a_j^i =0$, $j\ne 4$, and $\sum_i b^i\le b$, then
for appropriate $\epsilon^i_j\in\{0,1\}$, these inequalities
imply the full inequality for octahedral quad clusters.

By linear programming techniques, we were able to divide the
domain of all octahedra into about 1200 pieces and find inequalities
of this form on each piece, giving an explicit list of inequalities
that imply Proposition 4.2.  The inequalities involve six variables
and were verified by interval arithmetic.  

To find the optimal 
coefficients $a_j^i$ by linear programming we pose the problem
$$
\align
&\max t \\
&\hbox{st}\\
&\quad\sigma(S_i(x))+t \le a_1^i,\quad i=1,2,3,4, x\in C\\
&\quad\sum_i a_j^i \le 0,\\
&\quad\sum_i b^i \le b,
\endalign
$$
where $x$ runs over all octahedra in a given domain $C$.  We have
infinitely many constraints.  In practice we approximate
$C$ by a large finite set.  If the maximum of $t$ is positive,
then we have a collection of inequalities in small dimensions that
imply the inequality for octahedral quad clusters.  Otherwise,
we subdivide $C$ into smaller domains and try again.  Eventually,
we succeed in factoring the problem.

\smallskip
If the quad cluster is a mixed case, we have the approximations \cite{F}
$\sigma\le \sigma_0, -1.04\,\pt$, so also
$$\sigma \le {3\over 4}\sigma_0 + {1\over 4} (-1.04\,\pt).$$
In the pure Voronoi case with no quarters and no enclosed vertices,
we have the approximation
$$\sigma \le \sigma(\cdot,\sqrt2) \le \sigma_0.$$
If we prove $\sigma_0\le a (\dih_1+\dih_2) + b$, the mixed case is established.
This is how Inequalities 4.2.1 and 4.2.4 are established.  Dimension
reduction reduces this to a 7-dimensional verification.  We draw the
shorter of the two diagonals between corners of the quad cluster.
As we begin to subdivide this 7-dimensional domain, we are able to
separate the quad cluster into two simplices along the diagonal, each
scored by $\sigma_0$.  This reduces the dimension further, to make
it accessible.  Ther two cases, 4.2.2 and 4.2.3, are similar but we
establish the inequalities
$$
\align
{3\over 4}\sigma_0 + {1\over 4} (-1.04\,\pt) &\le a (\dih_1+\dih_2) + b,\\
\sigma(\cdot,\sqrt2)&\le a (\dih_1+\dih_2) + b.
\endalign
$$
This completes our sketch of how the verifications were made.

\bye
