%\magnification=\magstep1
\documentstyle{amsppt}
\topmatter
\parskip=\baselineskip
\baselineskip=1.1\baselineskip  % was 1.1
%\parindent=0pt
\loadmsbm
\UseAMSsymbols
\raggedbottom
\hoffset=0.75truein
\voffset=0.5truein



\font\twrm=cmr8
\def\refz{\relax}
\def\bul{\noindent$\bullet$\quad }
\def\lb#1{\line{$\bullet$ #1 \hfill}}
\def\cir{\noindent$\circ$\quad }
\def\heads#1{\rightheadtext{#1}}
\def\doct{\delta_{oct}}
\def\pt{\hbox{\it pt}}
\def\Vol{\hbox{vol}}
\def\sol{\operatorname{sol}}
\def\quo{\operatorname{quo}}
\def\anc{\operatorname{anc}}
\def\cro{\operatorname{crown}}
\def\vor{\operatorname{vor}}
\def\octavor{\operatorname{octavor}}
\def\dih{\operatorname{dih}}
\def\Adih{\operatorname{Adih}}
\def\arc{\operatorname{arc}}
\def\rad{\operatorname{rad}}
\def\gap{\operatorname{gap}}
\def\xiG{\xi_\Gamma}
\def\xiV{\xi_V}
\def\DLP{\operatorname{D}_{\hbox{\twrm LP}}}
\def\ZLP{\operatorname{Z}_{\hbox{\twrm LP}}}
\def\tLP{\operatorname{\hbox{$\tau$}LP}}  % 3 args small LP.
\def\tlp{\tau_{\hbox{\twrm LP}}}  % 2 args (p,q) tri, quad
\def\slp{\sigma_{\hbox{\twrm LP}}}  % 2 args (p,q) tri, quad
\def\sLP{\operatorname{\hbox{$\sigma$}LP}}
\def\LPmin{\operatorname{LP-min}}
\def\LPmax{\operatorname{LP-max}}
\def\A{{\bold A}}
\def\squander{(4\pi\zeta-8)\,\pt}

\def\Sfour{\Cal{\bold S}_4^+}
\def\Sminus{\Cal{\bold S}_3^-}
\def\Splus{\Cal{\bold S}_3^+}

\def\calc{\footnote"*"{add calculation at end}}

\def\R{{\Bbb R}}
\def\ldot{\cdot}
\def\S{{\Cal S}}
\def\del{\partial}
\def\ha{ \hangindent=20pt \hangafter=1\relax }
\def\ho{ \hangindent=20pt \hangafter=0\relax }
\def\i{I}

\def\diag|#1|#2|{\vbox to #1in {\vskip.3in\centerline{\tt Diagram #2}\vss} }
\def\v{\hskip -3.5pt }
% to place a fig file  align the bottom left corner at (0,-11).
% save as left justified.
% resize so that the box is just under 5 in width
\def\gram|#1|#2|#3|{
        {
        \smallskip
        \hbox to \hsize
        {\hfill
        \vrule \vbox{ \hrule \vskip 6pt \centerline{\it Diagram #2}
         \vskip #1in %
             \special{psfile=#3 hoffset=5 voffset=5 }\hrule }
        \v\vrule\hfill
        }
\smallskip}}


\title The Kepler Conjecture\endtitle
\author Thomas C. Hales\endauthor
\endtopmatter

\document
\footnote""{\hfill preliminary draft -- 4/12/98}
\footnote""{\noindent The results of this paper depend on the
hypotheses listed in the appendices.  These include various 
unfinished computer verifications. Many of the interval verifications
are presented as if they have been carried out, when in fact they
have not.  Sections 5.4, 6.2, and 7 leave various cases unfinished.}


\bigskip
%\heads{\refz 1. Introduction and Review}
\head \refz 1. Introduction and Review\endhead

\subhead \refz 1.1. The steps\endsubhead
The Kepler conjecture asserts that no packing of spheres in
three dimensions has density greater than 
$\pi/\sqrt{18} \approx 0.74048$.  This paper is the final in a series
devoted to a proof of that conjecture.

\bigskip
\head 2.  Parameters\endhead
\subhead \refz 2.1  \endsubhead The paper \cite{III} gives a 
general algorithm for generating all planar maps that are relevant
to the Kepler Conjecture.  We refer to it as the classification
algorithm.  This algorithm has been implemented
in \cite{H1}.  This algorithm depends on various 
parameters.  The purpose of Section 2 is to describe and justify
the parameters that enter into the classification of planar maps.

The parameters are given.  Array indices start at 0.

{\obeylines\tt
\parindent=0pt
\parskip=0pt

\hbox{}
\smallskip
vertexCountMin=1,\quad vertexCountMax=100,
scoreTarget=8,\quad squanderTarget=14.8,
squanderVertex= 
}
$$\pmatrix x&x&x&7.135&10.649&x&x\\
		   x&x&6.95&7.135&x&x&x\\
		   x&8.5&4.756&12.981&x&x&x\\
		   x&3.642&8.334&x&x&x&x\\
		   4.139&3.781&x&x&x&x&x\\
		   5.5&11.22&x&x&x&x&x\\
		   6.339&x&x&x&x&x&x
\endpmatrix$$
{\obeylines\tt
\parindent=0pt
\parskip=0pt

\hbox{}
x=14.8,
scoreFace=$\{1,1,1,1,0,-1.03,-2.06,-3.03,-3.03\}$,
squanderFace=$\{0,0,0,2.378,4.896,7.414,9.932,10.916\}$,
faceCountMax=6,\quad faceCountMaxAtExceptionalVertex=5,
vertexAdjustment=$\{x,x,x,1.4,1.5\}$,
excludePentQRtet= true,

}


\subhead {\tt vertexCountMin} \endsubhead
A lower bound on the number of vertices of height at most 2.51 in
the decomposition star is $1$.  If there are no vertices of height
at most 2.51, there is a single standard region, and the score of
the truncated cluster is $4\pi\phi_0 <0$.

\subhead {\tt vertexCountMax} \endsubhead
There are at most 100 vertices of height 2.51 in the decomposition
star.  The projections of these vertices to the unit sphere are
separated by arc-lengths at least 
$\alpha = 2\arcsin(1/2.51)$.  Each Delaunay triangle on the unit sphere
with vertices at these points has area $\ge3\beta-\pi$,
	with $\cos\beta=1/(1+\sec\alpha)$.
There are at most $4\pi/(3\beta-\pi) < 40$ triangles or at most
$2+40/2$ vertices.  See \cite{H2} for details.

\subhead {\tt scoreTarget} \endsubhead
Any decomposition star scoring less than $8\,\pt$ may be discarded.

\subhead {\tt squanderTarget} \endsubhead
Any decomposition star squandering more than $\squander$ may be
discarded.

\subhead {\tt squanderVertex}\endsubhead
This array gives lower bounds on what is squandered by the
standard regions around a vertex that has $p$ triangles, $q$
quadrilaterals, and no exceptional regions.  These constants
were computed in \cite{III.5.1}.  The entries marked $x$ do not
occur.

\subhead {\tt scoreFace,squanderFace} \endsubhead
A standard region is a polygon or one of the exceptions in IV.4.4.
Each of these exceptions has a polygonal hull, which is a polygon
obtained by removing the internal edges and vertices from the exceptional
region.
It will be shown in Section 3.3 that the polygonal hull can be treated
as a polygonal face in the classification algorithm.  This allows
us to assume, for the purposes of classification, that all standard
regions are polygons.  By the results of IV, each polygon is at most
an octagon.  The constants in the arrays {\tt scoreFace} and 
{\tt squanderFace} are the bounds on what is scored and squandered
by a face as a function of the number of sides.  These constants
are derived in Theorem IV.4.4.

\subhead 2.2. Face Counts\endsubhead

{\tt faceCountMax} is an upper bound on the number of faces at a vertex
of the planar map.  The bound $6$ on the number of faces at a vertex
is established by the lemma that follows.  {\tt faceCountMaxAtExceptionalVertex}
is an upper bound on the number of faces at a vertex at which there
is at least one exceptional face.  The lemma allows one case
with 6 faces, but we set the parameter at 5.  This is justified
in Section 3.2, where the case with 6 faces is folded into the
classification algorithm in a different way.  The proof of the lemma
will use the inequalities III.10 (Group 3. 4--8) and
III.4.1 several times.
Give $p$ quasi-regular tetrahedra and $q$ quad clusters at a vertex,
subject to the constraint that the dihedral angles at the vertex are
at most $d$, let $\tLP(p,q,d)$ denote a linear programming lower bound
on what these $p+q$ standard clusters squander.  Similarly,
let $\sLP(p,q,d)$ be a linear programming upper bound on score.


\proclaim{Lemma}  Suppose that there are six or more standard
clusters around a vertex.  Then either (1) all the clusters are
quasi-regular tetrahedron and the vertex has type $(6,0)$, or
(2) there are five quasi-regular tetrahedra, no quad clusters,
and one exceptional cluster.
\endproclaim

\demo{Proof}
 If there are no exceptional standard regions around the
vertex then the result follows from the inequalities of III.5.2
and IV.4.5:
$$\tau(D^*)\ge t_5 + \min
	\Sb p+q\ge6,\ q\ge1,\\
		|| p+q\ge7
	\endSb
	\tau(p,q)\ge 4.896\,\pt+11.22\,\pt \ge \squander.$$
Without loss of generality, we assume there is an exceptional
region at the vertex.
If there are seven or more clusters,
 then the dihedral angles
around the vertex cannot be $2\pi$:
$$1.153 + 6 (0.8638)> 2\pi$$
(See III.4.3. III.10.G1.3.)
The lower bound $1.153$
on dihedral angles was stated
for quad clusters, but the proof was written to apply
to exceptional clusters as well.  We now assume six standard
clusters at the vertex. There are several cases according to
the number $k$ of triangular regions at the vertex.


{\bf($k\le2$)}
If there are more than three nontriangular regions at the vertex, then
the dihedral angle around the vertex
is at least $4(1.153)+2(0.8638)>2\pi$, which is impossible.

{\bf($k=3$)}
If there are three nontriangular regions at the vertex, then 
we squander at least $2t_4+t_5+\tLP(3,0,2\pi-3(1.153))>\squander$.

{\bf($k=4$)}
If there are two exceptional regions at the vertex, 
we squander at least $2t_5+\tLP(4,0,2\pi-2(1.153))>\squander$.

If there are two nontriangular regions at the vertex, 
we have $t_5+\tLP(3,1,2\pi-1.153)>\squander$.

We are left with the case of five triangular regions and one exceptional
\qed
\enddemo

\bigskip

\subhead 2.3. {\tt vertexAdjustment}\endsubhead

Let $v_1,\ldots,v_k$ be vertices of exceptional regions such that
no two are adjacent an no two are opposite vertices of a quadrilateral.
Suppose that there are 5 faces at each vertex $v_i$ and
at $v_i$ there are $a_i$ triangles, with $0\le a_i\le4$.
Let $F_1,\ldots,F_r$ be the faces around the vertices
$v_1,\ldots,v_k$.  Assume that $F_i$ is an $f_i$-gon.
The vertex adjustments are constants, for $0\le i\le 4$, such
that the faces $F_j$ around the $v_i$ squander at least
$$\sum_{j=1}^4 \hbox{\tt squanderFace}[f_j]+\sum_{i=1}^k
	\hbox{\tt vertexAdjustment}[a_i].$$
For example, {\tt vertexAdjustment}$[i]=\squander$, for $i=0,1,2$,
so that any decomposition star with one of these vertices squanders
at least $\squander$.  The constants for $i=0,1$ are derived
in Section 2.4, that for $i=2$ in Section 2.6,
that for $i=3$ in Section 2.7, and that for
$i=4$ in Section 2.8.


\subhead 2.4. {\tt vertexAdjustment}$[0]$, \ {\tt vertexAdjustment}$[1]$  
\endsubhead

\proclaim{Lemma}  Consider a vertex that has five standard clusters
around it.  Assume that at most one of them is a quasi-regular tetrahedron.
Then the score of the decomposition star is less than $8\,\pt$.
\endproclaim

\demo{Proof}
If there are $a$ quasi-regular tetrahedra, $b$ quad clusters, and
$c$ exceptional clusters, then the clusters squander at least
$$c\, t_5 + \tLP(a,b,2\pi-c(1.153)).$$
We run the linear programs for each $(a,b,c)$, and find that the
bound is always greater than $\squander$.
$$\pmatrix
	(a,b,c)&\hbox{ lower bound }\\
	(0,5,0)&22.27\,\pt \hbox{ (III.5.2)}\\
	(0,b,c\ge1)& t_5+4 t_4 \\
	(1,b,c\ge3)& 3 t_5 + t_4\\
	(1,4,0)	&17.62\,\pt \hbox{ (III.5.2)}\\
	(1,3,1) &t_5 + 12.58\,\pt \ (\tLP)\\
	(1,2,2) &2t_5 + 7.53\,\pt \ (\tLP)
\endpmatrix
$$
\qed
\enddemo

\subhead 2.5. Some flat quarters\endsubhead

Set $\xiV=0.003521$, $\xiG=0.01561$, $\xiG'=0.00935$.
They are the penalties that result from erasing an upright
quarter of Voronoi type, an upright quarter of compression type,
and an upright quarter of compression type with diagonal $\ge2.57$,
respectively.  (See IV.$A_{10}$, IV.$A_{11}$.)

In the next lemma we score a flat quarter by any of the functions
	 $$\sigma= \cases \Gamma,& \eta_{234},\eta_{456}\le\sqrt2,\\
			 \vor, & \eta_{456}\ge\sqrt2,\hbox{ or }\eta_{234}\ge\sqrt2,\\
			\vor_0, & y_4\ge 2.6,\\
			\vor_0+0.035, & \eta_{456}\ge\sqrt2.
	\endcases
	$$
We also measure what is squandered by 
a flat quarter by $\sol\zeta\pt - \sigma$.


\proclaim{Lemma} (1)
Let $v$ be a corner of an exceptional cluster at which
the dihedral angle is at most $1.32$.
Then the vertex $v$ it a vertex (not on the diagonal) of a flat
quarter in the exceptional region.  
Moreover, 
the flat quarter squanders at least
$3.07\,\pt$.
\endproclaim


\demo{Proof}  
Let $S=S(y_1,\ldots,y_6)$ be the simplex inside the exceptional centered
at $v$, with $y_1=|v|$.   
The inequality $\dih\le 1.32$ gives the interval calculation%
\footnote"*"{The results of this paper rely on a large number of
interval calculations.  They are listed in an appendix, arranged
according to section number.}
$y_4\le 2\sqrt{2}$, 
so $S$ is a quarter.
The result now follows by interval arithmetic.
\qed
\enddemo

\proclaim{Lemma}  Let $R$ be an exceptional cluster with a dihedral
angle $\le1.32$ at a vertex $v$.  Let $n=n(R)$ be as in IV.4.4.
Then $R$ squanders $>t_n+1.47\,\pt$.
\endproclaim

\demo{Proof}
In most cases we establish the stronger bound $t_n+1.5\,\pt$.  
In the proof of IV.4.4, we erase all upright diagonals, except
those completely surrounded by anchored simplices.  The contribution
to $t_n$ from the flat quarter at $v$ in that proof is
$D(3,1)$ (IV.4.5, IV.5.5.1).
Replace $D(3,1)$ with $3.07\,\pt$ and we obtain the bound.
Now suppose the upright diagonal
is completely surrounded by anchored simplices.  
  Analyzing the constants of IV.5.11,
we see that $\DLP(n,k)-D(n,k)>1.5\,\pt$.
except when $(n,k)=(4,1)$.

Here we have $4$ anchored simplices around an upright diagonal.
Three of them are quarters.  We erase and take a penalty.
Two possibilities arise.  If the upright diagonal is
enclosed over the flat quarter, its height is $\ge2.6$ by
geometric considerations and the top face of the flat
quarter has circumradius at least $\sqrt2$.  The penalty
is $\pi_0 = 2\xiG' + \xiV< 0.035$, so the
bound holds for the various choices of scoring functions.

If, on the other hand, the upright diagonal is not
enclosed over the flat diagonal, the penalty is
$\pi_0 = \xiG + 2\xiV$.  In this case, we
obtain the weaker bound $1.47\,\pt+t_n$:
	$$3.07\,\pt > 1.189\,\pt + 1.47\,\pt +\pi_0.$$
\qed
\enddemo



\subhead 2.6. {\tt vertexAdjustment}$[2]$ \endsubhead

\proclaim{Lemma} Consider a vertex $v$ with 5 faces around it.
If only two are triangles, then the others are quadrilaterals.
\endproclaim

\demo{Proof}
The constants $0.55\,\pt$, $4.52\,\pt$ used throughout the proof
come from III.5.3; $t_n$ comes from IV.4.4.

($e=3$):
First, assume that there are three exceptional regions around it.
They must be pentagons ($2t_5+t_6>\squander$).  The aggregate
of the five faces is an $11$-gon (or less).  If there is a vertex
not on this aggregate, use $3t_5+0.55\,\pt>\squander$.
So there are at most $9$ quasi-regular tetrahedra.  The
score is at most $9\,\pt + 3 s_5 < 8\,\pt$.

The argument if there is a quad, pentagon, and hexagon is the
same $(t_4+t_6=2t_5,s_4+s_6=2s_5)$.

($e=2$):
Assume next that there are two pentagons and a quadrilateral
around the vertex.  
The aggregate of the two pentagons, quadrilateral, and two triangles
is a $10$-gon (at most).  
There must be an an vertex not on the aggregate of 5 faces, for
otherwise the score is at most $10\,\pt+2s_5<8\,\pt$.

The dihedral angle of one of 
the pentagons is at most $1.32$.  For otherwise, 
$\tLP(2,1,2\pi-2(1.32))+2t_5+0.55\,\pt>\squander$.

Thus, one of the two pentagons has a flat quarter.  Let us
show that this case of the lemma follows once we show that 
any pentagon with a dihedral angle less than $1.32$ 
squanders at least $5.66\,\pt$.   If both pentagons have a dihedral
angle $<1.32$ this is clear 
$2(5.66)\,\pt+\tLP(2,1,2\pi-2(1.153))+0.55\,\pt>\squander$.
If there is one pentagon with angle $>1.32$, 
we then have $5.66\,\pt+\tLP(2,1,2\pi-1.153-1.32)+t_5+0.55\,\pt>\squander$.

To obtain the bound $5.66\,\pt$, we argue as follows.  If
there are 5 anchored simplices surrounding a vertex, we have
the bound by (IV.5.11).  If the configuration $\Sfour$
or $\Sminus$ occurs, we squander at least $7.23\,\pt$.
So if there are any upright diagonals in the pentagon
 that carry a penalty,
we may assume they have $4$ anchors.  If there are no
penalties, Lemma 2.5 gives
$3.07\,pt+D(4,1)>5.66\,\pt$.
We do not need to deal with penalties from $\Splus$
in the score of the flat quarter at $v$ because of the
transfers defined in IV.5.5. 
The only remaining possibility is four simplices surrounding an
upright vertex.  Unless there are three upright quarters, the
bound follows from IV.5.11.  If there are 3 upright quarters,
erasing gives penalty $\pi_0=3\xiG$, and
$3.07\,\pt +D(4,1)-\pi_0 >5.66\,\pt$.
This proves the lemma for two pentagons and a flat quarter.

($e=1$): Assume finally that there is one exceptional
region at the vertex.
If the exceptional region is a hexagon
(or more), we are done $t_6+\tLP(2,2,2\pi-1.153)>\squander$.
Assume it is a pentagon.  The aggregate of the five regions at the
vertex is a $9$-gon (at most).  If there are no more than $11$
quasi-regular tetrahedra outside the aggregate, the score is
at most $(1+2(4.52))\,\pt+s_5+\sLP(2,2,2\pi-1.153)<8\,\pt$.  So we may assume
that there are at least three vertices not on the aggregate.

If the dihedral angle of the exceptional is greater than $1.32$,
we have 
$$\tLP(2,2,2\pi-1.32) +3(0.55)\,\pt +t_5 > \squander;$$ 
and if it is less than $1.32$, we have 
$$\tLP(2,2,2\pi-1.153)+3(0.55)\pt+1.47\,\pt+t_5> \squander$$
\qed
\enddemo

\subhead 2.7. {\tt vertexAdjustment}$[3]$ \endsubhead

When there are three triangles at a vertex with an exceptional,
we establish that the vertex adjustment is $1.4\,\pt$.
This follows from Lemma 2.6 if the dihedral angle at the
vertex of an exceptional region
is $\le1.32$.  Assume the angles are $\ge1.32$.
If there are three triangles, a quad, and an
exceptional at the vertex,
then we have $\tLP(3,1,2\pi-1.32)>1.4\,\pt+t_4$.

The final case is three triangles and two exceptionals
at the vertex.  There can be no heptagons and at most one hexagon
$2t_6 = t_5+t_7>\squander$.  
If neither exceptional has a flat quarter at $v$, the result follows
from an interval calculation.
In these calculations, the flat quarters are scored by $\vor_0$
or $\mu$ as appropriate.%
\footnote"*"{Why not $\vor_0-0.0114$?}

The vertex adjustment now follows from an interval calculation.

\subhead 2.8. {\tt vertexAdjustment}$[4]$ \endsubhead

If there are four triangles and one exceptional at a vertex,
we establish the vertex adjustment of $1.5\,\pt$.
There are two cases.  If there is a flat quarter at the vertex,
we use an interval calculation.  
If there is none, we again use an interval calculation.
  This completes
the justification of the vertex adjustments used in the classification
algorithm.

\head 3. The Classification\endhead

\subhead 3.1 {\tt excludePentQRtet=true} \endsubhead
This parameter tells the classification algorithm that a particular
unusual pentagonal subregion has been treated by hand,
and need not been considered as a separate case in the classification.
  The purpose of this section is to treat that subregion by hand.
The unusual configuration is a vertex that has only two subregions,
a triangle and a pentagon.  The aggregate of the two is a quadrilateral.

Let $v_1,v_2,\ldots,v_5$
be the five corners of the pentagonal cluster $R$, where $v_1$ is a
vertex at only two standard clusters, $R$ and a quasi-regular tetrahedron
$S = (0,v_1,v_2,v_5)$.   Since the four edges $(v_2,v_3)$, $(v_3,v_4)$,
$(v_4,v_5)$, and $(v_5,v_2)$ have length less than $2.51$, the
aggregate of 
$R$ and $S$ will resemble a quad cluster in many respects.

\proclaim{Lemma}  One of the edges $(v_1,v_3)$, $(v_1,v_4)$ has
length less than $2\sqrt{2}$.  Both of the edges have length less
than $3.02$. Also, $|v_1|\ge2.3$.
\endproclaim

\demo
{Proof}
This is a standard exercise in geometric considerations (III.2).
We deform the figure using pivots to a configuration $v_2,\ldots,v_5$
at height $2$, and $|v_i-v_j|=2.51$, $(i,j)=(2,3),(3,4),(4,5),(5,2)$.
We scale $v_1$ until $|v_1|=2.51$.
We can also take the distance from $v_1$ to $v_5$ and $v_2$ to be
$2$.  If we have $|v_1-v_3|\ge 2\sqrt{2}$, then we stretch
the edge $|v_1-v_4|$ until $|v_1-v_3|=2\sqrt{2}$.  The resulting
configuration is rigid.  Pick coordinates to find that $|v_1-v_4|<2\sqrt{2}$.
If we have $|v_1-v_3|\ge 2.51$, follow a similar procedure to
reduce to the rigid configuration $|v_1-v_3|=2.51$, to find that
$|v_1-v_4|<3.02$.
The estimate $|v_1|\ge2.3$ is similar.
\qed
\enddemo

There are restrictive bounds on the dihedral angles of the
simplices $(0,v_1,v_i,v_j)$ along the edge $(0,v_1)$.  
The quasi-regular tetrahedron has a
dihedral angle of at most $1.875$ (III.10.1.2).  The dihedral angles
of the simplices $(0,v_1,v_2,v_3)$, $(0,v_1,v_5,v_4)$
adjacent to it are at most $1.63$ (IV.$A_8$).
The dihedral angle of the remaining simplex $(0,v_1,v_3,v_4)$ is at
most $1.51$ (IV.$A_8$).   This leads to lower bounds as well.
The quasi-regular tetrahedron has a dihedral angle that is at least
$2\pi - 2(1.63)-1.51 > 1.51$.  The dihedral angles adjacent to the
quasi-regular tetrahedron is at least $2\pi- 1.63-1.51-1.875> 1.26$.
The remaining dihedral angle is at least $2\pi-1.875-2(1.63) > 1.14$.

\proclaim{Lemma} A decomposition star with this configuration
squanders $>\squander$.
\endproclaim

\demo{Proof}  We will show that the pentagonal exceptional and
quasi-regular tetrahedron squander at least $11.16\,\pt$.
Let $P$ denote the aggregate cluster formed by these two standard
clusters.

First we show how the lemma follows from this bound.  
There are no other exceptionals
($11.16\,\pt+t_5>\squander$), and every vertex not on $P$
has type $(5,0)$, by III.5.2.  In particular, there are no
quad clusters.  There are at most 4 quasi-regular tetrahedra
at every corner of $P$, because the 
$\tLP(5,0,2\pi-1.32)>6.02\,\pt$.
(The $1.32$ comes from that fact 
that $P$ has no flat quarters because
$v$ is too short to be enclosed over one \cite{F.1.3}, Lemma 2.5.)

The only 
triangulation with these properties is obtained by removing an
edge from the icosahedron.  This implies that 
there are two opposite corners of
$P$ each having four quasi-regular tetrahedra.  
Since the diagonals of $P$ have
lengths greater than $2\sqrt{2}$, the results of Section 2.8
show that these eight quasi-regular tetrahedra squander at least
$2(1.5)\,\pt$.  There are two adjacent vertices of type $(5,0)$
whose tetrahedra are distinct from these eight quasi-regular tetrahedra.
They give an additional $2(0.55)\,\pt$.  Now 
$(11.16+2(1.5)+2(0.55))\,\pt>\squander$.

We prove the bound $11.16\,\pt$.
Let $S_{ij}$ be the simplex $(0,v_1,v_i,v_j)$, for 
$(i,j)=(2,3),(3,4), (4,5),(2,5)$.  We have $\sum_{(4)}\dih(S_{ij}) = 2\pi$.
Suppose some edge $(v_1,v_3)$ or $(v_1,v_4)$ has length $\ge2\sqrt2$.
Say $(v_1,v_3)$.

We have 
$$
\align
\tau(S_{25}) &- 0.2529\dih(S_{25}) > -0.3442,\\
\tau_0(S_{23}) &- 0.2529\dih(S_{23}) > -0.1747,\\
\tau_0(S_{45}) &- 0.2529\dih(S_{45}) > -0.2137,\\
\tau_0(S_{34}) &- 0.2529\dih(S_{34}) > -0.1371,\\
\endalign
$$
Summing, we find 
$\sum_{(4)}\tau(S_{ij}) >2\pi(0.2529)-0.3442-0.1747-0.2137-0.1371>12.99\,\pt$.
The penalty is at most $0.06688$, and $12.99\,\pt -0.06688>11.16\,\pt$.

Now suppose $(v_1,v_3)$ and $(v_1,v_4)$ have length $\le2\sqrt2$.
So 
$$\sum_{(4)}\tau(S_{ij})>2\pi(0.2529)-0.3442-2(0.2137)-0.1371 >12.28\,\pt.$$
If there is an upright diagonal that is not enclosed over either
flat quarter, the penalty is at most $3\xiG+2\xiV+0.0268$.
Otherwise, the penalty is at most $4\xiG'+\xiV+0.0268$.
In both cases, we the penalty is at most $0.973\,\pt$.  We have
$(12.28-0.973)\,\pt>11.16\,\pt$.
\qed
\enddemo





\subhead 3.2.  Six regions\endsubhead

We return to the case of 5 quasi-regular tetrahedra and one
exceptional cluster around a vertex that was left unresolved
in Section 2.2.
When there is an exceptional region at a vertex
of degree six, we claim that the exceptional region must be 
a pentagon.  
If the exceptional region is a heptagon or more, we squander
$t_7+\tLP(5,0,2\pi-1.153) > \squander$.  

If the exceptional region is
a hexagon, we squander $t_6 + \tLP(5,0,2\pi-1.153) > t_9$.
Also, $s_6+\sLP(5,0,2\pi-1.153) < s_9$.
The aggregate of the six clusters is a $9$-gon.
The argument of IV.4.6 extends to this context to give the bound
of $8\,\pt$.

Thus, we can assume that the exceptional region is a pentagon.
The six standard clusters score at most $s_5+\sLP(5,0,2\pi-1.153)< -3.03\,\pt$,
and squander at least $t_5+\tLP(5,0,2\pi-1.153)> 10.916\,\pt$.
The constants $-3.03\,\pt$ and $10.916\,\pt$ are the bounds
for an octagon in {\tt scoreFace} and {\tt squanderFace}.
We note that there can be at most one vertex of degree six with
an exceptional region.  Indeed, if there are two, then they must
both be vertices of the same pentagon $(10.916+4.896)\,\pt>\squander$.
A vertex on the octagon gives
$$\align
	10.916\,\pt &+\tLP(4,0,2\pi-1.153-0.8638) > \squander\\
	10.916\,\pt &+\tLP(5,0,2\pi-1.153) > \squander.
\endalign
$$

\subhead 3.3.  Nonpolygonal standard regions\endsubhead

In the next section we will turn to the classification of
planar maps.  Certain combinatorial complications arise if the
standard regions are not polygons.  Because of these complications
it is best to classify the planar maps arising from the
polygonal hulls of standard regions.  That is, we suppress the internal
structure of the nonpolygonal standard regions.

We know from IV.4.4
what the possibilities for the standard regions are.
If we have, for example, the
eight sided figure whose hull is a pentagon, it will be treated as
a pentagon in the combinatorial classification.  Moreover, the
constants $t_5$ and $s_5$ from IV.4.4 may be used in the
determination of the possibilities, because we have $t_8>t_5$,
$s_8+1\,\pt<s_5$.  This allows us to run the combinatorial classification
assuming that  every face is a polygon. After the classification is
complete, we will go back and interpret various faces as one of
the cases IV.4.4. (See Section 3.6, 3.7, and 3.8.)

\subhead 3.4. Classification\endsubhead

We have now explained and justified all the parameters in Section 2.1.
There are a few additional properties of the planar map that we make
use of in the classification.  The graph has no loops or multiple
joins.  Two degree four vertices cannot be adjacent.  A cycle
of $4$-edges bounds a face or there is an edge between two
opposite corners.  A cycle of $3$ edges bounds a triangle.
(See I.4, I.5.)

The classification algorithm constructs all the planar maps
of decomposition stars that have the potential of scoring $<8\,\pt$,
as implied by the parameters of Section 2.1.  As we construct the
planar maps face by face, the lower bound on whatever is
squandered increases monotonically.  It is this property that
insures the termination of the algorithm.  As soon as a case reaches
$\squander$ squandered, it may be discarded.  Details of the
algorithm appear in III.

We have made no comprehensive effort to make the list as short
as possible, because extraneous cases or redundancies are eliminated
quickly in Section 3.5 by linear programming.  The archive \cite{H1}
contains a list of all the possibilities obtained in the classification.
In summary, the archive contains the following planar maps

{
\parindent=0pt
\parskip=0pt
\hbox{}

$$
\matrix
\format\l&\quad\r&\quad\l\\
\text{(quad)}&1749&\text{planar maps}\\
\text{(pent)}&2459&\text{planar maps}\\
\text{(hex)}&429&\text{planar maps}\\
\text{(hept)}&413&\text{planar maps}\\
\text{(oct)}&44&\text{planar maps}\\
&5094&\text{total}
\endmatrix
$$

}

These planar maps are listed according to the standard region whose
hull has the greatest number of sides.  So for example, the planar
maps in the hexagon list have triangles, quadrilaterals, pentagons,
and at least one hexagon. The planar maps will be referred
to as quad\#n, $n=1,\ldots,1749$, pent\#n, $n=1,\ldots,2459$, etc.
A constant is added to numbers of the planar map in the archive.
Map (pent\#n) corresponds to archive $2000+n$,
(hex\#n) corresponds to $6000+n$,
(hept\#n) corresponds to $8000+n$,
and (oct\#n) corresponds to $10000+n$.

If we eliminate all these possibilities the Kepler Conjecture
will be established.  The quadrilaterals have been treated in
full by III, IV.  The decomposition stars of the
fcc and hcp packings score $8\,\pt$.  They are local
maxima in the strong sense that any deformation of a
quasi-regular tetrahedron or quad cluster from  the 
regular ones appearing in the fcc and hcp brings a drop
in score below $8\,\pt$ \cite{II}.
There is one case that is far more difficult than the
others.  It is the pentahedral prism, treated by S. Ferguson
in \cite{V}.

The next few sections will eliminate the cases that are
most easily disposed of.
Section 3.5 eliminates all but 180 cases by linear programming.
Section 3.6 treats the nonpolygonal region $n(R)=8$ with
pentagonal hull.  Section 3.7 treats nonpolygonal regions
with $n(R)=8$ and hexagonal hull.  Section 3.8 treats
$n(R)=7$, pentagonal hull.  After Section 3.8 we may assume
that all standard regions are polygons.
Section 3.9 eliminates the configurations $\Sminus$, $\Sfour$
of upright quarters.  Section 3.10 treats the case of six vertices,
from Section 3.2.

\subhead 3.5. Linear programs\endsubhead

Following the methods outlined in III, we obtain linear programming
bounds on the scores of the 5094 planar maps generated by the
classification algorithm.  In all but 180 cases, the bound is
less than $8\,\pt$.  According to the ordering of planar maps
in the archive \cite{H1}, we are left with the following cases.

{\obeylines
\parskip=0pt
\parindent=0pt
\hbox{}

(pentagon)=$\{2,45,263,273,274,275,288,290,302,\ldots,2321,2351\}$ (149 cases)
(hexagon)= $\{59,70,104,111,129,130,131,146,226,248,$
\hfill\quad\quad $250,251,256,296,302,303,310,363,368,385\}$ (20 cases)
(heptagon)=$\{27,36,46,70,75\}$ (5 cases)
(octagon)=$\{6,7,8,12,14,16\}$ (6 cases)

}

The linear program uses all the linear inequalities from 
Sphere Packings III for
quasi-regular tetrahedra and quad clusters (Section 10 group 1, group 3,
group 5, Prop. 4.1, 4.2, 4.3).  It uses the linear relations
among solid angles and dihedral angles.  It uses the bounds
from 
the arrays {\tt scoreFace}, {\tt squanderFace}, and {\tt vertexAdjustment}.
It does not use any of the linear inequalities such as
III.10 group 5, or III.A.6 
that depend on special conditions being met.  It does not
use any of the inequalities that assume the quad cluster has a special
structure.  It does not use, for example, the inequalities for
the octahedral case of the quad cluster in III.A.3.

As with all these calculations, code and details can be found in \cite{H1}.

If the linear program associated with a particular planar map
is expressed as $\max \sum\sigma_i$
such that $A\,x\le b$, we can determine bounds on the variables
$x_j$ as follows.   Define a new linear program $\max c\,x$
such that $A\,x\le b$, and $\sum\sigma_i\ge 8\,\pt$.   If
$c\,x = x_j$ or $-x_j$, or in fact, any linear combination of
variables, then solving the linear program gives a bound on $x_j$
or any linear combination $c\,x$.  The bounds on $x_j$ obtained
by this method will be called the {\it LP-bounds} on a variable.  
We write $\LPmin(x_j)\le x_j\le\LPmax(x_j)$ for the linear programming
bounds that follow in this manner.  The LP-bounds depend on the
system of linear equations $A\,x\le b$ that we use, although this
is not indicated in the notation.

\subhead 3.6. A pentagonal hull with $n=8$
\endsubhead

The next few sections treat the nonpolygonal standard regions illustrated
in IV.4.4.
In this standard cluster, the aggregate of the octagonal cluster and
the quasi-regular tetrahedron forms a pentagonal cluster.  Let $P$
denote this aggregated cluster.

We have bounds $(-3.03+1)\,\pt$ on what is scored and $10.916\,\pt$ on
what is squandered.  There is no other exceptional cluster 
$10.916\,\pt+t_5>\squander$.  We make the substitutions
$4.896\mapsto 10.916$, $-1.03\mapsto -2.03$ in {\tt squanderFace}
and {\tt scoreFace} respectively.  With these changes we rerun
the linear programs for the 149 remaining pentagonal planar maps.
This time through, all but 1 case has a bound less than $8\,\pt$.

The one case (pentagon\#575) can be described at the planar map
with 15 triangles obtained by deleted all five triangles at a vertex
from the icosahedral planar map.  The property of this triangulation
that we exploit is that it has three triangles at every vertex of
the pentagonal region. Let $v$ be the vertex of the pentagonal region
that is a corner of the quasi-regular tetrahedron in the aggregate $P$.
The four quasi-regular tetrahedra at $v$ give
$t_8+\tLP(4,0,2\pi-2(1.153)) > \squander$.  

\subhead 3.7. $n=8$, hexagonal hull\endsubhead

We treat the two cases from IV.4.4 that have a hexagonal hull.
One can be represented as a hexagonal cluster with an enclosed vertex
that has height at most $2.51$ and distance at least $2.51$ from
each corner of the hexagon.  The other is represented as a hexagonal cluster
with an enclosed vertex of height at most $2.51$, but this time
with distance less than $2.51$ from one of the corners of the
hexagonal cluster.

The argument for the case $n=8$ with hexagonal hull is similar
to the argument of Section 3.7.
Make the substitutions $7.414\mapsto10.916$, and $-2.06\mapsto-4.12$
in {\tt squanderFace} and {\tt scoreFace}, respectively.
The constant $-4.12$ comes from IV.4.4.
Rerunning the 20 remaining hexagonal cases with these changes brings
all but one planar map under $8\,\pt$. 

The planar map that remains is (hex\#131).  The planar
map is shown.  There are 16 triangles and one hexagonal standard
region.  

\smallskip
\gram|2|3.7|diag3.7.ps|
\smallskip

In discussing various maps, we let $v_i$ be the vertices of
the clusters, and we set $y_i = |v_i|$ and $y_{ij}= |v_i-v_j|$.
If $F$ is a standard region, subregion, cluster, or subcluster,  
we let $\dih_{F,i}$ denote
the dihedral angle of $F$ along $(0,v_i)$.  The subscript $F$
is dropped, when there is no great danger of ambiguity.

We have $\LPmax(\dih_8)\le1.647$ and 
$\LPmax(y_9+y_{8,9}+y_{8,12}+y_{12})\le 8.57$.
An interval arithmetic
 upper bound on $y_{9,12}$, 
subject to these two constraints
 is $2\sqrt2$.  Similarly, $y_{10,11}\le 2\sqrt2$.
There are two flat quarters $(0,v_5,v_{10},v_{11})$ and
$(0,v_8,v_9,v_{12})$.  The vertex $v_{13}$ enclosed over the hexagonal
region cannot be over the flat quarters.  
Also, $\LPmax(y_{11,12}+y_{9,10}) \le 4.378$.  
So $y_{11,12},y_{9,10}\le 2.378$.
Geometric considerations as in Section 3.1 give
$y_{13,i}\le 2.97$, $i=9,10,11,12$.

Case 1.  $y_{i,13}\ge 2.51$, for $i=9,10,11,12$.
Interval calculations give 
$$
\align
\dih(0,v_{13},v_9,v_{12}) &< 1.77,\\
\dih(0,v_{13},v_{10},v_{11}) &< 1.77,\\
\dih(0,v_{13},v_{11},v_{12}) &<1.23 + 0.7(y_{11,12}-2),\\
\dih(0,v_{13},v_9,v_{10})&< 1.23 + 0.7(y_{9,10}-2)
\endalign
$$
This gives the contradiction
$$2\pi > 2 (1.23)+0.7(0.378) >\sum\dih = 2\pi.$$

Case 2.  $y_{i,13}\ge 2.51$, for $i=9,10,11$, $y_{12,13}\in[2,2.51]$.
We have $y_{11,12},y_{9,10}\le 2.378$.  Geometric considerations
as in Section 3.1 give $y_{11,13}\le 2.97$, $y_{10,13}\le 3.16$,
$y_{9,13}\le 2.96$.
We also obtain by interval arithmetic if $y_{13}\le 2.28$,
that 
$$2\pi > 1.427+1.356 + 1.684+1.806 > \sum\dih = 2\pi.$$
So $y_{13}\ge 2.28$.

The LP-lower bound on the score of the hexagonal cluster is
$-4.93\,\pt$.  The simplex $(0,v_{11},v_{12},v_{13})$ is
possibly a flat quarter.  Since we do not have further information
about it, we use the worst case bound of 0.
We use interval arithmetic worst case bounds on the score of
$(0,v_{13},v_9,v_{10})$, $(0,v_{13},v_{10},v_{11})$, 
$(0,v_{13},v_{11},v_{12})$, and $(0,v_{13},v_{12},v_9)$.
$$\sigma<-4.15\,\pt-1.19\,\pt-2.52\,\pt-0 < -4.93\,\pt
	-8(0.01561)-2 (0.014).$$





\subhead 3.8. $n=7$, heptagonal hull\endsubhead

We treat the two cases from IV.4.4 that have a heptagonal hull.
One can be represented as a heptagon with an enclosed vertex
that has height at most $2.51$ and distance at least $2.51$ from
each corner of the heptagon.  The other is represented as a heptagon
with an enclosed vertex of height at most $2.51$, but this time
with distance less than $2.51$ from one of the corners of the
heptagon. They are illustrated in IV.4.4.

Make the substitutions $4.896\mapsto 9.932$, $-1.03\mapsto -3.03$
in {\tt squanderFace} and {\tt scoreFace}.  There is no
other exceptional region $(9.932+4.896)\,\pt>\squander$.  
With these changes, of the 149 pentagonal
planar maps, all but one of the linear programs give a bound
under $8\,\pt$.  The case that remains (pentagon\#575) is the one
that occurred in Section 3.6.  It is the planar map obtained by 
removing the five triangles around a vertex
from an icosahedron.

We treat the case (pent\#575).  Let $v_{12}$ be the vertex enclosed
over the pentagon.  We let $v_1,\ldots,v_5$ be the five corners
of the pentagon.
Break the pentagon into
five simplices along $(0,v_{12})$:  $S_i = (0,v_{12},v_i,v_{i+1})$.

Case 1.  {\it The vertex $v_{12}$ has distance at least $2.51$ from
the five corners of the pentagon.}

We have $\LPmax(y_i)\le2.168$, for $i=1,2,3,4,5$.  The penalty
to switch the pentagon to a pure $\vor_0$ score is at most $5\xiG$.
There cannot be two flat quarters because then
$$|v_{12}+>\Cal E(S(2,2,2,2.51,2\sqrt2,2\sqrt2),2.51,2.51,2.51)>2.51.$$
Suppose there is one flat quarter, $|v_1-v_4|\le2\sqrt2$.
Geometric considerations give $|v_{12}-v_i|\le3.39$, $i=1,2,3,4$.
The four simplices around $v_{12}$ squander
	$$2\pi(0.2529)-3(0.1452) - 0.29 >15.5\,\pt.$$
The 15 quasi-regular tetrahedra squander $>4(0.55)\,\pt$ by
Lemma III.5.3.  We have
$$15.5\,\pt - 5\xiG + 4(0.55)\,\pt>\squander.$$

Now assume there are no flat quarters.  Disregard all vertices
but $v_1,\ldots,v_5,v_{12}$.  If a vertex $|v_i-v_{12}|>2.51$,
deform $v_i$ as in Section IV.4.9 until $|v_i-v_{i-1}|=|v_i-v_{i+1}|=2$,
or $|v_i-v_{12}|=2.51$.  Move $v_{12}$ keeping $v_1,\ldots,v_5$
fixed and not increasing $|v_{12}|$ until $|v_{12}-v_i|=2.51$.

Two of these three are necessarily adjacent, say 
$|v_3-v_{12}|=|v_4-v_{12}|=2.51$.  The vertex $v_{12}$ is not
enclosed over $(0,v_3,v_4,v_5)$ or $(0,v_1,v_2,v_3)$\footnote"*"{prove!}
so if the third vertex is $\ne v_1$, we can continue the deformations.
So assume $|v_1-v_2|=2.51$.

There are three cases. In all three cases 
$|v_i-v_{12}|=2.51$, for $i=1,3,4$.
$$
\align
(a)\quad &|v_1-v_2|=|v_2-v_3|=|v_4-v_5|=|v_5-v_1|=2,\\
(b)\quad &|v_{12}-v_2|=2.51,\quad |v_4-v_5|=|v_5-v_1|=2,\\
(c)\quad &|v_{12}-v_2|=|v_{12}-v_5|=2.51.
\endalign
$$
Case (c) follows from interval calculations
$$
\sum\tau_0 \ge 2\pi(0.2529) - 5 (0.1452) -5 \xiG - 4(0.55)\,\pt > \squander.
$$
Case (a) does not exist,
$$2\pi=\sum\dih < 1.51 + 4(1.16).$$
In case (b), we have again
	$$2\pi(0.2529)-5 (0.1452)$$
if the dihedral angles of both angles along $(v_{12},v_5)$ are
$>0.65$.  We have
	$$(2\pi-1.49)(0.2529) - 3 (0.01452),$$
if an angle $<0.65$ (so that by an interval calculation the sum is
less than $1.49$).  Either way we squander $>\squander$.


Case 2. The vertex $v_{12}$ has distance at most $2.51$ from
the vertex $v_1$ and distance at least $2.51$ from the others.
The five simplices squander
\footnote"*"{Add calculation at the end and stabilize!}
$$\sum\tau_0(S_i) \ge 2\pi (0.2529) - 2 (0.239)-3 (0.1452)\ge0.675.$$
The LP-upper bound on what is squandered by the subregion
is $0.644$.  So if the penalty is $\le 0.031$, we are done.

An LP-lower bound on the score of the subregion is $-0.234$.
Since $-0.4339$ is less than this the lower bound, 
the configuration $\Sminus$
does  not occur.  Similarly, since $-0.25$ breaks the lower bound,
$\Sfour$ does not occur (IV.3.7, IV.3.8).  
If there are no loops, then the penalty
$\le 2(0.008)< 0.031$, and we are done.

Suppose that there is a loop in context $(4,2)$.  
Again the score is less than the LP-lower
bound, showing that this does not occur.
$$s_7+ (-0.2 +0.1141)< -0.234.$$ The constants come from table
IV.5.11 and IV.4.4.
If there is a loop other than $(4,2)$ and $(4,1)$, we break the
bound again
	$$t_7 + (\tau\text{LP}-D(n,k)) \ge 0.5498 +0.1 \ge 0.644.$$
We conclude that the only loops have context $(4,1)$.

Let $(0,v_{13})$ be the upright diagonal of a loop $(4,1)$.  The
vertices of the loop are not $v_2,v_3,v_4,v_5$ with $v_{12}$ enclosed
over $(0,v_2,v_5,v_{13})$ by IV.3.6.  The vertices of the loop
are not $v_2,v_3,v_4,v_5$ with $v_{12}$ enclosed over $(0,v_1,v_2,v_5)$
because this would lead to a contradiction
$$y_{12}\ge \Cal E(S(2,2,2,2.51,2.51,3.2),2.51,2.51,2)>2.51,$$
or
$$y_{12}\ge \Cal E(S(2,2,2,2.51,2.51,3.2),2,2.51,2)>2.51.$$
We get the same contradiction unless $(v_1,v_{12})$ is an edge of some
upright quarter of every loop of type $(4,1)$.

We represent the two situations as follows.  In the first
(case 2-a) there is one upright diagonal $(0,v_{13})$ with anchors to
$v_1,v_2,v_3,v_{12}$.  The anchored simplex $(0,v_{13},v_3,v_{12})$ is
not a quarter, but the other three anchored simplices around $(0,v_{13})$
are.  In the second (case 2-b) there are two
upright diagonals $(0,v_{13})$ and $(0,v_{14})$.  The anchors
of $(0,v_{13})$ are $v_1,v_2,v_3,v_{12}$.  The anchors of 
$(0,v_{14})$ are $v_1,v_5,v_4,v_{12}$.  The 
anchored simplices $(0,v_{14},v_4,v_{12})$ and $(0,v_{13},v_3,v_{12})$
are not quarters.

Case 2-a.  If $(0,v_{12},v_1,v_2)$ is a flat quarter, the penalty
is at most
$$0.008 + \xiG + 2 \xiV < 0.031,$$
or 
$$0.008 + 2 \xiG' + \xiV < 0.031,$$
and the bound is established.  So $y_{2,12}\ge 2\sqrt2$.

The penalty is at most $0.008 + 3 \xiG < 0.0471$.
The exceptional region squanders at least $0.675- 0.0471\ge 0.6279$.
The LP-upper bound on $\dih(0,v_1,v_2,v_5)$ is $2(1.38)=2.76$.
This implies, by an interval calculation that
$y_{12,5}\le2\sqrt2$.  
(This interval calculation uses that $y_{2,12}\ge2\sqrt2$.)
If $y_{13}\ge2.57$, then $\LPmax(y_5)\le 2.168$.
We have interval arithmetic bounds 
$$
\align
\dih(0,v_1,v_{12},v_5)&>1.19,\\
\dih(0,v_1,v_2,v_{13})&>0.8,\\
\dih(0,v_1,v_{12},v_{13})&>0.8.
\endalign
$$
This gives a contradiction $2.76>\sum\dih>2.79$.

Case 2-b.  $\LPmax(\dih(0,v_1,v_2,v_5))\le2.891$.
The 4 upright quarters at $v_1$ give by interval arithmetic
$$\sum_{(4)}nu < 0.252 (2.891) - 4 (0.24) = -0.231468.$$
By IV, the simplex $(0,v_3,v_4,v_{12})$ scores $<Z(3,2)$.  
The exceptional
region gives
$$-0.231468 +Z(3,2)< -0.234.$$
The other anchored simplices
\footnote"*"{Show these are negligible even if $\dih>2.46$}
at $v_{13},v_{14}$ score $\le0$. Either way,
we are done.




\subhead 3.9. $\Sminus$, $\Sfour$\endsubhead

If we have the configuration $\Sminus$, there is only one exceptional
region ($10.12\,\pt+t_5>\squander$) (Section IV.3.7).  
The configuration with
six standard regions around a vertex from Section 3.2 does not occur
because the five qrtets in the configuration squander $>6\,\pt$,
giving $(6+10.12)\,\pt>\squander$.

Based on Section IV.3.7,
we make the substitutions 
$$
\align
(4.896,7.414,9.932,10.916)&\mapsto (10.12,10.12,10.12,10.12)\\
(-1.03,-2.06,-3.03,-3.03)&\mapsto (-7.83,-7.83,-7.83,-7.83),
\endalign
$$
in {\tt squanderFace} and {\tt scoreFace} respectively.
All 180 linear programming bounds are under $8\,\pt$ when
these changes are made.

The configuration $\Sfour$ requires more work.  Make the
substitution
$$(-2.06,-3.03,-3.03)\mapsto (-4.52,-4.52,-4.52)$$
in {\tt squanderFace}.  The configuration $\Sfour$ does not
appear in a pentagon.  Rerun the other 31 linear programs with
these changes.  All but 7 cases are under $8\,\pt$.
They are (hexagon 59, 70, 131), (heptagon 36, 75), (octagon 8, 14).

The following argument eliminates these 7 cases.  Let $v_1$ and
$v_2$ be the corners of $\Sfour$ that are on the large gap.
For the remaining 7 cases we solve the linear programming
problem maximizing $|v_1|+|v_2|$ subject to all the usual
constraints, together with the additional constraint that the
score of the decomposition star is at least $\squander$.  Since
we do not know how $\Sfour$ is situated with respect to the
planar map there are several cases involved here.  We find
that $|v_1|+|v_2|<4.6$ in every case.

The LP-lower bound on the score of the exceptional cluster is
$-0.31547$.
We solve the linear programming problem involving just the four
quarters of $\Sfour$ maximizing the dihedral angle $\alpha$ of
the gap subject to the constraint that the score of the four
quarters is at least $-0.31547$.  The solution gives $\alpha<1.743$.
This is inconsistent with an interval calculation, which asserts $\alpha>1.78$.
Thus, the decomposition star scores less than $8\,\pt$.

\subhead 3.10.  Six vertices\endsubhead

Turn to the case of six vertices around a vertex, introduced in Section 3.2.
We claim that the aggregate of the six subclusters around the
vertex squanders at least $11.839\,\pt$.
In fact, if the dihedral angle of the pentagon is $\le1.32$, we get
$$\tLP(5,0,2\pi-1.153)+1.47\,\pt + t_5 > 11.839\,\pt.$$
If the angle is $\ge1.32$, we get
$$\tLP(5,0,2\pi-1.32)+t_5>11.839\,\pt.$$

The score of the aggregate is $<s_5+\sLP(5,0,2\pi-1.153)<-3.03\,\pt$.

Rerun the linear programs for the six remaining octahedral cases making
the substitution $10.916\mapsto11.839$ in {\tt squanderFace}. 
All but one linear program gives a bound under $8\,\pt$.  The planar
map that remains is a triangulation of an octagon (octagon \#7).  
Although it might
seem there are several ways to combine them, there is actually only
1 viable planar map.  All others produce a vertex of degree 3 surrounded
by 3 triangles, which is inadmissible.   De-aggregating the arrangement
gives a planar map with one pentagonal region.  We run the usual
linear programming optimization on this planar map and obtain
a bound under $8\,\pt$.

\subhead 3.11. Type $(n,k)=(5,1)$\endsubhead

Five quarters around a common upright diagonal in a pentagonal
region can certainly occur.  We claim that any other upright
diagonal with five anchors leads to a decomposition star that
squanders $\ge\squander$.  In fact, the only other possible
context is $(5,1)$, and there is squanders $>10.72\,\pt$, by IV.5.11.
It scores $<-6.79\,\pt$.

If this appears in an octagon, we squander $>10.72\,\pt+D(5,1)>\squander$.
If this appears in a heptagon, we squander
$$>10.72\,\pt + D(4,1) + 0.55\,\pt > \squander,$$
because there must be a vertex that is not a corner of the heptagonal
cluster.
It cannot appear on a pentagon.  If it appears on a hexagonal region,
it squanders $10.72\,\pt+D(3,1)>11.91\,\pt$ and scores $<-6.79\,\pt$.
Weaker constants than these are used in Section 3.7 with the
conclusion that only one particular planar map is possible (hex\#131).
So here too it is enough to consider that case.  Make the
substitutions $7.414\mapsto 11.91\,\pt$ and $-2.06\mapsto -6.79\,\pt$
in {\tt squanderFace} and {\tt scoreFace}.  The LP-bound obtained
with these changes is $>\squander$.

\subhead 3.12.  Summary.\endsubhead
We summarize some of the results obtained up to this point.
The worst pathologies have now been eliminated.
Let $D^*$ be a decomposition star scoring $\ge8\,\pt$ that
has an exceptional region.  Then

1.  The exceptional region is a polygon with $n=5,6,7$, or $8$ sides.

2.  There are no $\Sfour$ or $\Sminus$ configurations (even in
	the original decomposition star before anything is erased).

3.  The planar map is one of the 180 cases obtained in Section 3.5.

4.  If an upright quarter with 5 anchors appears, it must have
	context $(5,0)$ and lie over a pentagonal standard region.


\head 4.  Refined Linear Programming Bounds\endhead

\subhead 4.1. Internal structures\endsubhead

From the results of Section 3, we know that every standard region
is a polygon, and there are no $\Sminus$ or $\Sfour$ configurations
of exceptional regions.  We describe linear programming arguments that
reduce the list of possibilities from 180 to 67.  We do this by
separating each planar map into a  number of subcases according to
the detailed structure of the exceptional cluster.
We list the structures and afterward we give an extended explanation
of what the diagram means.


\vfill\eject
\gram|5.3|4.1|diag41.ps|
\vfill\eject

The possibilities are listed in the diagram only up to symmetry
by the dihedral group action on the polygon.  We do not prove
the completeness of the list, but its completeness can be seen
by inspection, in view of the comments that follow here and
in Section 4.2.

Each enclosed vertex in the diagrams is an upright diagonal.  
Four types of edges are drawn. (1) an edge from the endpoint
of an upright diagonal to its anchor. (2) the diagonal of
a  flat quarter. (3) dashes representing edges of anchored
simplices that are not quarters.  (4) the edges of the polygon.
The type of the edge can be inferred from 
the diagram.  The diagonals of erased upright quarters and
masked flat quarters are not drawn.

The upright diagonals in the diagrams are surrounding by anchored
simplices.  (We know from Section 4 that there are no configurations
$\Sfour$ or $\Sminus$, and we have erased $\Splus$, taking a penalty.)
Dashes have length $[2.51,3.2]$.  Except in the two cases
of a the hexagonal regions, where dashes join opposite corners,
the length of a dash is at least $2\sqrt2$.

In one case involving pentagonal clusters, one case involving
hexagonal clusters, and all cases involving heptagonal and octagonal
clusters, some upright quarters have been erased that incur a penalty.
In these cases we have indicated an upper bound on the penalty.
By erasing those we have, there does not remain more than one upright
diagonal. In the lists of heptagonal and octagonal regions, no
upright quarters remain.


We will continue our description of the diagrams in Section 4.2, where
the penalties are described.

\subhead 4.2. Penalties\endsubhead



Erasing an upright quarter of compression type gives $\xiG$
and one of Voronoi type gives $\xiV$.
We take the worst possible penalty.  It is at most $n\xiG$
in an $n$-gon.  If there is a masked flat quarter, the
penalty is only $2\xi_V$ from the two 
upright quarters along the flat quarter.  We note in this connection
that both edges of a polygon along a flat quarter lie on
upright quarters, or neither does.

If an upright diagonal appears enclosed over a flat quarter,
the flat quarter is part of a loop with context $(4,1)$, for
a penalty $2\xi'_\Gamma+\xi_V$.  This is smaller than the
penalty obtained from a loop with context $(4,1)$, when the
upright diagonal is not enclosed over the flat quarter
	$$\xi_\Gamma + 2\xi_V.$$
So we calculate the worst-case penalties under the assumption
that the upright diagonals are not enclosed over flat quarters.

A loop of context $(4,1)$ gives $\xi_\Gamma+2\xi_V$ or
$3\xi_\Gamma$.  A loop of context $(4,2)$ gives $2\xi_\Gamma$
or $2\xi_V$. 


The penalties are obtained as follows.  If we erase $\Splus$,
the is a penalty of $0.008$. (Recall that the penalty is 0 if
it masks a flat quarter.)  This is dominated by the penalty
$3\xi_\Gamma$.

Suppose we have an octagonal standard region.  We claim
that a loop does not occur in context $(4,2)$.
If there are at most three vertices that are not corners
of the octagonal cluster, then there are at most 12 quasi-regular
tetrahedra, and the score is at most 
$$s_8 + 12\,\pt<8\,\pt.$$
Assume there are more than three vertices that are not corners
of the octagonal cluster.
We squander
$$t_8+ (\DLP(4,2)-D(4,2))+4\tlp(5,0) > \squander.$$
As a consequence, there are at most 2 upright diagonals.

If there are
three upright diagonals, we have
$$\tlp(4,1)+\tlp(4,2) + 0.55\,\pt > \squander.$$
So there are at most $2$ upright diagonals and at most $6$ quarters:
$\pi_0\le 6\xi_\Gamma$. Let $f$ be the number of flat quarters
This leads to
	$$
	\pi_0 = \cases 6\xiG, & f=0,1,\\
				   4\xiG+2\xiV, & f=2,\\
					2\xiG+4\xiV, & f=3,\\
					0, & f=4.
			\endcases
	$$
The 0 is justified by a parity argument.  Each upright quarter occurs
in a pair at each masked flat quarter.  But there are an odd
number of quarters along the upright diagonal, so no penalty at
all can occur.

Suppose we have a heptagonal standard region.  For the same reason
as given for octagons, there are at most 2 upright diagonals.
If there are at most 2 upright diagonals, and both have context
$(4,1)$, then
	$$
	\pi_0 = \cases 6\xiG, & f=0,\\
				 4\xiG+2\xiV, & f=1,\\
				2\xiG+4\xiV, & f=2.
			\endcases
	$$
If there are at most 2 upright diagonals, and one has context
$(4,2)$, then
	$$
	\pi_0 = \cases 5\xiG, & f=0,1,\\
				3\xiG + 2\xiV, & f=2,\\
				\xiG + 4 \xiV, &f = 3.\\
			\endcases
	$$
The constant $0.0114$ will be explained in Section 4.3.
This gives the bounds used in the diagrams of cases.

\subhead 4.3 Additional inequalities for flats\endsubhead

We discuss the context $(2,1)$ that occurs when two upright
quarters in the $Q$-system lie over a flat quarter.  
Let $(0,v)$ be the upright diagonal, and assume that $(0,v_1,v_2,v_3)$
is the flat quarter, with diagonal $(v_2,v_3)$.
Let $\sigma$ denote the score of the upright quarters
and other anchored simplex lying over the flat quarter.

\proclaim{Lemma} $\sigma\le \min(0,\vor_0)$.
\endproclaim

\demo{Proof}  The bound of $0$ is established in II and F.

By F.4.7.5, if $|v|\ge 2.69$, then the upright quarters satisfy
$$\nu < \vor_0 + 0.01 (\pi/2-\dih)$$
so the upright quarters can be erased.  Thus we assume
without loss of generality that $|v|\le 2.69$.

We have $$|v|\ge\Cal E(S(2,2,2,2.51,2.51,2\sqrt2),2,2,2)>2.6.$$
If $|v_1-v_2|\le 2.1$,  or $|v_1-v_3|\le 2.1$, then
	$$|v|\ge \Cal E(S(2,2,2,2.1,2.51,2\sqrt2),2,2,2)>2.72,$$
contrary to assumption.  So take $|v_1-v_2|\ge 2.1$ and $|v_1-v_3|\ge2.1$.
Under these conditions we have the interval calculation
$\nu(Q) < \vor_0(Q)$ where $Q$ is the upright quarter.
\qed
\enddemo

\proclaim{Remark}  If we have an upright diagonal enclosed over
a masked flat quarter in the context $(4,1)$, then there are
3 upright quarters.  By the same argument as in the lemma, the
two quarters over the masked flat quarter score $\le\vor_0$. The
third quarter can be erased with penalty $\xiV$.
\endproclaim

\proclaim{Lemma} $\mu < \vor_0 +0.0268$ for all flat quarters.
\endproclaim

\demo{Proof} This is an interval calculation.\qed\enddemo

We introduce some scoring conventions that are somewhat different
from those presented in IV.3.10.
Flat quarters in the $Q$-system are scored by $\mu$.  Erased
flats with $y_1\ge 2.2$ are scored with $\vor_0$ and erased
flats with $y_1\le 2.2$ are scored with $\vor_0-0.0114$.
We will use these scoring conventions in the inequalities of
Section 4.5.

The penalty we introduced in Section 4.2 for masked flat quarters
is $\xiG+2\xiV$.  The penalties introduced in Section 4.2 do not
take the constant $0.0114$ into account.  We show here that the
bounds on the penalties from Section 4.2 still hold even if we
adjust some flat quarters by $0.0114$ according to these new
scoring conventions.  

Begin with the context $(4,1)$. In Section 4.2, we erase with
penalty $\pi_0=2\xiV+\xiG$.
If $|v|\ge 2.696$, the penalty is actually
$3\xiV < \pi-0.0114$.  If $|v|\in[2.57,2.696]$, the penalty
is $\xiV+\xiG' - 0.004131 < \pi_0-0.0014$.  
The constant $0.004131$ comes from IV.$A_{11}$.
If $|v|\le2.57$, then $\Cal E(S(2,2,2,2.51,2.2,2.57),2,2,2)>2\sqrt2$,
so this case does not occur.

The only other context to consider is a loop in context $(4,2)$.  We
have seen in Section 4.3 that this context does not occur in
octahedra.  This context is never erased in pentagons or hexagons.
This leaves heptagons.  If there is one flat quarter and a context $(4,2)$,
we have penalty
	$$3\xiG + 2\xiV < 5\xiV - 0.0114.$$
If there are two flats we have
	$$\xiG+4\xiV < 3\xiG+2+2\xiV-0.0114.$$
If there are three flats, we have
	$$\xiG+4\xiV+0.0114.$$
This completes the justification of the penalties in Diagram 4.1.


\subhead 4.4. Dihedral angles\endsubhead

We gave in Section 4.1 the possible decompositions of a standard cluster
into smaller pieces.  We give lower an upper bounds in this Section
on the dihedral angles at each vertex of each of the smaller pieces.

We divide the edges into 6 categories, 0--6. In 0--3 it is assumed
that neither endpoint of the edge is the endpoint $v$ of an upright
diagonal $(0,v)$.  In 4--6 we assume that one of the endpoints is
the endpoint of an upright diagonal.

0.  The edge has length $[2,2.51]$.  

1.  The edge has length $[2.51,2\sqrt2]$.

2.  The edge has length $\ge2.51$.

3.  The edge has length $\ge2\sqrt2$.

4.  The edge has length $[2,2.51]$.

5.  The edge has length $\ge2.51$.

The categories are not exclusive.  In general we consider an
edge as belonging to the most restrictive category that the information
of the diagrams of 4.1 permit us to conclude that it lies in.

The following chart summarizes the bounds.  The dihedral angle is
computed along the first edge.

$$
\matrix 
y_5  & y_6 & y_4 & \dih_{\min} & \dih_{\max} \\
%	(* first section 5&6 short. *)
    0&0&1&1.153&2.28\\
    0&0&3&1.32&2\pi\\
% 
%   (* second section 5 short& 6 long *)
    0&1&0&0.633&1.624\\
    0&1&1&1.033&1.929\\
    0&1&2&1.033&2\pi\\
    0&1&3&1.259&2\pi\\
% 
%   (* 5 long& 6 short *)
    1&0&0&0.633&1.624\\
    1&0&1&1.033&1.929\\
    1&0&2&1.033&2\pi\\
    1&0&3&1.259&2\pi\\
%
%   (* 5 long& 6 long *)
    1&1&0&0.817&1.507\\
    1&1&1&1.07&1.761\\
    1&1&2&1.07&2\pi\\
    1&1&3&1.23&2\pi\\
%
%   (* upright diagonal& 5 drawn& 6 drawn *)
    4&4&0&0.956&2.184\\
    4&4&1&1.23&\pi\\
    4&4&2&1.23&\pi\\
    4&4&3&1.416&\pi\\
%
%   (* upright diagonal at vertex 3& 5 drawn *)
    4&0&4&0.633&1.624\\
    4&0&5&1.033&2\pi\\
    4&1&4&0&1.381\\
    4&1&5&0.777&2\pi\\
%
%   (* upright diagonal at vertex 2& 6 drawn *)
    0&4&4&0.633&1.624\\
    0&4&5&1.033&2\pi\\
    1&4&4&0&1.381\\
    1&4&5&0.777&2\pi.
\endmatrix
$$


\subhead 4.5. Other inequalities \endsubhead

A number of other inequalities hold on the various smaller pieces of
the decomposition of standard clusters into the finer pieces
described in Section 4.2.


\subhead Flat Quarters \endsubhead
The following inequalities hold for flat quarters.
In these inequalities
$$\sigma = \cases \mu&\text{flat quarter in the Q-system}\\
				  \vor_0&y_1\ge2.2,\\
				  \vor_0-0.0114&y_1\le2.2,\\
	\endcases
$$
and $\tau_\sigma = \sol\zeta\pt-\sigma$.
The fourth edge is the diagonal.

$$
\align
- \dih_2& + 0.35 y_2 - 0.15 y_1 - 0.15 y_3 + 0.7022 y_5 - 0.17 y_4 > -0.0123,\\
- \dih_3& + 0.35 y_3 - 0.15 y_1 - 0.15 y_2 + 0.7022 y_6 - 0.17 y_4 > -0.0123,\\
\dih_2& - 0.13 y_2 + 0.631 y_1 + 
	   0.31 y_3 - 0.58 y_5 + 0.413 y_4 + 0.025 y_6 > 2.63363,\\
\dih_3& - 0.13 y_3 + 0.631 y_1 + 
	   0.31 y_2 - 0.58 y_6 + 0.413 y_4 + 0.025 y_5 > 2.63363,\\
-\dih_1& + 0.714 y_1 - 0.221 y_2 - 0.221 y_3 + 
	   0.92 y_4 - 0.221 y_5 - 0.221 y_6 > 0.3482,\\
\dih_1& - 0.315 y_1 + 0.3972 y_2 + 0.3972 y_3 - 
	   0.715 y_4 +  0.3972 y_5 + 0.3972 y_6 > 2.37095,\\
- \sol& - 0.187 y_1 - 0.187 y_2 - 
	   0.187 y_3 + 0.1185 y_4 + 0.479 y_5 + 0.479 y_6 > 0.437235,\\
\sol& + 0.488 y_1 + 0.488 y_2 + 
	   0.488 y_3 - 0.334 y_5 - 0.334 y_6 > 2.244,\\
- \sigma& - 0.145 y_1 - 0.081 y_2 - 0.081 y_3 - 
		0.133 y_5 - 0.133 y_6 > -1.17401,\\
\sigma& + 0.153 y_4 + 0.153 y_5 + 0.153 y_6 < 1.05382,\\
\sigma& < 0.00005\\
\tau_\sigma& > 1.189 \,\pt.
\endalign
$$

\subhead Upright Quarters \endsubhead

In the following equations, the simplex is an upright quarter.
The first edge is the diagonal.

$$
\align
 \dih_1& - 0.636 y_1 + 0.462 y_2 + 0.462 y_3 - 0.82 y_4 + 0.462 y_5 + 
 0.462 y_6 > 1.82419,\\
 - \dih_1& + 0.55 y_1 - 0.214 y_2 - 0.214 y_3 + 1.24 y_4 - 0.214 y_5 
 - 0.214 y_6 > 0.75281,\\
 \dih_2& + 0.4 y_1 - 0.15 y_2 + 0.09 y_3 + 0.631 y_4 - 0.57 y_5 + 0.23 y_6
  > 2.5481,\\
 - \dih_2& - 0.454 y_1 + 0.34 y_2 + 0.154 y_3 - 0.346 y_4 + 
0.805 y_5 > -0.3429,\\
 \dih_3& + 0.4 y_1 - 0.15 y_3 + 0.09 y_2 + 0.631 y_4 - 0.57 y_6 + 0.23 y_5 
  > 2.5481,\\
 - \dih_3& - 0.454 y_1 + 0.34 y_3 + 0.154 y_2 - 0.346 y_4 + 
0.805 y_6 > -0.3429,\\
 \sol& + 0.065 y_2 + 0.065 y_3 + 0.061 y_4 - 0.115 y_5 - 
0.115 y_6 > 0.2618,\\
 - \sol& - 0.293 y_1 - 0.03 y_2 - 0.03 y_3 + 0.12 y_4 + 
0.325 y_5 + 0.325 y_6 > 0.2514,\\
 -\nu& - 0.0538 y_2 - 0.0538 y_3 -0.083 y_4 - 0.0538 y_5 - 
0.0538 y_6 > -0.59834,\\
 \nu& \le 0,\\
 \tau_\nu& - 0.5945 \,\pt > 0.\\  % (* part4sec2.cc:527,528 *)
\endalign
$$

We also include in this group, the inequalities IV.$A_2$, IV.$A_3$
for upright quarters.

\subhead Miscellaneous Inequalities\endsubhead

Depending on the types of the edges $y_5$, $y_6$, $y_4$, there
are additional inequalities that hold on the simplices in
the decomposition of Section 4.2.  We list the types of
the edges $y_5$, $y_6$, $y_4$ as a triple $(a,b,c)$ and
then list the additional inequalities that hold.

$$\align
(0, 0, 3)&\quad 
\dih-0.372 y_1 +0.465 y_2 +0.465 y_3 + 0.465 y_5 + 0.465 y_6 >4.885,\\ 
%
%
(0, 1, 1)&\quad 
0.291 y_1 -0.393 y_2 -0.586 y_3 +0.79 y_4 -0.321 y_5 -0.397 y_6 -\dih  
	<  -2.47277,\\
%
(0, 1, 2)&\quad 
0.291 y_1 -0.393 y_2 -0.586 y_3 +0. y_4 -0.321 y_5 -0.397 y_6 -\dih  
	<  -4.45567,\\
%
(0, 1, 3)&\quad  
0.291 y_1 -0.393 y_2 -0.586 y_3 +0. y_4  -0.321 y_5 -0.397 y_6 -\dih  
	<  -4.71107,\\
%
(1, 0, 1)&\quad 
0.291 y_1 -0.586 y_2 -0.393 y_3 +0.79 y_4 -0.397 y_5 -0.321 y_6 -\dih  
	<  -2.47277,\\
%
(1, 0, 2)&\quad 
0.291 y_1 -0.586 y_2 -0.393 y_3 +0. y_4 -0.397 y_5 -0.321 y_6 -\dih  
	< -4.45567,\\
%
(1, 0, 3)&\quad 
0.291 y_1 -0.586 y_2 -0.393 y_3 +0. y_4  -0.397 y_5 -0.321 y_6 -\dih  
	< -4.71107,\\
%
%
(1, 1, 0)&\quad 
\tau_x -2.518 pt > 0,\\
	&\quad \sigma_x +1.03 pt < 0,\\ 
	&\quad - \sol -0.492 y_1 - 0.492 y_2 -0.492 y_3 +0.43 y_4 +0.038 y_5+0.038 y_6 
	> -2.71884,\\
	&\quad - \sigma_x -0.058 y_1 -0.105 y_2  -0.105 y_3 -0.115 y_4 -0.062 y_5 -0.062 y_6
		> -1.02014 ,\\
	&\quad 0.115 y_1 -0.452 y_2  -0.452 y_3 +0.613 y_4 -0.15 y_5 -0.15 y_6 -\dih  < -2.177,\\
%
(1, 1, 1)&\quad 
0.115 y_1 -0.452 y_2 -0.452 y_3 +0.618 y_4 -0.15 y_5 -0.15 y_6 -\dih  < -2.17382,\\
%
(1, 1, 2)&\quad 
0.115 y_1 -0.452 y_2 -0.452 y_3 +0. y_4 -0.15 y_5 -0.15 y_6 -\dih < -3.725,\\
%
(1, 1, 3)&\quad 
0.115 y_1 -0.452 y_2 -0.452 y_3 +0. y_4 -0.15 y_5 -0.15 y_6 -\dih < -3.927,\\
%
(4, 4, 1)&\quad \sigma_x < 0\\
&\quad 0.47 y_1 -0.522 y_2 -0.522 y_3 +0.812 y_4 -0.522 y_5 -0.522 y_6 -\dih  < -2.82998,\\
%
(4, 4, 2)&\quad 
0.47 y_1 -0.522 y_2 -0.522 y_3 +0. y_4 -0.522 y_5 -0.522 y_6 -\dih  
		< -4.8681,\\
%
(4, 4, 3)&\quad 
0.47 y_1 -0.522 y_2 -0.522 y_3 +0. y_4 -0.522 y_5  -0.522 y_6 -\dih  
		< -5.1623,\\
%
(4, 0, 5)&\quad 
-0.4 y_3 +0.15 y_1 -0.09 y_2 -0.631 y_6-0.23 y_5-\dih < -3.9788,\\
%
(4, 1, 4)&\quad 
0.289 y_1 -0.148 y_2 -1.36 y_3 +0.688 y_4 -0.148 y_5 -1.36 y_6 -\dih  
	< -6.3282,\\
%
(4, 1, 5)&\quad 
0.289 y_1 -0.148 y_2 -0.723 y_3 -0.148 y_5 -0.723 y_6 -\dih  
	< -4.85746,\\
%
(0, 4, 5)&\quad  
-0.4 y_2 +0.15 y_1 -0.09 y_3 -0.631 y_5-0.23 y_6-\dih < -3.9788,\\
%
(1, 4, 4)&\quad  
0.289 y_1 -1.36 y_2 -0.148 y_3 +0.688 y_4 -1.36 y_5 -0.148 y_6 -\dih  
	< -6.3282,\\
%
(1, 4, 5)&\quad 
0.289 y_1 -0.723 y_2 -0.148 y_3 +0. y_4 -0.723 y_5 -0.148 y_6 -\dih  < -4.85746,\\
%
\endalign
$$



\subhead 4.6. Variable relations\endsubhead

We are now ready to describe an augmentation of the linear programs
that were presented in Section 3.5 to reduce to 180 cases.

Take the decompositions of Section 4.1 and generate all the
decompositions that can be obtained from them by the action
of the dihedral group on the vertices.  This gives
$n_5$ pentagonal cases $\Cal P_5$, $n_6$ hexagonal cases $\Cal P_6$,
$n_7$ heptagonal case $\Cal P_7$, and $n_8$ octagonal cases
$\Cal P_8$.  

For each of the 180 planar maps that remain, and for each
exceptional region of that planar map with $n$ sides, and a
case $p$ in $\Cal P_n$, we associate a new linear program.

For each face $F$ in $p$, introduce new variables 
$\sol_F$, $\sigma_F$, $\dih_{F,i}$, where $i$ indexes the
vertices of $F$.  Let $n(F)$ be the number of vertices of $F$.
These variables are related to the previous variables by the
linear inequalities
	$$
	\align
	\sol_F &= -(n(F)-2)\pi + \sum \dih_{F,i} \\
	\dih_i &= \sum_F \dih_{F,i},\\
	2\pi &= \sum_F\dih_{F,i},\\
	\sigma &\le \sum_{F,i}+\pi_0.
	\endalign
	$$
Here $\sigma$ is a variable representing the score of the standard
region and $\pi_0$ is a constant giving the penalty.
The sum of the dihedral angles at a vertex $i$ is $\dih_i$, the
dihedral angle of the standard region at $i$, if $i$ indexes
a vertex of the standard region.  The sum is $2\pi$, if $i$
indexes an enclosed upright diagonal.

For each $F$, we add the inequalities giving bounds on the
dihedral angles from Section 4.4, and all the relevant
inequalities relating edge lengths, scores, solid angles, and
dihedral angles from Section 4.5.  The inequalities selected depend
on the particularities of $F$.  

If the region $F$ is scored by $\vor_0$, we also add the inequalities
of III.A for $\vor_0$.  Recall that $\vor_0$ is defined by
a formula of the form (F.3.7)
$$\vor_0 =\sol\phi_0 +\sum \Adih - \sum 4\doct\quo.$$
$\phi_0$ is a constant.  We introduce a new variable $\Adih$
for each vertex of $F$ and two new variables $\quo$ for each edge
between adjacent vertices (one for each Rogers simplex along the
edge).  These variables are subject to the collection of linear
inequalities of III.A.4.

To write down the linear inequalities from III.A.4 for $\Adih$,
it is necessary to have bounds on the heights of vertices and on
the dihedral angles of $F$.  The LP-bounds on these variables
are used for these bounds.

The bound on the score of an octagonal cluster is $-4.12\,\pt$
(See IV.4.4).  Change {\tt scoreFace[7]=-3.09, scoreFace[8]=-4.12}.

Information about the internal structure of a exceptional cluster
gives improvements to 
the vertex adjustments $1.4\,\pt$ and $1.5\,\pt$.
The vertex adjustments contribute to the bound on the score
through the bound
$$\sum_{j=1}^4 \hbox{\tt squanderFace}[f_j]+\sum_{i=1}^k
	\hbox{\tt vertexAdjustment}[a_i]\tag 4.6.1$$
from Section 2.3.  Assume that 
at the vertex $v$ there are has four tetrahedra
and an exceptional, and that the exceptional cluster has a
flat quarter whose first edge lies along $(0,v)$. 
The 
calculations of Section 2.8 show that
the four quarters and exceptional squander at least $1.5\,\pt$.
If we know that there is no flat quarter (masked or unmasked)
whose first edge lies along $(0,v)$, then the four quasi-regular
tetrahedra at $v$ squander at least $1.5\,\pt$.
These are stronger inequalities than those used in Section 2.3,
because the adjustment is spread over a small region.
We can make similar improvements in {\tt vertexAdjustment[3]}.


If the planar map has more than one exceptional cluster, then the
weak bounds described in Section 3.5 are to be used on the other
exceptional clusters.

The LP problem in Section 3.5 has been augmented by all these
additional inequalities.  New linear programming bounds are obtained
by running these in all possible cases.  All but 67 
planar maps gave LP bounds
under $8\,\pt$.  (For each planar map that remains, often a number
of different patterns in $\Cal P_n$ lead to an LP-bound over $8\,\pt$.)
The 67 remaining planar maps are



{\obeylines
\parskip=0pt
\parindent=0pt
\hbox{}

(pentagon)=$\{45, 365, 374, 380, 383, 384, 577, 599, 603, 604, 607, 649,$ 
\quad\quad $651, 658, 661, 663, 670, 882, 893, 895, 896, 914, 947, 951,$
\quad\quad $953, 957, 993, 997, 1140, 1310, 1338, 1405, 1438, 1442, 1445, 1451,$
\quad\quad $1454, 1588, 1590, 1894, 2003, 2011, 2016, 2034, 2246, 2268,%
2300, 2302\}$ (48 cases)
(hexagon)= $\{59,70,129,131,146,248,250,256,296,$
\hfill\quad\quad$302,303,310,368,385\}$ (14 cases)
(heptagon)=$\{27,36,46,75\}$ (4 cases)
(octagon)=$\{14\}$ (1 case)

}

The numbers are consistent with the numbering in the archive \cite{H1}

\subhead 4.7. Octagons \endsubhead

All but one case is under $8\,\pt$.  It is (oct\#14).  The octagon
has two flat quarters.  The planar map and the decomposition of
the octagonal cluster are shown ($p(21)\in\Cal P_8$).

\smallskip
\gram|2|4.7|diag47.ps|
\smallskip

The LP-bounds on the heights gives\footnote"*"{These numbers
need to be rechecked. The argument in my scratch paper was done
by a somewhat different method.}
	$$
\align
y_2&\le 2.123,\quad y_6\le 2.119,\quad y_5\le 2.123,\quad\\
		y_{12}&\le 2.109,\quad y_{13}\le 2.217,\le y_{10}\le 2.109.
\endalign
$$
Let $\dih_{F,i}$ denote the dihedral angle along $(0,v_i)$ of the
hexagonal subcluster determined by the corners $i=2,6,5,12,13,10$.
The linear programming bounds on the variables $\dih_{F,i}$
are
$$
\align
\dih_{F,2}&\in[1.32,1.706],\quad
\dih_{F,6}\in[3.78,3.94],\quad
\dih_{F,5}\in[1.32,1.706],\quad\\
\dih_{F,12}&\in[1.26,1.642],\quad
\dih_{F,13}\in[4.96,5.097],\quad
\dih_{F,20}\in[1.26,1.642].
\endalign
$$
With these new bounds on $y_i$ and $\dih_{F,i}$, we revise the linear
inequalities on $\Adih_{F,i}$ from III.A.4.  The LP-bound is now
$<8\,\pt$.  

This argument gives improved bounds on $\Adih$ by determining
new LP-bounds on the $y_i$ and $\dih_{F,i}$ variables and using
these new bounds in the inequalities constraining $\Adih$.  We
will make use of this argument several times.  We refer to this
procedure as {\it updating the $\Adih$ variables}.  Improvements
in the bounds on $\Adih_{F,i}$ are possible in this way because 
$\Adih$ is a nonlinear function of $y_i$ and $\dih_{F,i}$.

\smallskip

\head 5.  Heptagons\endhead

\subhead 5.1. Cases\endsubhead

There are four planar maps that have an LP-bound over $8\,\pt$.
If we include the various internal structures $c$ of the heptagon,
there are 10 cases in all.
The cases $(n,c)=(27,3),(46,13),(46,21)$ drop under $8\,\pt$
when we update the $\Adih$ variables. This leaves 7 cases.
We show the planar maps and the possible positions of the flat quarters.

\gram|6|5.1|diag51.ps|


\subhead 5.2. $(n,c)=(36,7)$\endsubhead

For the case $(n,c)=(36,7)$, we use a branch and bound approach.
The set of inequalities A.5.2 hold for a quasi-regular tetrahedron
satisfying $y_4+y_5+y_6\le 6.25$.  
$$
\align
\sol& + 0.377076 y_1 + 0.377076 y_2 + 0.377076 y_3 - 0.221 y_4 - 
	 0.221 y_5 - 0.221 y_6 > 1.487741,\\
 -\sol&+ 0.221 y_4 + 0.221 y_5 + 0.221 y_6   > 0.76822,\\
\dih_1& + 0.34 (y_2 + y_3) - 0.689 y_4 + 0.27 y_5 + 0.27 y_6  > 2.29295,\\
\dih_2& + 0.34 (y_1 + y_3) - 0.689 y_5 + 0.27 y_4 + 0.27 y_6  > 2.29295,\\
\dih_3& + 0.34 (y_1 + y_2) - 0.689 y_6 + 0.27 y_4 + 0.27 y_5  > 2.29295,\\
 - \dih_1& + 0.498 y_1 + 0.731 y_4 - 0.212 (y_5 + y_6)  > 0.37884,\\
 - \dih_2& + 0.498 y_2 + 0.731 y_5 - 0.212 (y_4 + y_6)  > 0.37884,\\
 - \dih_3& + 0.498 y_3 + 0.731 y_6 - 0.212 (y_4 +y_5)  > 0.37884,\\
 - \sigma& - 0.109 (y_1 +y_2 +y_3) - 0.14135 y_4 - 
         0.14135 y_5 - 0.14135 y_6  > -1.5574737,\\
 - \sigma& - 0.419351 \sol - 0.2 (y_1 +y_2 +y_3) 
	- 0.048 (y_4 +y_5 +y_6) > -1.77465,\\
 \tau& - 0.0845696 (y_1 +y_2 +y_3) 
	- 0.163 (y_4 +y_5 + y_6)  > -1.48542.\\
 &y_4 +y_5 +y_6 \le 6.25,\\
\endalign
$$

The following inequalities hold if
$y_4+y_5+y_6\ge6.25$.  
$$\align
\sol& + 0.378 (y_1 + y_2 + y_3) 
	- 0.1781 (y_4 +y_5 + y_6)  > 1.761445,\\
- \sol& - 0.171 (y_1+y_2+y_3) 
	+ 0.3405 (y_4 + y_5 + y_6)  > 0.489145,\\
- \sigma& - 0.1208 (y_1 + y_2 + y_3) 
	- 0.0781 (y_4+y_5+y_6)  > -1.232965,\\
- \sigma& - 0.419351 sol - 0.2 (y_1 + y_2 +y_3) 
	+ 0.0106 (y_4 + y_5 + y_6)  > - 1.40816,\\
&y_4 +y_5 +y_6  \ge 6.25.\\
\endalign
$$

These two sets of inequalities are small
modifications of those in III.A.6.  We pick the quasi-regular
tetrahedra
	$$\{1,2,3,4,11,12,14,15,16,17\}.$$
Divide the domain into $2^{10}$ parts by imposing the
constraint $y_4+y_5+y_6\le6.25$ on some of the 10 quasi-regular
tetrahedra, and $y_4+y_5+y_6\ge6.25$ on the rest of the 10.
Depending on which constraint is picked, we add to the LP-problem
(described in Section 4.6) the inequalities in the first or
second set.  The
LP-bound on the score is less than $8\,\pt$ in every case.

\subhead 5.3. The case $(n,c)=(46,27)$\endsubhead

In this case there are 3 flat quarters 
$$(0,v_1,v_5,v_6),\quad (0,v_2,v_6,v_7),\quad (0,v_{10},v_5,v_{11}).$$

\proclaim{Lemma} $|v_6-v_{11}|\le 2\sqrt2$.
\endproclaim

\demo{Proof}  We have LP-bounds $\dih_5\le 1.68$,
$\dih_7\le 1.715$, $y_5+y_6+y_{11}\le 6.28$, 
$y_7+y_6+y_{11}\le 6.28$, $y_{6,7}+y_{7,11}\le 4.94$.

We have $\LPmin(\sigma_F)>-4.632\,\pt$, where
$\sigma_F$ denotes the score ($\vor_0$) of 
the simplex $(0,v_5,v_{11},v_7,v_6)$.

We have the interval calculation  bound
	$|v_6-v_{11}|\le 3.22$.
If $|v_6-v_{11}|\ge2\sqrt2$, then interval calculations give
the contradiction
$$
\align
-4.632\,\pt&<\sigma_F\le \vor_0(0,v_6,v_7,v_{11})+
	\vor_0(0,v_5,v_6,v_{11})\\
 		&<-1.52\,\pt -3.51\,\pt<-4.632\,\pt.
\endalign
$$
\qed
\enddemo

We draw the edge $(v_6,v_{11})$.
Suppose that there are no penalties.  We follow the approach of
Section 4.6 using penalty $\pi_0=0$ and the decomposition of
the heptagonal cluster shown in the first frame of the diagram.

\smallskip
\gram|1.5|5.3|diag53.ps|
\smallskip

The LP-bound is still over $8\,\pt$.  If we combine this
with the branch and bound method described in Section 5.2 for the
10 quasi-regular tetrahedra
	$$\{1,2,3,4,5,6,9,10,14,15\},$$
then the LP-bounds are in all $2^{10}$ under $8\,\pt$.

Now suppose that there is an upright diagonal that cannot be erased
without incurring penalties.  The 5-anchor cases were treated in
Section 3.11, so we assume it has only four anchors.  We claim that
there is no upright diagonal with context $(4,2)$.  Such would
score 
$$t_7 + \ZLP(4,2)-Z(4,2)= -0.171-0.2+0.1141< \LPmin(\sigma_F).$$

The only context that is possible is $(4,1)$.
One possibility is shown.  There is one masked flat quarter,
$(0,v_5,v_6,v_7)$.
Its LP-bound is $<8\,\pt$.

We claim that there is no other way to arrange the anchors.  If
there are at least 4 anchors and an anchor crosses the edge
$(v_4,v_7)$ then there are 5-anchors, contrary to assumption.  Similarly,
if an anchor crosses $(v_4,v_5)$ and $(v_5,v_7)$, there are five anchors.
If there is an upright diagonal over 
$(v_1,v_5,v_{10},v_{11},v_6)$, the context $(4,2)$ is created.
Thus, the case shown was the only remaining possibility.

\subhead 5.4.  Remaining cases\endsubhead

The cases $(n,c)=(27,7),(27,15),(75,2),(75,7),(75,11)$ need
to be filled in.

\head 6. Hexagons\endhead

\subhead 6.1. Quad clusters\endsubhead

There are 67 planar maps in Section 4.6 that require further attention.
14 of them are on the list of planar maps containing a hexagonal
standard region.  The 14 are (hex\#n), 
$$n= 59,70,129,131,146,248,250,256,296,302,303,310,368,385.$$

There is a quad cluster in each of the planar maps (hex\#n),
$$n=129,250,256,368.$$  III.A.1--III.A.4 describes four types of quad
clusters and derives various inequalities for each type.  The
four types are the octahedron, a pair of flat quarters with diagonal
between two opposite corners of the quad cluster, a pair of
flat quarters with diagonal between the other two opposite corners
of the quad cluster, and a $\vor_0$-scored quad with both diagonals
$\ge2\sqrt2$.

For each of the 4 planar maps, for each decomposition of the
hexagon that gave a bound over $8\,\pt$, and for each type of
quad cluster, we determine an LP-bound on the score.
The linear program combines the
inequalities of 4.6 with the inequalities adapted to the type
of quad cluster from Section III.A.1--III.A.4.  
In each case the LP-bound
is $8\,\pt$.

\subhead 6.2\endsubhead

There are ten more cases to be treated
	$$n=59,70,131,146,248,296,302,303,310,385.$$

\head 7. Pentagons\endhead

This section still needs to be written.  Treat the final $48$
cases here.


\vfill\eject
%\baselineskip=0.9\baselineskip
\parskip=0.2\parskip
\head References\endhead

\noindent
[F] S. Ferguson, T. Hales, A Formulation of the Kepler
    Conjecture, preprint

\noindent
[I] Thomas C. Hales, Sphere Packings I,
    Discrete and Computational Geometry, 17 (1997), 1-51.

\noindent
[II] Thomas C. Hales, Sphere Packings II,
    Discrete and Computational Geometry, 18 (1997), 135-149.

\noindent
[III] Thomas C. Hales, Sphere Packings III, preprint.

\noindent
[IV] Thomas C. Hales, Sphere Packings IV, preprint.
 
\noindent
[V] S. Ferguson, Sphere Packings V, thesis, University of Michigan,
    1997.

\noindent
[H1] Thomas C. Hales, Packings, \hfill\break
    \hfill{\tt http://www.math.lsa.umich.edu/\~%
    \relax hales/packings.html}

[H2] Thomas C. Hales, Remarks on the Density of Sphere Packings,
        Combinatorica, 13 (2) (1993) 181-197.

 

\newpage

\head Appendix. Calculations\endhead

Interval calculations are arranged according to the section
in which they appear.
Each inequality is accompanied by one or more reference numbers.
These identification numbers are needed to find further details
about the calculation in \cite{H1}.
Some of these inequalities were checked numerically
using a nonlinear
optimization package {\tt cfsqp}
\footnote"*"{\tt \quad www.isr.umd.edu/Labs/CACSE/FSQP/fsqp.html}
I thank the University of Maryland
for this software.

  Edge lengths whose bounds are not specified are assumed
to be between 2 and $2.51$.  The first edge of an upright
quarter is its diagonal.  The fourth edge of a flat
quarter is its diagonal.

\def\refno#1{\hfill {\tt (#1)}}
\parindent=0pt

\footnote""{\it The results of this paper depend on
various unverified inequalities listed in this appendix.
It is expected that they will be verified by computer
with interval arithmetic.}



\subhead Section A.2.5\endsubhead

$\dih>1.32$, if $y_4=2\sqrt{2}$.\refno{814398901}

$\tau>3.07\,\pt$, 
	if a flat quarter satisfies $\eta_{234},\eta_{456}\le\sqrt2$,
	$\dih\le1.32$.
	\refno{786190957}

$\tau_V>3.07\,\pt$,
	if a flat quarter satisfies $\eta_{456}\ge\sqrt2$,
	$\dih\le1.32$.
	\refno{369969833}

$\tau_V>3.07\,\pt$,
	if a flat quarter satisfies $\eta_{234}\ge\sqrt2$,
	$\dih\le1.32$.
	\refno{672373996}

$\tau_0>3.07\,\pt$,
	if a flat quarter satisifies $y_4\ge2.6$,
	$\dih\le1.32$.
	\refno{440293318}

$\tau_0>3.07\,\pt$,
	if a flat quarter satisifies $\eta_{456}\ge\sqrt2$,
	$\dih\le1.32$.
	\refno{50759223}

\subhead Section A.2.7%
\footnote"*"
{Some penalties and adjustments may need to be added to 2.7 and 2.8.}
\endsubhead

In Sections A.2.7 and A.2.8,
let $S_1,\ldots,S_5$ be 5 simplices arrranged around a common
edge $(0,v)$, with $|v|\in[2,2.51]$.   Let $y_i(S_j)$ be the
edges, with $y_1(S_j)=|v|$ for all $j$, and $y_3(S_j)= y_2(S_{j+1})$,
where the subscripts $j$ are extended modulo 5.
In Sections A.2.7 and A.2.8, $\sum\dih(S_j)\le2\pi$.

$\tau(S_1)+\tau(S_2)+\tau(S_4) > 1.4\,\pt$, 
	if $y_4(S_3),y_4(S_5)\ge2\sqrt2$.
	\refno{551665569}

$\tau(S_1)+\tau(S_2)+\tau(S_3) > 1.4\,\pt$,
	if $y_4(S_4),y_4(S_5)\ge2\sqrt2$.
	\refno{824762926}

$\tau(S_1)+\tau(S_2)+\tau_0(S_3) +\tau(S_4)> 1.4\,\pt
	+1.189\,\pt$,
	if $y_4(S_4)\in[2.6,2\sqrt2]$, 
	$y_4(S_5)\ge2.51$, $\dih(S_5)>1.32$,
	\refno{860887659}

$\tau(S_1)+\tau(S_2)+\tau_\mu(S_3) +\tau(S_4)> 1.4\,\pt
	+1.189\,\pt$,
	if $y_4(S_4)\in[2.51,2\sqrt2]$,  
	$y_4(S_5)\ge2.51$, $\dih(S_5)>1.32$.
	\refno{520708192}

$\tau(S_1)+\tau(S_2)+\tau(S_3) +\tau_0(S_4)> 1.4\,\pt
	+1.189\,\pt$,
	if $y_4(S_4)\in[2.6,2\sqrt2]$,  
	$y_4(S_5)\ge2.51$, $\dih(S_5)>1.32$.
	\refno{812812040}

$\tau(S_1)+\tau(S_2)+\tau(S_3) +\tau_\mu(S_4)> 1.4\,\pt
	+1.189\,\pt$,
	if $y_4(S_4)\in[2.51,2\sqrt2]$,  
	$y_4(S_5)\ge2.51$, $\dih(S_5)>1.32$.
	\refno{400364383}


\subhead Section A.2.8\endsubhead

$\tau(S_1)+\tau(S_2)+\tau(S_3) +\tau(S_4)> 1.5\,\pt$,
	if $y_4(S_5)\ge\sqrt2$.
	\refno{325738864}


$\tau(S_1)+\tau(S_2)+\tau(S_3) +\tau(S_4)+\tau_\mu(S_5)> 1.5\,\pt
	+1.189\,\pt$,
	if $y_4(S_5)\in[2.51,2\sqrt2]$.
	\refno{248940660}


$\tau(S_1)+\tau(S_2)+\tau(S_3) +\tau(S_4)+\tau_0(S_5)> 1.5\,\pt
	+1.189\,\pt$,
	if $y_4(S_5)\in[2.7,2\sqrt2]$.
	\refno{440713588}

$\tau(S_1)+\tau(S_2)+\tau(S_3) +\tau(S_4)+\tau_0(S_5)> 1.5\,\pt
	+1.189\,\pt$,
	if $y_4(S_5)\in[2.6,2\sqrt2]$, $y_1\in[2.2,2.51]$.
	\refno{306125712}


\subhead Section A.3.1\endsubhead

$\tau - 0.2529\dih > -0.3442$,
	if $y_1\in[2,2.51]$, and $\dih\ge1.51$.
	\refno{718074849}

$\tau_0  - 0.2529\dih > -0.1747$,
	if $y_1\in[2.3,2.51]$, $y_6\in[2\sqrt2,3.02]$, $1.26\le\dih\le 1.63$.
	\refno{378662012}

$\tau_0- 0.2529\dih > -0.2137$,
	if $y_1\in[2.3,2.51]$, $y_6\in[2.51,3.02]$, $1.26\le\dih\le 1.63$.
	\refno{465031274}

$\tau_0 - 0.2529\dih > -0.1371$,
	if $y_1\in[2.3,2.51]$, $y_5,y_6\in[2.51,3.02]$,
	$1.14\le\dih\le 1.51$.
	\refno{535502975}

\subhead Section A.3.7\endsubhead

$\dih>1.647$, if $y_2+y_3+y_5+y_6\le 8.57$, $y_4=2\sqrt2$.
	\refno{181416564}

$\dih<1.77$, if $y_4=2.51$, $y_5,y_6\ge2.51$.
	\refno{59900722}

$\dih< 1.23 + 0.7(y_4-2)$ if $y_5,y_6\ge 2.51$.
	\refno{87586361}

$\dih< 1.427$, if $y_1\in[2,2.28]$, $y_4=2.378$, $y_6\in[2.51,2.97]$.
	\refno{707727501}

$\dih< 1.356$, if $y_1\in[2,2.28]$, $y_4=2.378$, $y_5\in[2.51,3.16$],
		$y_6\in[2.51,2.96]$.
	\refno{523067541}

$\dih<1.684$, if $y_1\in[2,2.28]$, $y_4=2\sqrt2$,
	$y_5\in[2.51,2.97]$, $y_6\in[2.51,3.16]$.
	\refno{405500141}

$\dih<1.806$, if $y_1\le2.28$, $y_4=2\sqrt2$, $y_6\in[2.51,2.96]$.
	\refno{711127690}

$\vor_0 < -4.15\,\pt$, if $y_1\in[2.28,2.51]$,
	$y_5\in[2.51,2.97]$, $y_4\in[2.51,2\sqrt2]$, $y_6\in[2.51,3.16]$.
	\refno{422371452}

$\vor_0< -1.19\,\pt$, if $y_1\in[2.28,2.51]$, $y_5\in[2.51,3.16]$,
	$y_6\in[2.51,2.96]$.
	\refno{66204488}

$\vor_0 < -2.51\,\pt$, if $y_1\in[2.28,2.51]$, $y_4\in[2.51,2\sqrt2]$,
		$y_5\in[2.51,2.96]$.
	\refno{699970320}

$\vor_0 < 0$, fi $y_1\in[2.28,2.51]$, $y_3\in[2.51,2.97]$.
	\refno{548202396}







\subhead Section A.3.8\endsubhead

$\tau_0 > 0.2529\dih - 0.1452$, if $y_5,y_6\in[2.51,3.39]$,
		$y_2,y_3\in[2,2.168]$.
	\refno{323390408}

$\tau_0 > 0.2529\dih - 0.29$, if $y_4\in[2.51,2\sqrt2$,
	$y_5,y_6\in[2.51,3.39]$, $y_2,y_3\in[2,2.168]$.
	\refno{429841465}

$\dih<1.25$, if $y_1=y_4=2$, $y_5,y_6\ge 2.51$, $y_2,y_3\in[2,2.168]$.
	\refno{152858738}

$\dih< 1.61$, if $y_5,y_6\ge 2.51$, $y_2,y_3\in[2,2.168]$.
	\refno{750126217}

$\dih<1.16$, if $y_5,y_6\ge 2.51$, $y_4=2$, $y_2,y_3\in[2,2.168]$.
	\refno{60314528}

$\tau_0 > 0.02529 \dih - 0.1452$, if $y_2,y_3\in[2,2.168]$,
		$y_4=2$, $y_5=2.51$, $\dih>0.68$.
	\refno{334755730}

$\dih < 0.84$, if $y_2,y_3\in[2,2.168]$, $y_4=2$, $y_5=2.51$,
	and there is some $y_3'\in[2,2.168]$ such that
	$\dih(S(y_1,y_2,y_3',y_4,y_5,y_6))\le 0.65$.
	\refno{276313138}

$\dih>1.38$, if $y_4=2\sqrt2$, $y_2\in[2,2.168]$.
	\refno{777094391}

$\dih>1.19$, if $y_4=2.51$, $y_1,y_2\in[2,2.168]$.
	\refno{824874646}

$\dih_2>0.8$, if $y_2\in[2,2.168]$, $y_1\in[2.51,2.57]$.
	\refno{476071948}

$\nu < 0.252\dih_2 - 0.24$, for all upright quarters.
	\refno{131914556}

\subhead Section A.3.9\endsubhead

$\dih>1.78$, if 
	$y_4=3.2$, $y_1\in[2.51,2\sqrt2]$, $y_2+y_3\le4.6$.
	\refno{161665083}

\subhead Section A.4.3\endsubhead

$\nu < \vor_0$ if $y_1\in[2.6,2.69]$, $y_4\in[2.1,2.51]$.
	\refno{835135186}

\subhead Section A.4.4\endsubhead

The bounds on the entries of Table 4.4 hold by interval
calculations.
	\refno{219414783}

\subhead Section A.4.5\endsubhead

There are three sets of inequalities in Section 4.5: flat
quarters, upright quarters, miscellaneous inequalities.
These hold by interval calculations.
	\refno{249432024}, \refno{193857347}, \refno{413688513}


\subhead Section A.5.3\endsubhead

$\dih>1.715$, if $y_1+y_2+y_3\le6.28$, $y_5+y_6\le4.94$, $y_4=3.22$.
	\refno{96077248}

$\vor_0 < -3.51\,\pt$, if $y_4\in[2\sqrt2,3.22]$, $y_1+y_2+y_3\le 6.28$,
	$y_5,y_6\in[2.51,2\sqrt2]$.
	\refno{983938527}

$\vor_0 < -1.52\,\pt$, if $y_4\in[2\sqrt2,3.22]$,
	$y_1+y_2+y_3\le 6.28$, $y_5\in[2.51,2\sqrt2]$, $y_6\in[2,2.51]$,
		$y_5+y_6\le4.94$.
	\refno{648168371}


\bye
