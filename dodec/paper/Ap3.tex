%\magnification=\magstep1
\documentstyle{amsppt}

\topmatter
\parskip=\baselineskip
\baselineskip=1.1\baselineskip  % was 1.1
\parindent=0pt
\loadmsbm
\UseAMSsymbols
\raggedbottom
\hoffset=0.75truein
\voffset=0.5truein


\def\reft{\relax}  % items
\def\refz{\relax}
\def\bul{\noindent$\bullet$\quad }
\def\lb#1{\line{$\bullet$ #1 \hfill}}
\def\cir{\noindent$\circ$\quad }
\def\heads#1{\rightheadtext{#1}}
\def\doct{\delta_{oct}}
\def\pt{\hbox{\it pt}}
\def\Vol{\hbox{vol}}
\def\and{\operatorname{and}}
\def\Tan{\operatorname{Tan}}
\def\sol{\operatorname{sol}}
\def\quo{\operatorname{quo}}
\def\anc{\operatorname{anc}}
\def\cro{\operatorname{crown}}
\def\vor{\operatorname{vor}}
\def\octavor{\operatorname{octavor}}
\def\dih{\operatorname{dih}}
\def\arc{\operatorname{arc}}
\def\rad{\operatorname{rad}}
\def\if{\operatorname{if}}
\def\A{{\bold A}}
\def\squander{(4\pi\zeta-8)\,\pt}
\def\score{8\,\pt}
\def\Sfour{\Cal{\bold S}_4^+}
\def\Sminus{\Cal{\bold S}_3^-}
\def\Splus{\Cal{\bold S}_3^+}
\def\maxpi{\pi_{\max}}
\def\xiG{\xi_\Gamma}
\def\xiV{\xi_V}
\def\xik{\xi_\kappa}
\def\xikG{\xi_{\kappa,\Gamma}}

\def\rom{\uppercase\romannumeral}


\font\twrm=cmr8
\def\DLP{\operatorname{D}_{\hbox{\twrm LP}}}
\def\ZLP{\operatorname{Z}_{\hbox{\twrm LP}}}
\def\tLP{\operatorname{\hbox{$\tau$}LP}}  % 3 args small LP.
\def\tlp{\tau_{\hbox{\twrm LP}}}  % 2 args (p,q) tri, quad
\def\slp{\sigma_{\hbox{\twrm LP}}}  % 2 args (p,q) tri, quad
\def\sLP{\operatorname{\hbox{$\sigma$}LP}}
\def\LPmin{\operatorname{LP-min}}
\def\LPmax{\operatorname{LP-max}}
\def\geom{{\operatorname{g}}}
\def\anal{{\operatorname{an}}}



\def\sol{\operatorname{sol}}
\def\dih{\operatorname{dih}}
\def\V{\operatorname{V}}
\def\vol{\operatorname{vol}}
\def\Area{\operatorname{Area}}
\def\Per{\operatorname{Per}}
\def\rad{\operatorname{rad}}
\def\quo{\operatorname{quo}}

\def\R{{\Bbb R}}
\def\ldot{\cdot}
\def\S{{\Cal S}}
\def\del{\partial}
\def\B#1{{\bold #1}}
\def\tri#1#2{\langle#1;#2\rangle}
\def\ha{ \hangindent=20pt \hangafter=1\relax }
\def\ho{ \hangindent=20pt \hangafter=0\relax }
\def\i{I}
\def\was#1{\relax}

\def\diag|#1|#2|{\vbox to #1in {\vskip.3in\centerline{\tt Diagram #2}\vss} }
\def\v{\hskip -3.5pt }
% to place a fig file  align the bottom left corner at (0,-11).
% save as left justified.
% resize so that the box is just under 5 in width
\def\gram|#1|#2|#3|{
        {
        \smallskip
        \hbox to \hsize
        {\hfill
        \vrule \vbox{ \hrule \vskip 6pt \centerline{\it Diagram #2}
         \vskip #1in %
             \special{psfile=#3 hoffset=5 voffset=5 }\hrule }
        \v\vrule\hfill
        }
\smallskip}}

\def\today{\ifcase\month\or
    January\or February\or March\or April\or May\or June\or
    July\or August\or September\or October\or November\or December\fi
    \space\number\day, \number\year}




\bigskip

\centerline{{\bf Appendix 3. Pentagon Cases.}}

\bigskip

\proclaim{Theorem 1} The graph shown 
squanders more than the target.  \endproclaim

\gram|2.1|1|dia39.ps|  %  ceq0


We have the following inequalities by [L].\newline

$\dih_7(2) < 1.701$ \newline
$y(7) < 2.219$ \newline
$y(1) < 2.330$ \newline
$y(6) < 2.207$ \newline
$y(7) + y(1) + y(6) < 6.406$ \newline
$y(1) + y(6) < 4.399$ \newline

Using these constraints, we find by [I] that

$y(1,6)<3.179$ \newline

\proclaim{Lemma 1} Inequality (*) holds for this situation. \endproclaim

{\bf Proof:}


The only case which is different than the proof of (*) in 
the quad case is case 3.  We adjust the constants.

{\it Case 3:} If $x_1\le 2.25841, 3.1\ge y_4\ge 2.95,2.25841 \ge x_7 \ge 2$,

$$\align \vol(A)>-0.166 y_1 & - 0.114552 y_2 - 0.114552 y_3 + 0.115382 y_4 + 0.143 y_5 \\& + 0.143 y_6-0.153420(\dih_1(A)-1.76) +0.22. \endalign$$
$$\align \vol(B)>-0.166 y_7 & -0.051448 y_2 -0.051448 y_3 - 0.115382 y_4 + 0.143 y_8 \\& + 0.143 y_9 +0.55. \endalign$$



Optimizing with (*) gives

$\dih_3(1) < 1.681$ \newline
$\dih_1(1) < 1.917$ \newline
$y(1)<2.329$ \newline
$y(2)<2.207$ \newline
$y(3)<2.161$ \newline
$y(4)<2.280$ \newline
$y(5)<2T$ \newline
$y(2)+y(4)<4.307$ \newline
$y(3)+y(2)+y(4)<6.313$ \newline
$y(2)+y(5)<4.657$ \newline
$y(1)+y(5)+y(2)<6.757$ \newline

With these constraints, we have

$y(2,4) < 3.101$ \newline
$y(2,5) < 3.611$ \newline

by interval arithmetic.

Similarly, using all the above inequalities we have

$\vol(A) > .273$ \newline
$\vol(B)+\vol(C) > .50$ \newline

So $\vol(1)>0.773$,

Adding this constraint to the linear optimizer we find

$\dih_1(1)<1.875$ \newline
$dih_3(1)<1.634$ \newline
$y(1)<2.263$ \newline
$y(2)<2.134$ \newline
$y(3)<2.126$ \newline
$y(4)<2.169$ \newline
$y(2)+y(5)<4.616$ \newline
$y(1)+y(2)+y(5)<6.702$ \newline

There is a problem in this case in that the edge $y(5)$ is bounded above only by 2T.  This ``loose'' edge allows the volume of the simplices to be small.
We solve this by using $\mu$ when the edge is long and vol when it is short.

Case 1: Suppose $y(5)<2.26$.

Then 

$\vol(A)>.283$ \newline
$\vol(B)>.277$ \newline
$\vol(C)>.335$ \newline

so $\vol\{1\}>0.924.$  

This is adequate to push this graph over the target.

Case 2 : Suppose $y(5)>2.26.$

Then

$\mu(A)>0.0156$ \newline
$\mu(B)>0.0304$ \newline
$\mu(C)>0.0406$ \newline

so $\mu(1)>.0866$

Again, this is adequate to eliminate the graph.

This completes the proof of Theorem 1.

\qed



\bigskip



\proclaim{Theorem 2} The graph shown 
squanders more than the target.  \endproclaim

\gram|2.1|2|dia39.ps|  %  ceq2

We have 

$\dih_5(1) < 1.705$ \newline
$\dih_8(1) < 1.682$ \newline
$y(5)<2.207$ \newline
$y(12)<2.222$ \newline
$y(8)<2.206$ \newline
$y(13)<2.181$ \newline
$y(11)<2.199$ \newline

These imply 

$y(11,12)<3.188$ \newline
$y(12,13)<3.146$ \newline

which in turn imply

$\vol(A)>.278$ \newline
$\vol(B)>.277$ \newline
$\vol(C)>.362$ \newline

so $\vol(1)>0.917$ which pushes the graph over the target. \qed



\bigskip



\proclaim{Theorem 3} The graph shown 
squanders more than the target.  \endproclaim

\gram|2.1|3|dia39.ps|  %  ceq3


We have the following inequalities by [L].

$\dih_7(12) < 1.676$ \newline
$y(7) < 2.176$ \newline
$y(3) < 2.343$ \newline
$y(6) < 2.205$ \newline
$y(3,7)+y(6,7) < 4.965$ \newline
$y(2) + y(3) < 4.386$ \newline

Using these constraints, we find by [I] that

$y(3,6)<3.126.$

\proclaim{Lemma 1} Inequality (*) holds for this situation. \endproclaim

{\bf Proof:}

The reader may check that the edge and dihedral constraints are tighter than those used in the proof of (*) in Theorem 1.  Thus, (*) holds here as well. \qed

Optimizing with (*) gives

$\dih_4(1) < 1.685$ \newline
$\dih_3(1) < 1.828$ \newline
$y(1)<2.221$ \newline
$y(4)<2.165$ \newline
$y(9)<2.202$ \newline
$y(8)<2.230$ \newline
$y(3)<2.343$ \newline


With these constraints, we have

$y(1,9) < 3.161$ \newline
$y(1,8) < 3.370$ \newline

by interval arithmetic.

Similarly, using all the above inequalities we have

$\vol(A) > 0.277$ \newline
$\vol(B) > 0.272$ \newline
$\vol(C)> 0.360$ \newline

So $\vol(1)>0.909$, which pushes the graph over the target.

\qed


\bigskip



\proclaim{Theorem 4} The graph shown 
squanders more than the target.  \endproclaim

\gram|2.1|4|dia39.ps|  %  ceq4

We have 

$\dih_3(1) < 1.713$ \newline
$\dih_1(1) < 1.319$ \newline
$y(1)<2.243$ \newline
$y(2)<2.269$ \newline
$y(3)<2.169$ \newline
$y(4)<2.360$ \newline
$y(5)<2T$ \newline

These imply 

$y(2,4)<2.683$ \newline
$y(2,5)<3.390$ \newline

which in turn imply

$\vol(A)>.320$ \newline
$\vol(B)>.247$ \newline
$\vol(C)>.300$ \newline

so $\vol(1)>0.865$ which pushes the graph over the target. \qed



\bigskip



\proclaim{Theorem 5} The graph shown 
squanders more than the target.  \endproclaim

\gram|2.1|5|dia39.ps|  %  ceq6


We have the following inequalities by [L].

$dih_9(1) < 1.314$ \newline
$dih_1(1) < 1.684$ \newline
$y(9)<2.194$ \newline
$y(4)<2.314$ \newline
$y(1)<2.153$ \newline
$y(5)<2.174$ \newline
$y(10)<2T$ \newline


Case 1: Suppose $y(5)<2.26$.

Then 

$y(4,10)<2.697$ \newline
$y(4,5)<3.365$ \newline

$\vol(A)>.328$ \newline
$\vol(B)>.272$ \newline
$\vol(C)>.350$ \newline

so $\vol(1)>0.950$.  

This is adequate to push this graph over the target.

Case 2 : Suppose $y(5)>2.26$.

Then

$y(4,10)<2.807$ \newline
$y(4,5)<3.365$ \newline

$\mu(A)>0.0453$ \newline
$\mu(B)>0.0156$ \newline
$\mu(C)>0.0480$ \newline

so $\mu\{1\}>.1089$

Again, this is adequate to eliminate the graph.

This completes the proof of Theorem 5.

\qed

\bigskip



\proclaim{Theorem 6} The graph shown 
squanders more than the target.  \endproclaim

\gram|2.1|6|dia39.ps|  %  ceq7

We have 

$dih_2(1) < 1.577$ \newline
$dih_4(1) < 1.610$ \newline

$y(2)<2.123$ \newline
$y(6)<2.150$ \newline
$y(7)<2.217$ \newline
$y(11)<2.095$ \newline
$y(12)<2.101$ \newline

Use the proof of Theorem 2 to show $\vol(1)>0.900$ which pushes the graph over the target. \qed

\bigskip

\proclaim{Theorem 7} The graph shown 
squanders more than the target.  \endproclaim

\gram|2.1|7|dia39.ps|  %  ceq9


We have the following inequalities by [L].

$\dih_2(1) < 1.735$ \newline
$\dih_4(1) < 1.735$ \newline
$y(1)<2T$ \newline
$y(2)<2.199$ \newline
$y(3)<2.206$ \newline
$y(4)<2.199$ \newline
$y(5)<2T$ \newline


Case 1: Suppose $y(1)<2.26$.

Then 

$y(1,3)<3.254$ \newline
$y(4,5)<3.396$ \newline

$\vol(A)>.275$ \newline
$\vol(B)>.275$ \newline
$\vol(C)>.357$ \newline

so $\vol(1)>0.907$.  

This is adequate to push this graph over the target.

Now, interchange $y(1)$ with $y(5)$.  The solutions are the same.
Therefore, both $y(1)$ and $y(5)$ must be greater than 2.26.



Case 2 : Suppose $y(5),y(1)>2.26.$


Then

$y(1,3)<3.396$ \newline
$y(4,5)<3.396$ \newline

$\mu(A)>0.0313$ \newline
$\mu(B)>0.0313$ \newline
$\mu(C)>0.0336$ \newline

so $\mu\{1\}>.0962$

Again, this is adequate to eliminate the graph.

This completes the proof of Theorem 7.

\qed


\bigskip



\proclaim{Theorem 8} The graph shown 
squanders more than the target.  \endproclaim

\gram|2.1|8|dia39.ps|  %  ceq11

We have 

$\dih_2(1) < 1.637$ \newline
$\dih_4(1) < 1.625$ \newline
$y(1)<2.329$ \newline
$y(2)<2.130$ \newline
$y(3)<2.179$ \newline
$y(4)<2.189$ \newline
$y(5)<2.373$ \newline

These imply 

$y(1,3)<3.139$ \newline
$y(3,5)<3.145$ \newline

which in turn imply

$\vol(A)>.271$ \newline
$\vol(B)>.267$ \newline
$\vol(C)>.346$ \newline

so $\vol(1)>0.884$.

Add this inequality to the linear optimizer.

Then

$dih_6(12)<1.545$ \newline
$y(6)<2.078$ \newline
$y(1)<2.167$ \newline
$y(12)<2.076$ \newline

which imply

$y(1,12)<2.87.$ \newline

As these constraints easily satisfy the proof of (*) from
quad case 1, add (*) to push the graph over the target.

\qed

\bigskip

\proclaim{Theorem 9} The graph shown 
squanders more than the target.  \endproclaim

\gram|2.1|9|dia39.ps|  %  ceq13


We have 

$dih_3(1) < 1.519$ \newline
$dih_1(1) < 1.636$ \newline
$y(1)<2.206$ \newline
$y(2)<2.122$ \newline
$y(3)<2.191$ \newline
$y(4)<2.113$ \newline
$y(5)<2.077$ \newline

These imply 
so $\vol(1)>0.911$ by the proof of Theorem 2, which pushes the graph over the target. \qed

\bigskip

This completes the list of pentagon graphs.




\bye