%\magnification=\magstep1
\documentstyle{amsppt}

\topmatter
%\parskip=\baselineskip
\baselineskip=1.1\baselineskip  % was 1.1
%\parindent=0pt
\loadmsbm
\UseAMSsymbols
%\raggedbottom
\hoffset=0.75truein
\voffset=0.5truein


\def\reft{\relax}  % items
\def\refz{\relax}
\def\bul{\noindent$\bullet$\quad }
\def\lb#1{\line{$\bullet$ #1 \hfill}}
\def\cir{\noindent$\circ$\quad }
\def\heads#1{\rightheadtext{#1}}
\def\doct{\delta_{oct}}
\def\pt{\hbox{\it pt}}
\def\Vol{\hbox{vol}}
\def\and{\operatorname{and}}
\def\sol{\operatorname{sol}}
\def\quo{\operatorname{quo}}
\def\anc{\operatorname{anc}}
\def\cro{\operatorname{crown}}
\def\vor{\operatorname{vor}}
\def\octavor{\operatorname{octavor}}
\def\dih{\operatorname{dih}}
\def\arc{\operatorname{arc}}
\def\rad{\operatorname{rad}}
\def\if{\operatorname{if}}
\def\A{{\bold A}}
\def\squander{(4\pi\zeta-8)\,\pt}
\def\score{8\,\pt}
\def\Sfour{\Cal{\bold S}_4^+}
\def\Sminus{\Cal{\bold S}_3^-}
\def\Splus{\Cal{\bold S}_3^+}
\def\maxpi{\pi_{\max}}
\def\xiG{\xi_\Gamma}
\def\xiV{\xi_V}
\def\xik{\xi_\kappa}
\def\xikG{\xi_{\kappa,\Gamma}}

\font\twrm=cmr8
\def\DLP{\operatorname{D}_{\hbox{\twrm LP}}}
\def\ZLP{\operatorname{Z}_{\hbox{\twrm LP}}}
\def\tLP{\operatorname{\hbox{$\tau$}LP}}  % 3 args small LP.
\def\tlp{\tau_{\hbox{\twrm LP}}}  % 2 args (p,q) tri, quad
\def\slp{\sigma_{\hbox{\twrm LP}}}  % 2 args (p,q) tri, quad
\def\sLP{\operatorname{\hbox{$\sigma$}LP}}
\def\LPmin{\operatorname{LP-min}}
\def\LPmax{\operatorname{LP-max}}
\def\geom{{\operatorname{g}}}
\def\anal{{\operatorname{an}}}



\def\sol{\operatorname{sol}}
\def\dih{\operatorname{dih}}
\def\V{\operatorname{V}}
\def\vol{\operatorname{vol}}
\def\Area{\operatorname{Area}}
\def\Per{\operatorname{Per}}
\def\rad{\operatorname{rad}}
\def\quo{\operatorname{quo}}

\def\rom{\uppercase\romannumeral}

\def\R{{\Bbb R}}
\def\ldot{\cdot}
\def\S{{\Cal S}}
\def\del{\partial}
\def\B#1{{\bold #1}}
\def\tri#1#2{\langle#1;#2\rangle}
\def\ha{ \hangindent=20pt \hangafter=1\relax }
\def\ho{ \hangindent=20pt \hangafter=0\relax }
\def\i{I}
\def\was#1{\relax}

\def\diag|#1|#2|{\vbox to #1in {\vskip.3in\centerline{\tt Diagram #2}\vss} }
\def\v{\hskip -3.5pt }
% to place a fig file  align the bottom left corner at (0,-11).
% save as left justified.
% resize so that the box is just under 5 in width
\def\gram|#1|#2|#3|{
        {
        \smallskip
        \hbox to \hsize
        {\hfill
        \vrule \vbox{ \hrule \vskip 6pt \centerline{\it Diagram #2}
         \vskip #1in %
             \special{psfile=#3 hoffset=5 voffset=5 }\hrule }
        \v\vrule\hfill
        }
\smallskip}}

\def\today{\ifcase\month\or
    January\or February\or March\or April\or May\or June\or
    July\or August\or September\or October\or November\or December\fi
    \space\number\day, \number\year}


{\bf References.}

[H1] Hales, T.C. {\it Sphere Packings 1}, Dis. and Comp. Geom., 
{\bf 17} (1997), 1-51.

[H3] Hales, T.C. {\it Sphere Packings 3}, Preprint.

[H4] Hales, T.C. {\it Sphere Packings 4}, Preprint.

[1] Leech, John.  {\it The Problem of the Thirteen Spheres},  Math. Gazette,  
Feb. 1956, 22-23.

[2] L. Fejes T\'oth, {\it Uber die dichteste Kugellagerung}, Math 
Z., {\bf 48} (1943), 676-684.

[3] L. Fejes T\'oth, {\it Regular Figures}, Pergamon Press, Oxford-London-New York, 1964.


[4] C.A. Rogers, {\it The packing of equal spheres}, J. London Math. Soc.,
{\bf 3/8} (1958),609-620.

[5] D.J. Muder, {\it Putting the best face on a Voronoi polyhedron}, 
Proc. London Math. Soc., {\bf 3/56} (1988),329-348.

[6] D.J. Muder, {\it A new bound on the local density of sphere packings}, 
Discrete and Comp. Geom., {\bf 10} (1993), 351-375.

[7] W.-Y. Hsiang, {\it On the sphere packing problem and teh proof of Kepler's conjecture},
Int. J. Math., {\bf 4/5} (1993),739-831.

[8] K. Bezdek, {\it Isoperimetric inequalities and the dodecahedral conjecture}, Int. J. Math.,
{\bf 8/6} (1997), 759-780.

