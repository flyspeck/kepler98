\magnification=\magstep1
\parskip=\baselineskip
\centerline{\bf Testing Inequalities with CFSQP}
\bigskip

There are many inequalities that enter into the proof of the Kepler
conjecture.  They were proved rigorously by computer using 
interval arithmetic.  They have also been tested by 
numerical methods.  These numerical methods are not entirely reliable,
but in practice they are usually dependable, and they give a
partial independent confirmation that the methods of interval
arithmetic are functioning properly.

The computer program CFSQP has been used for the numerical tests of the
inequalities.  This program was developed by Craig Lawrence,
Jian L. Zhou, and Andr\'e L. Tits, at the Electrical Engineering
Department and Institute for Systems Research at the University of
Maryland.   It is available, free of charge, to non-profit
organizations.

The documentation describes CFSQP as ``a set of C functions for the
minimization of the maximum of a set of smooth functions (possibly
a single one, or even none at all) subject to general smooth
constraints.''

CFSQP has been extended through a series of functions relating to
the Kepler conjecture.  Each inequality arising in the proof of
the Kepler conjecture has been tested numerical by taking each
inequality $F(x)>0$, for $x \in D$, on a domain $D$, and expressing
it as a constrained minimization problem 
$$\min F(x)\hbox{ s.t. }   x\in D.$$
If the minimimum to this problem found by CFSQP is positive, it gives
numerical evidence that the inequality is true.

Many optimization problems of this type were considered during
the development stages of the Kepler conjecture.


This packet of materials contains

\noindent
the C++ code for the Kepler conjecture inequalities\hfil\break
the particular inequalities arising in Sphere Packings IV\hfil\break
the output from the inequality checking for Sphere Packings IV\hfil\break
the User's Guide for CFSQP, Version 2.5

See {\tt www.math.lsa.umich.edu/\~\relax hales/countdown/numerical/}.  There
you can find more extensive resources.
This directory contains all the inequalities appearing
in any of the papers as well as the output from testing
them.


\bye
